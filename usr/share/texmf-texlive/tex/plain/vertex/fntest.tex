\topmatter 
\runningtitle{TEST FILE} 
\runningname{HAL R. VARIAN}
\thanks{Thanks to various \TeX\ wizards and beta testers for aid in this
endeavor.  I also want to thank mom and dad, little brother, my teachers,
the National Science Foundation, Jim and Tammy Bakker, and anyone else I
can think of that will make this a nice long footnote for testing purposes.}
\title{A Test File for Ver\TeX} 
\author{Hal R. Varian}
\affil{Perversity of Michigan} 
\date{September, 1985} 
\version{\today} 
\abstract{This paper provides a torture test for Ver\TeX in order
to see if a few of its features work as advertised.}

\keywords{Ver\TeX, typesetting, desktop publishing}

\address{Prof. Hal R. Varian, Department of Economics,
University of Michigan, Ann Arbor, MI 48109, U.S.A.}

\endtopmatter
%\doublespace
\document

\noindent This document is provides a test of the various features of the
formatting\fnote{Here is the first footnote to see if the in-sentence
spacing works correctly.} package Ver\TeX. It doesn't really do much else. 
Ver\TeX\ allows easy formatting of papers.\fnote{Here is a footnote, in order
to see if the between-footnote spacing works correctly.}

\section Features of Ver\TeX

Among Ver\TeX's many features are subsections, proofs, etc.\fnote{Here is
yet another footnote!}  Here is another line to add to this sentence to
check the spacing.

\subsection This is a subsection

Here is a theorem:\fnote{Here is a very, very, very, very, very, very, very,
very, very, very, very, very, very, very, very, very, very, very, very,
very, very, very, very, very, very, very, very, very, very, very, very,
very, very, very, very, very, very, very, very, very, very, very, very,
very, very long footnote.} 

\proclaim Theorem.  Consider the following equation
 $$F(x) = \int_0^x f(t) dt.$$
It follows that $F'(x) = f(x)$.

\proof The fundamental theorem of calculus. \qed

That's about all there is to it.\fnote{Except for footnotes.  Except for
footnotes.  Except for footnotes.  Except for footnotes.  Except for
footnotes.  Except for footnotes.  Except for footnotes.  Except for
footnotes.  Except for footnotes.  Except for footnotes.  Except for
footnotes.  Except for footnotes.  Except for footnotes.  Except for
footnotes.} 

That's about all there is to it.  That's about all there is to it.  That's
about all there is to it.  That's about all there is to it.  That's about
all there is to it.  That's about all there is to it.  That's about all
there is to it.  That's about all there is to it.  That's about all there
is to it.  That's about all there is to it.  That's about all there is to
it.  

You can put quotations in Ver\TeX\ quite easily, such as:

\quote{Nothing ventured, nothing gained.  Nothing ventured, nothing gained. 
Nothing ventured, nothing gained.  Nothing ventured, nothing gained. 
Nothing ventured, nothing gained.  Nothing ventured, nothing gained. 
Nothing ventured, nothing gained.  Nothing ventured, nothing gained. 
Nothing ventured, nothing gained.}

Now that's a nice quotation, even though it is somewhat repetitive.

%Use in REStud.sty
%\Notes

\Refs
 
\ref \by{Afriat, S.} \yr{1967a} \paper{The Construction of a Utility
Function from Expenditure Data} \jour{International Economic Review}
\vol{8} \pages{67--77} \endref

\ref \by{Breeden, D. and R. Litzenberger} \yr{1978} \paper{Prices of
State-Contingent Claims Implicit in Option Prices} \jour{Journal of
Business} \vol{9} \pages{621--851} \endref

\ref \by{Varian, Hal R.} \yr{1986} \book{Intermediate Microeconomics}
\publ{W. W. Norton \& Co.} \publaddr{New York} \endref

%\PrintEndNotes

