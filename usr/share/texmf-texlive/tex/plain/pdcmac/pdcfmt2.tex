% pdcfmt2.tex 2.4 1995/04/06 -- macros for formatting

%%%@TeX-definition-file {
%%% filename       = "$texmf/tex/plain/pdcmac/pdcfmt2.tex",
%%% version        = "2.4",
%%% date           = "1995/04/06",
%%% package        = "pdcmac 1.0",
%%% author         = "P. Damian Cugley",
%%% email          = "damian.cugley@comlab.ox.ac.uk",
%%% address        = "Oxford University Computing Laboratory,
%%%                   Parks Road, Oxford  OX1 3QD, UK",
%%% codetable      = "USASCII",
%%% keywords       = "TeX, plain TeX, macros",
%%% supported      = "Maybe",
%%% abstract       = "Formatting macros for plain TeX documents.
%%%                   This file was generated by running
%%%                   plain TeX on pdcfmt2.dtx",
%%% copyright      = "Copyright (c) 1991-1995 P. Damian Cugley",
%%% copying        = "DO NOT DISTRIBUTE THIS FILE.
%%%                   Distribute pdcfmt2.dtx only as part of the
%%%                   package it came in.",
%%% dependencies   = "",
%%% }

\message{2.4 <pdc 1995/04/06>}

\toksdef\toksa=0
\chardef\other=12
\def\declareactivechar#1{%
    \toksa\expandafter{\verbatimplains\do#1}%
    \edef\verbatimplains{\the\toksa }%
    \catcode`#1\active
}
\def\verbatimplains{\do\\\do\{\do\}\do\_\do\$\do\#\do\&\do\%}
\def\verbatimactives{\do\-\do\`\do\'\do\~\do\^\do\ }
\chardef\other=12
\newtoks\everyverbatim
\bgroup \catcode`\-=13\catcode`\^=13 \catcode`\'=13 \catcode`\`=13 \toksa={\egroup
    \def\setupverbatim{%
        \frenchspacing
        \spaceskip0pt \xspaceskip0pt % use spacing of font
        \def\do##1{\catcode\lq##112 }\verbatimplains
        \def\do##1{\catcode\lq##1\active }\verbatimactives
        \let`\ttlq \let'\ttrq
        \let~\tttilde \let^\ttcircum \let-\ttminus
        \the\everyverbatim
    }
}\the\toksa
\def\ttlq{\lower0.125ex \hbox{\char18 }}
\def\ttrq{\lower0.125ex \hbox{\char19 }}
\def\tttilde{\lower0.5ex \hbox{\char`\~ }}
\def\ttcircum{\lower0.5ex \hbox{\char`\^ }}
\def\ttminus{-}
\def\defverbatim#1{%
    \ifcat\noexpand#1\noexpand~\else \declareactivechar#1 \fi
    \begingroup \uccode`\~=`#1 \uppercase{\toksa={\endgroup
        \def~{%
            \leavevmode
            \begingroup \tt \setupverbatim
            \catcode`#1\active \let~\endgroup
       }%
    }}\the\toksa
}
\defverbatim\|
\newtoks\everylisting
\def\listfile#1{
    \medskip
    \begingroup
        \parindent=0pt \parskip=0pt
        \def\par{\null\endgraf}\obeylines
        \setupverbatim \maketabstab
        \tt \the\everylisting
        \input#1
    \endgroup
    \medskip\noindent\ignorespaces
}
{\catcode`\^^I=\active
 \gdef\maketabstab{\catcode`\^^I\active \def^^I{\hskip 4em}}
}
\newif\ifnoindent
\newbox\parbox
\newdimen\parboxsep \parboxsep=1pc
\everypar={%
    \ifvoid\parbox
        \ifnoindent {\setbox0=\lastbox}\global\noindentfalse \fi
    \else
        {\setbox0=\lastbox}\global\noindentfalse
        \dp\parbox=0pt
        \hbox to 0pt{\hss \box\parbox \hskip\parboxsep}%
    \fi
}
\def\beginthe#1{%
    \begingroup\def\PDCFMTblockname{#1}%
}
\def\endthe#1{%
    \def\tmp{#1}%
    \ifx\tmp\PDCFMTblockname
        \endgroup
    \else
        \errmessage{You should have said \string\endthe{\blockname}}%
    \fi
}
\def\PDCFMTendenv#1{
    \smallskip
    \endthe{#1}
    \global\noindenttrue
}
\newdimen\envindent \envindent=1pc
\def\PDCFMTindent{%
    \ifdim\parindent>0pt
        \parindent
    \else
        \envindent
    \fi
}
\newtoks\everyquotation
\def\quotation{
    \smallskip
    \beginthe{quotation}
    \advance\leftskip\PDCFMTindent
    \noindenttrue
    \the\everyquotation
}
\def\endquotation{\PDCFMTendenv{quotation}}
\newtoks\everytextlist
\newif\ifnumbered
\newcount\textlistdepth  \textlistdepth=-1
\newcount\textlistcount
\def\textlist{%
    \par
    \beginthe{textlist}
    \advance\textlistdepth 1
    \textlistcount0
    \def\\{
        \smallskip\noindent
        \advance\textlistcount1
        \llap{%
            \ifnumbered
                \numberfordepth\textlistdepth\textlistcount
            \else
                \bulletfordepth\textlistdepth
            \fi\enspace}%
        \ignorespaces
    }
    \advance\leftskip\PDCFMTindent
    \the\everytextlist
}
\def\endtextlist{\PDCFMTendenv{textlist}}
\newtoks\everybullets
\def\bullets{\textlist \numberedfalse \the\everybullets}
\let\endbullets=\endtextlist
\def\bulletfordepth#1{%
    \ifcase#1 $\bullet$\or --\or $\circ$\else $\cdot$\fi
}
\newtoks\everynumbered
\def\numbered{\textlist \numberedtrue \the\everynumbered}
\let\endnumbered=\endtextlist
\def\numberfordepth#1#2{%
    \ifcase#1 \n{\number#2}.\or (\n{\number#2})\or
        ({\it\alphabetletter#2\/})\else (\romannumeral#2)\fi
}
\let\n\relax
\def\alphabetletter#1{%
    \ifcase#1 ??? \or a\or b\or c\or d\or e\else
        \xxxalphabetletter#1\fi
}
\def\xxxalphabetletter#1{%
    \ifcase#1 \or\or\or\or\or\or f\or g\or h\or i\or j\or
        k\or l\or m\or n\or o\or p\or q\or r\or s\or t\or u\or
        v\or w\or x\or y\or z\else !!!\fi
}
\newdimen\tagmaxwidth
\newtoks\everytagged
\def\tagged{%
    \par
    \beginthe{tagged}
    \let\\\TAG
    \ifdim\leftmargin=0pt
        \tagmaxwidth\PDCFMTindent
    \else
        \tagmaxwidth\leftmargin
    \fi
    \the\everytagged
    \ifdim\tagmaxwidth>\leftmargin
        \leftskip\tagmaxwidth \advance\leftskip-\leftmargin
    \fi
}
\def\endtagged{\PDCFMTendenv{tagged}}
\newtoks\everytag
\def\TAG{%
    \smallskip\noindent
    \setbox0=\hbox\bgroup        % matched by \TAGfinish
    \the\everytag\ignorespaces
    \futurelet\next\TAGtest
}
\def\TAGtest{%
    \ifcat\bgroup\noexpand\next
        \let\next\TAGgotbrace
    \else
        \let\next\TAGnobrace
    \fi \next
}
\def\TAGgotbrace{%
    \bgroup\aftergroup\TAGfinish
    \let\next
}
\def\TAGnobrace#1{%
    #1\TAGfinish
}
\def\TAGfinish{%
    \unskip\hskip0.5em\egroup       % matches \TAG
    \ifdim \wd0 < \tagmaxwidth
        \wd0=\tagmaxwidth
        \llap{\box0}%
    \else
        \hskip-\tagmaxwidth
        \unhbox0 \unskip\quad
    \fi
    \ignorespaces
}
\newtoks\everylines
\def\lines{
    \par
    \beginthe{lines}
    \nobreak\smallskip\hrule\nobreak\smallskip
    \obeylines
    \parindent=0pt \parskip=0pt
    \parfillskip=0pt plus 1fil
    \the\everylines
    \nobreak
}
\def\endlines{
    \nobreak\smallskip
    \endthe{lines}
    \hrule\smallskip
    \global\noindenttrue
}
\newcount\linenumber
\newcount\PDCFMTcount
\def\startlinenumbering{%
    \global\linenumber=0 \global\PDCFMTcount=5
    \everypar{\numberthisline}%
}
\def\continuelinenumbering{%
    \everypar{\numberthisline}%
}
\newtoks\everylinenum \everylinenum{\the\scriptfont0 }
\def\numberthisline{%
    \strut
    \global\advance\linenumber1 \global\advance\PDCFMTcount-1
    \ifnum\PDCFMTcount>0 \else
 \global\advance\PDCFMTcount 5
 \rlap{\the\everylinenum \kern\hsize\kern1em \the\linenumber}%
    \fi
}%
\def\linesskipped#1{%
    \strut \hskip20pt $\vdots$ \hskip20pt
    {\rm(\it #1 lines omitted\rm)}\par
    \advance\linenumber#1\relax
}
\outer\def\display{\obeylines\startdisplay}
\bgroup\obeylines \toksa={\egroup %
    \def\startdisplay#1^^M{%
        \catcode`\^^M=5 $$ #1 % matched by \enddisplay
        \displayindent\PDCFMTindent %
        \halign\bgroup##\hfil&&\quad##\hfil\cr %
    } %
}\the\toksa %
\def\enddisplay{\crcr\egroup$$}
\def\table{%
    $$                  % matching $$ is in \endtable
    \displayindent\PDCFMTindent
    \halign \bgroup
}
\let\endtable=\enddisplay
\newtoks\everybnf
\def\bnf{
    \nobreak\smallskip
    \beginthe{bnf}
    \advance\leftskip2\parindent \parindent=-\parindent
    \parskip0pt plus 1pt
    \rightskip=1\rightskip plus 3em
    \def\\{$\mid$}
    \def\>{\unskip\enspace$::=$\enspace\ignorespaces}
    \def|{`\begingroup\tt\setupverbatim\catcode`\|=13
            \def|{\endgroup'}}
    \def\{{$\lbrace$} \def\}{$\rbrace$}
    \the\everybnf
}
\def\endbnf{
    \smallskip
    \endthe{bnf}
    \global\noindenttrue
}
\def\<#1>{\leavevmode\hbox{$\langle${\it#1\/}$\rangle$}}
\newdimen\leftmargin
\newskip\headingtemp
\def\doheading#1#2#3#4{
    \ifdim\lastskip<#1\relax \removelastskip \vskip#1\relax \fi
    \ifdim \leftmargin>0pt
        \global\setbox\parbox=\vtop{%
            \hsize=\leftmargin \advance\hsize-\parboxsep
            \parindent=0pt
            \leftskip=0pt \rightskip=0pt plus 3em
            \hyphenpenalty=10000 \exhyphenpenalty=5000
            \strut#2#4#3
        }
    \else
        \begingroup
            \parindent=0pt \parfillskip=0pt plus 1fil
            \leftskip=0pt \rightskip=0pt plus0.25\hsize
            \hyphenpenalty=10000 \exhyphenpenalty=5000
            \strut#2#4#3
            \global\headingtemp=\baselineskip
            \par
        \endgroup
        \advance\headingtemp-\baselineskip
        \ifdim\headingtemp>0pt \nobreak \vskip 1.0\headingtemp \fi
        \smallskip
        \noindenttrue
    \fi
}
\def\newpageheading#1#2#3#4{
    \vfill\supereject % ensure no insertions still floating
    \null\vskip#1\relax
    \moveleft\leftmargin\vbox{
        \advance\hsize\leftmargin
        \parindent=0pt \parfillskip=0pt plus 1fil
        \leftskip=0pt \rightskip=0pt plus0.25\hsize
        \hyphenpenalty=10000 \exhyphenpenalty=5000
        #2\strut#4#3
        \global\headingtemp=\baselineskip
        \par
    }
    \advance\headingtemp-\baselineskip
    \advance\headingtemp\smallskipamount
    \vskip \headingtemp
    \noindenttrue
    \def\tmp{#4}
    \message{*\expandafter\TOCtrim\meaning\tmp.  }
}
\newcount\notecount
\def\note{%
    \global\advance\notecount+1
    \footnote{\number\notecount}%
}
\newdimen\footnoteparindent
\footnoteparindent=\parindent
\newtoks\everyfootnote
\catcode`\@=11
\def\footnote#1{\let\@sf\empty
  \ifhmode\edef\@sf{\spacefactor\the\spacefactor}\/\fi
  \footnotetextmark{#1}\@sf\vfootnote{#1}}
\def\vfootnote#1{\insert\footins\bgroup % matched by \@foot
    \interlinepenalty=\interfootnotelinepenalty
    \parindent=\footnoteparindent
    \leftskip=0pt
    \the\everyfootnote
    \splittopskip=\ht\strutbox \splitmaxdepth=\dp\strutbox
    \floatingpenalty=20000
    \indent\footstrut
    \ifdim\parindent>1em
        \llap{\footnotenotemark{#1}\enspace}%
    \else
        \footnotenotemark{#1}\enspace
    \fi
    \futurelet\next\fo@t
}
\def\@foot{\smallskip\egroup}
\catcode`\@=12
\def\footnotetextmark#1{$^{#1}$}
\def\footnotenotemark#1{$^{#1}$}
\def\today{\n{\number\day} \monthname\month\ \n{\number\year}}
\def\monthname#1{%
    \ifcase#1\or
        January\or February\or March\or April\or
        May\or June\or July\or August\or
        September\or October\or November\or December%
    \fi
}
\def\isodate{\n{\number\year}--\twodigits\month--\twodigits\day}
\def\twodigits#1{%
    \ifnum#1<10 0\fi \number#1%
}
\def\flushtop#1{%
    \leavevmode
    \begingroup
        \setbox0\hbox{#1}\setbox2\hbox{X}%
        \dimen0\ht2 \advance\dimen0-\ht0
        \raise\dimen0\box0
    \endgroup
}
\def\La{L\negthinspace\flushtop{a}}
\def\LaTeX{\La\TeX}
\def\superiorletter#1{%
    \flushtop{\the\scriptfont\fam \vphantom{x}\smash{#1}}%
}
\def\Mc{M\flushtop{\the\scriptfont\fam \b{c}}}
