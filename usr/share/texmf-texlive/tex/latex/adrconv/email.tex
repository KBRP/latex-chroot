%%
%% This is file `email.tex',
%% generated with the docstrip utility.
%%
%% The original source files were:
%%
%% adrconv.dtx  (with options: `tex,email')
%% Copyright (c) 2001 Axel Kielhorn
%% 
%% This file will generate fast loadable files from adrconv.dtx when
%% run through LaTeX or TeX.
%% 
%% This file is part of the adrconv bundle.
%% 
%% This file can be distributed and/or modified under the conditions of
%% the LaTeX Project Public License, either version 1.2 of the license
%% or (at you option) any later version.
%% The latest version of this license is in
%%   http://www.latex-procejt.org/lppl.txt
%% and version 1.2 or later ist part of all distributions of LaTeX
%% version 1999/12/01 or later.
%% 
%% The adrconv bundle consists at least of the files adrconv.dtx,
%% adrconv.ins and adrguide.tex.
%% 
%% You are NOT ALLOWED to change this file.
%% 
%% You are NOT ALLOWED to distribute this file without adrconv.dtx,
%% adrconv.ins or adrguide.tex.
\ifx\ProvidesFile\undefined\def\ProvidesFile#1[#2]{}\fi
\ProvidesFile{%
  email%
  .tex%
  }
  [2002/06/23 v1.2b LaTeX2e
  Interactive driver of BibTeX database to addressfile converter]
\catcode`\@=11
\newlinechar`\^^J
\message{%
  Now you have to typein the name of the BibTeX addressfile, you want
  to^^J%
  convert to script-address-file-format (without extension
  `.bib'):^^J%
  Geben Sie nun den Namen der BibTeX-Adressdatei ein, die sie in
  das^^J%
  Script-Adressdateiformat konvertieren wollen (ohne `.bib'):^^J%
  ^^J%
  addressfile=%
}
\def\skiplastspace#1 \@e@o@l@{#1}
\read-1 to \addressfile
\edef\addressfile{\expandafter\skiplastspace\addressfile\@e@o@l@}
\newwrite\auxfile
\immediate\openout\auxfile=\addressfile.aux
\immediate\write\auxfile{%
  \string\citation{*}^^J%
  \string\bibstyle{email}^^J%
  \string\bibdata{\addressfile}%
}
\immediate\closeout\auxfile
\message{%
  After running BibTeX rename file `\addressfile.bbl' to
  `\addressfile.adr'!^^J%
  Nach dem BibTeX-Lauf benennen Sie bitte die Datei `\addressfile.bbl'
  in^^J%
  `\addressfile.adr' um!^^J%
}
\ifx\@@end\undefined\let\@@end\end\fi
\@@end
\endinput
%%
%% End of file `email.tex'.
