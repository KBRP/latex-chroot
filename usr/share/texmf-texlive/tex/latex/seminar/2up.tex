%% BEGIN 2up.tex/2up.sty
%%
\def\fileversion{1.2}
\def\filedate{93/01/28}
%%
%% COPYRIGHT 1992, 1993 by Timothy Van Zandt, Timothy.VAN-ZANDT@insead.edu
%%
%% DESCRIPTION:
%%   2up.tex/2up.sty provides two-up printing for Generic TeX (e.g.,
%%   Plain, LaTeX, AmSTeX and AmS-LaTeX). It produces a standard dvi file,
%%   and does not involve an additional dvi or PostScript filter. It has a
%%   flexible interface for specifying paper size and layout.
%%
%% INSTALLATION:
%%   Put this file where your TeX looks for inputs, under the name 2up.tex.
%%   Name a copy 2up.sty to use as a LaTeX style option, or create a file
%%   2up.sty with the lines:
%%     \input 2up.tex
%%     \endinput
%%
%% DOCUMENTATION:
%%   Input 2up.tex, or include 2up as a LaTeX style option. There is a
%%   good chance you will get the desired layout. (But you will probably
%%   need to generate new font bitmaps to get high quality output.) See
%%   2up.doc, which might be appended to this file, for detailed
%%   documentation.
%%
%% This file may be distributed and/or modified under the conditions of
%% the LaTeX Project Public License, either version 1.2 of this license
%% or (at your option) any later version.  The latest version of this
%% license is in:
%% 
%%    http://www.latex-project.org/lppl.txt
%% 
%% and version 1.2 or later is part of all distributions of LaTeX version
%% 1999/12/01 or later.
%%
%%
%% CODE:
%
\csname TwoUpLoaded\endcsname
\let\TwoUpLoaded\endinput
%
\edef\TheAtCode{\the\catcode`\@}
\catcode`\@=11\relax
\message{\space\space v\fileversion\space\space \filedate\space\space <tvz>}
%
% Parameter registers:
\newdimen\@targetwidth
\newdimen\@targetheight
\newdimen\@sourcewidth
\newdimen\@sourceheight
\newdimen\pageseplength
\newdimen\pagesepwidth
\newdimen\pagesepoffset
\newif\if@sidebyside
\@sidebysidetrue
\newif\if@twosided
%
% Registers used by output routine.
\newif\if@leftpage
\@leftpagetrue
\newbox\@leftpage
\newbox\@rightpage
\newcount\@physicalpage
%
% Since pages are shipped out half as often:
\multiply\maxdeadcycles by 2
%
% Registers used only for booklet layout:
\begingroup
  \let\newcount\relax
  \gdef\booklet@registers{%
    \newcount\bookletpage
    \bookletpage=0
    \newcount\leftpagenumber
    \newcount\rightpagenumber
    \multiply\maxdeadcycles by 20}
\endgroup
%
% A useful extension of the \magstep macro.
\def\magstepminus#1{%
  \ifcase#1 \@m\or 833\or 694\or 579\or 482\or 401\fi\relax}
%
% \@targetwidth and \@targetheight are set to the *unmagnified* dimensions
% of the target page. \inv@targetmag is the inverse of the target
% magnification.
{\catcode`\p=12\catcode`\t=12\gdef\@@inv@@mag#1pt#2{\def#2{#1}}}
\def\target#1#2#3{%
  \mag #1\relax
  \@targetwidth=1000pt
  \divide\@targetwidth by #1\relax
  \expandafter\@@inv@@mag\the\@targetwidth\inv@targetmag
  \@targetwidth=#2\relax
  \@targetwidth=\inv@targetmag\@targetwidth
  \@targetheight=#3\relax
  \@targetheight=\inv@targetmag\@targetheight}
%
% Like \target, but for the source:
\def\source#1#2#3{%
  \@sourcewidth=1000pt
  \divide\@sourcewidth by #1\relax
  \expandafter\@@inv@@mag\the\@sourcewidth\inv@sourcemag
  \@sourcewidth=#2\relax
  \@sourcewidth=\inv@sourcemag\@sourcewidth
  \@sourceheight=#3\relax
  \@sourceheight=\inv@sourcemag\@sourceheight}
%
% \targetlayout does a loop that reads the comma separated arguments.
% There can be no extraneous spaces.
\def\targetlayout#1{\process@targetlayout#1,stop,}
\def\process@targetlayout#1,{%
  \expandafter\let\expandafter\next\csname target@#1\endcsname
  \ifx\next\relax
    \begingroup
      \errhelp{Valid target layouts are "topbottom", "twosided",
        "booklet", "Booklet" and "dvidvi".}%
      \errmessage{`#1' is invalid 2up target layout - ignored.}%
    \endgroup
    \expandafter\process@targetlayout
  \else
    \next
  \fi}
\def\target@stop{}
\def\target@booklet{%
  \booklet@registers
  \def\ship@@@leftpage{\save@booklet\@leftpage}%
  \def\ship@@@rightpage{\save@booklet\@rightpage}%
  \@leftpagefalse
  \def\twoup@eject{\twoup@eject@booklet}%
  \expandafter\process@targetlayout}
\def\target@Booklet{%
  \def\booklet@@loop{\Booklet@@loop}%
  \target@booklet}
\def\target@twosided{%
  \@twosidedtrue
  \expandafter\process@targetlayout}
\def\target@topbottom{%
  \def\make@@halfpage{\make@@halftopbottom}%
  \def\make@fullpage{\make@fulltopbottom}%
  \@sidebysidefalse
  \expandafter\process@targetlayout}
\def\target@dvidvi{%
  \def\ship@@@leftpage{\ship@dvidvi\@leftpage}%
  \def\ship@@@rightpage{\ship@dvidvi\@rightpage}%
  \expandafter\process@targetlayout}
%
% TeX's \shipout primitive is saved as \&normal@shipout, and then \shipout
% is defined to save each page to \@leftpage or \@rightpage and to print out
% every two. With the twosided layout, filler pages are added when needed.
\expandafter\let\csname &normal@shipout\endcsname\shipout
\def\shipout{%
  \if@leftpage
    \global\@leftpagefalse
    \def\next{\afterassignment\ship@leftpage\global\setbox\@leftpage=}%
    \if@twosided
      \ifodd\count\z@
        \global\setbox\@leftpage=\hbox{}%
        \make@@halfpage\@leftpage\ship@@@leftpage
        \def\next{\shipout}%
      \fi
    \fi
  \else
    \global\@leftpagetrue
    \def\next{\afterassignment\ship@rightpage\global\setbox\@rightpage=}%
    \if@twosided
      \ifodd\count\z@
      \else
        \global\setbox\@rightpage=\hbox{}%
        \make@@halfpage\@rightpage\ship@@@rightpage
        \def\next{\shipout}%
      \fi
    \fi
  \fi
  \next}
%
% The job of \ship@leftpage and \ship@rightpage is to invoke \ship@@leftpage
% or \ship@@rightpage at the right time. \shipout is followed either
% (i) by an \hbox, \vbox or \vtop, in which case \ship@leftpage is invoked
% after the opening {. \@leftpage is void, and \ship@leftpage invokes
% \ship@@leftpage after the closing }, or
% (ii) by a \box or \copy, in which case \ship@leftpage is invoked after
% the full assignment. \@leftpage is not voide, and \ship@leftpage invokes
% \ship@@leftpage immediately.
\def\ship@leftpage{%
  \ifvoid\@leftpage\aftergroup\ship@@leftpage\else\ship@@leftpage\fi}
\def\ship@rightpage{%
  \ifvoid\@rightpage\aftergroup\ship@@rightpage\else\ship@@rightpage\fi}
%
% \ship@@leftpage/\ship@@rightpage take the output box, and first make it
% into a fully-size source page (with \make@halfpage) and then this is
% centered horizontally and vertically in half of a target page (with
% \make@@halfpage). Then they are shipped individually or together.
\def\ship@@leftpage{\make@halfpage\@leftpage\ship@@@leftpage}
\def\ship@@rightpage{\make@halfpage\@rightpage\ship@@@rightpage}
\def\make@halfpage#1{%
  \dp#1=\z@
  \setbox#1=\vbox to\@sourceheight{%
    \vskip \inv@sourcemag in
    \vskip \voffset
    \hbox to\@sourcewidth{\hskip\inv@sourcemag in\hskip\hoffset\box#1\hss}%
    \vss}%
  \make@@halfpage#1}
%
% The definition of \make@@halfpage depends on the target layout.
\def\make@@halfsidebyside#1{%
  \global\setbox#1=\vbox to\@targetheight{\vss
    \hbox to.5\@targetwidth{\hss\box#1\hss}\vss}}
\def\make@@halftopbottom#1{%
  \global\setbox#1=\vbox to.5\@targetheight{\vss
    \hbox to\@targetwidth{\hss\box#1\hss}\vss}}
\def\make@@halfpage{\make@@halfsidebyside}
%
% The pages are generaly shipped in pairs:
\def\ship@twoup{%
  \begingroup
    \voffset=-\inv@targetmag in
    \hoffset=\voffset
    \global\advance\@physicalpage by 1
    \count\z@=\@physicalpage
    \csname &normal@shipout\endcsname\make@fullpage
  \endgroup}
\let\ship@@@leftpage\relax
\def\ship@@@rightpage{\ship@twoup}
%
% The definition of \make@fullpage depends on the layout:
\def\make@fullsidebyside{%
  \hbox{\box\@leftpage\pagesep@sidebyside\box\@rightpage}}
\def\make@fulltopbottom{%
  \vbox{\offinterlineskip\box\@leftpage\pagesep@topbottom\box\@rightpage}}
\def\make@fullpage{\make@fullsidebyside}
%
% A vertical or horizontal rule can be inserted. These can be redefined
% for other tricks:
\def\pagesep@sidebyside{%
  \begingroup
    \advance\pageseplength by \pagesepoffset
    \pagesepwidth=\inv@targetmag\pagesepwidth
    \kern -.5\pagesepwidth
    \vrule height \inv@targetmag\pageseplength
           depth -\inv@targetmag\pagesepoffset
           width \pagesepwidth
    \kern -.5\pagesepwidth
  \endgroup}
\def\pagesep@topbottom{%
  \begingroup
    \pagesepwidth=\inv@targetmag\pagesepwidth
    \vskip -.5\pagesepwidth
    \moveright\inv@targetmag\pagesepoffset\hbox{%
      \vrule height\pagesepwidth width\inv@targetmag\pageseplength}%
    \vskip -.5\pagesepwidth
  \endgroup}
%
% With the dvidvi layout, the pages are shipped individually:
\def\ship@dvidvi#1{%
  \begingroup
    \voffset=-\inv@targetmag in
    \hoffset=\voffset
    \csname &normal@shipout\endcsname\box#1%
  \endgroup}
%
% With the booklet or Booklet layout, the pages are saved rather than
% shipped.
\begingroup
\let\newbox\relax
\gdef\save@booklet#1{%
  \begingroup
    \globaldefs=1
    \advance\bookletpage by 1
    \expandafter\newbox\csname bookletbox\the\bookletpage\endcsname
    \expandafter\setbox\csname bookletbox\the\bookletpage\endcsname\box#1%
  \endgroup}
\endgroup
%
% The pages are then printed at the end with the following macros:
\def\make@bookletpage#1{%
  \setbox\ifodd#1\@rightpage\else\@leftpage\fi=%
    \expandafter\box\csname bookletbox\the#1\endcsname}
\def\booklet@loop{%
  \count\z@\rightpagenumber
  \make@bookletpage\leftpagenumber
  \make@bookletpage\rightpagenumber
  \ship@twoup
  \booklet@@loop}
\def\booklet@@loop{%
  \advance\rightpagenumber by 2
  \advance\leftpagenumber by -2
  \ifnum\leftpagenumber<1\else\expandafter\booklet@loop\fi}
\def\Booklet@@loop{%
  \advance\rightpagenumber by 1
  \advance\leftpagenumber by -1
  \ifnum\leftpagenumber<\rightpagenumber\else\expandafter\booklet@loop\fi}
%
% This one is easy:
\def\twoupemptypage{\shipout\hbox{}}
%
% This clears a whole target page if there is a saved left page. Note that
% this does not invoke the output routine; i.e., it is not like \clearpage
% or \supereject. See \twoupclearpage and \twoupeject below.
\def\twoup@eject{%
  \if@leftpage\else
    \global\setbox\@rightpage\hbox{}%
    \make@@halfpage\@rightpage\ship@@@rightpage
    \global\@leftpagetrue
  \fi}
%
% This is the definition of \twoup@eject with the booklet option:
\def\twoup@eject@booklet{%
  \leftpagenumber\bookletpage
  \advance\leftpagenumber by 3
  \divide\leftpagenumber by 4
  \multiply\leftpagenumber by 4
  \rightpagenumber=1
  \ifnum\leftpagenumber>\bookletpage
    \setbox\@leftpage\hbox{}%
    \make@@halfpage\@leftpage
    \loop
      \setbox\@rightpage\copy\@leftpage
      \save@booklet\@rightpage
    \ifnum\leftpagenumber>\bookletpage
    \repeat
  \fi
  \booklet@loop}
%
% This modification is needed for \LaTeX in order to get the last page
% printed out if the final page is a left page (the catcode business is
% because \enddocument is \let to \bye in amstex):
\begingroup
\expandafter\ifx\csname @latexerr\endcsname\relax
  \catcode`\>=14\else\catcode`\>=9\fi\relax
>>\gdef\twoupclearpage{\clearpage\twoup@eject}
>>\expandafter\@temptokena\expandafter{\enddocument}
>>\xdef\enddocument{\noexpand\twoupclearpage\the\@temptokena}
\endgroup
%
% For most other macro packages we could just leave be and all pages would
% always be printed because of the way the \end primitive works (except that
% TeX will go bonkers with the booklet layout). However,
% sometimes a blank filler page would be printed *with* headings. We prefer
% the filler page to be truly blank. To achieve this,  we hack the definition
% of \end. This may cause problems with some macros.
\expandafter\ifx\csname @latexerr\endcsname\relax
  \let\twoup@@@end\end
  \def\end{\twoup@eject\twoup@@@end}
  \def\twoupeject{\par\vfil\supereject\twoup@eject}
\fi
%
% This is one workaround for the page cross-references problem
\def\TwoupWrites{%
  \let\TwoupSaved@write\write
  \let\TwoupSaved@read\read
  \let\TwoupSaved@openout\openout
  \let\TwoupSaved@closeout\closeout
  \def\write{\TwoupSaved@write-1{}\immediate\TwoupSaved@write}%
  \def\read{\TwoupSaved@write-1{}\immediate\TwoupSaved@read}%
  \def\openout{\TwoupSaved@write-1{}\immediate\TwoupSaved@openout}%
  \def\closeout{\TwoupSaved@write-1{}\immediate\TwoupSaved@closeout}%
  \let\TwoupWrites\relax}
%
% Defaults:
\def\twouparticle{%
  \target{\magstepminus1}{11in}{8.5in}%
  \source{\magstep0}{8.5in}{11in}}
\def\twoupplain{%
  \target{\magstepminus2}{11in}{8.5in}%
  \source{\magstep0}{8.5in}{11in}}
\def\twouplegaltarget{%
  \target{\magstepminus1}{14in}{8.5in}%
  \source{\magstep0}{8.5in}{11in}}%
\def\twouplandscape{%
  \target{\magstepminus2}{8.5in}{11in}%
  \source{\magstep0}{11in}{8.5in}%
  \targetlayout{topbottom}}
\expandafter\ifx\csname @latexerr\endcsname\relax
  \twoupplain\else\twouparticle\fi
\pagesepwidth 0pt
\pageseplength 6.5in
\pagesepoffset 1in
%
\expandafter\catcode`\@=\TheAtCode\relax
\endinput
%% END 2up.tex/2up.sty
