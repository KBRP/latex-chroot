%%
%% This is file `apendint.tex',
%% generated with the docstrip utility.
%%
%% The original source files were:
%%
%% stepe.dtx  (with options: `apf1')
%% 
%%     This work has been partially funded by the US government
%%  and is not subject to copyright.
%% 
%%     This program is provided under the terms of the
%%  LaTeX Project Public License distributed from CTAN
%%  archives in directory macros/latex/base/lppl.txt.
%% 
%%  Author: Peter Wilson (CUA and NIST)
%%          now at: peter.r.wilson@boeing.com
%% 
\ProvidesFile{apendint.tex}[1996/05/31 AP end intro boilerplate]
\typeout{apendint.tex [1996/05/31 AP end intro boilerplate]}

    Application protocols provide the basis for developing
implementations of ISO~10303 and abstract test suites for
the conformance testing of AP implementations.

    Clause~\ref{;i1} defines the scope of the application protocol
and summarizes  the functionality and data covered by the AP.
Clause~\ref{;i3} lists the words defined in this part of ISO~10303 and
gives pointers to words defined elsewhere.
An application activity model that is the basis for the definition
of the scope is provided in \aref{;saam}. The information requirements
of the application are specified in \cref{;sireq} using terminology
appropriate to the application. A graphical representation of the
information requirements, referred to as the application reference
model, is given in \aref{;sarm}.

    Resource constructs are interpreted to meet the information
requirements. This interpretation produces the application
interpreted model (AIM). This interpretation, given in~\ref{;smap}, shows
the correspondence between the information requirements and the
AIM. The short listing of the AIM specifies the interface to the
integrated resources and is given in~\ref{;saesl}. Note that the definitions
and \Express{} provided in the integrated resources for constructs
used in the AIM may include select list items and subtypes which are
not imported into the AIM. The expanded listing given in \aref{;saeel}
contains the complete \Express{} for the AIM without annotation. A
graphical representation of the AIM is given in \aref{;saeg}. Additional
requirements for specific implementation methods are given in
\aref{;simreq}.

\endinput
%%
%% End of file `apendint.tex'.
