%%%%%%%%%%%%%%%%%%%%%%%%%%%%%%%%%%%%%%%%%%%%%%%%%%%%%%%%%%%%%%%%%%%%%%%%%%%%%
%
% Typographic ruler for TeX, version 1.1
% Copyright Victor Eijkhout 1999
% file name: ruler.tex
%
% Author:
% Victor Eijkhout
% Department of Computer Science
% University of Tennessee, Knoxville TN 37996
%
% victor@eijkhout.net
%
% This program is free software; you can redistribute it and/or
% modify it under the terms of the GNU General Public License
% as published by the Free Software Foundation; either version 2
% of the License, or (at your option) any later version.
% 
% This program is distributed in the hope that it will be useful,
% but WITHOUT ANY WARRANTY; without even the implied warranty of
% MERCHANTABILITY or FITNESS FOR A PARTICULAR PURPOSE.  See the
% GNU General Public License for more details.
%
% For a copy of the GNU General Public License, write to the 
% Free Software Foundation, Inc.,
% 59 Temple Place - Suite 330, Boston, MA  02111-1307, USA,
% or find it on the net, for instance at
% http://www.gnu.org/copyleft/gpl.html
%
%%%%%%%%%%%%%%%%%%%%%%%%%%%%%%%%%%%%%%%%%%%%%%%%%%%%%%%%%%%%%%%%%%%%%%%%%%%%%
%
% This file is identical to an earlier version which was on ctan and
% other archive sites; only the license and copyright information
% has changed.
%%%%%%%%%%%%%%%%%%%%%%%%%%%%%%%%%%%%%%%%%%%%%%%%%%%%%%%%%%%%%%%%%%%%%%%%%%%%%
%
\hsize=14.5cm \vsize=32cm \voffset=-1.5cm 
\nopagenumbers \raggedbottom
\overfullrule=0pt
%\tracingoutput=1 
\showboxbreadth=20 \showboxdepth=7

\font\tiny=cmr5 \font\ss=cmr8 
\font\big=cmss10 scaled \magstep3
\font\manual=cmr10

%\font\tiny=helv at 5pt \font\ss=helv at 8pt 
%\font\big=helv at 16pt
%\font\manual=times at 10pt 

\let\tt=\ss \let\rm=\ss
\parindent=0cm \offinterlineskip
\def\\{\hfill\break}

\newdimen\margin \margin=.5cm
\newdimen\linewidth \linewidth=0.05mm
\newdimen\scalewidth \scalewidth=6mm \newdimen\magboxwidth
\newdimen\gridheight \gridheight=8cm
\newdimen\rulewidth \rulewidth=\scalewidth
\newdimen\ruleboxheight \ruleboxheight=6mm

\newbox\magscalenames \newbox\scale \newbox\grid 
\newbox\stairs \newbox\ruler
\newcount\ncol \newcount\nrow
\newcount\tempcount \newcount\tempcountb \newcount\tempcounth
\newbox\tempboxa \newbox\tempboxb
\newskip\tempskip \newskip\tempskipb
\newdimen\tempdimen \newdimen\tempdimenb \newdimen\tempdimenc

\newbox\storea \newbox\storeb \newbox\storec 
\newbox\stored \newbox\storee \newbox\storef
\newbox\storeg \newbox\storeh \newbox\storei
\newbox\storej

\def\gobble#1{}
\def\gobbletwo#1#2{}
\def\gobblethree#1#2#3{}
\def\gobblefive#1#2#3#4#5{}

\def\hfilll{\hskip 0cm plus 1filll\relax}

%%%%%%%%%%%%%%%% trace
%\tracingonline=1 
%\tracingmacros=1

%%%%%%%%%%%%%%%% auxiliaries
\def\decimala     %% #1=getal; a-> 1 plaats achter de komma
    #1{{\tempcounth=#1\relax
        \tempcountb=\tempcounth
        \divide\tempcountb by 10\relax
        \number\tempcountb.%
        \multiply\tempcountb by 10\relax
        \advance\tempcounth by -\tempcountb
        \number\tempcounth}}
        
\def\decimalb     %% #1=getal; b-> 2 plaatsen achter de komma
    #1{{\tempcounth=#1\relax
        \tempcountb=\tempcounth
        \divide\tempcountb by 100\relax
        \number\tempcountb.%
        \multiply\tempcountb by 100\relax
        \advance\tempcounth by -\tempcountb
        \ifnum\tempcounth<10 0\fi
        \number\tempcounth}}
        
\def\hundred#1{{\tempcounth=#1 
                \tempcountb=\tempcounth
                \divide\tempcounth by 100\relax
                \ifnum\tempcounth>0 
                            \number\tempcounth.\fi
                \multiply\tempcounth by 100
                \advance\tempcountb by -\tempcounth
                \ifnum\tempcountb<10 0\fi
                \number\tempcountb}}

\def\thousand#1{{\tempcount=#1 \tempcountb=\tempcount
                 \divide\tempcount by 1000
                 \ifnum\tempcount>0 \number\tempcount.\fi
                 \multiply\tempcount by 1000
                 \advance\tempcountb by -\tempcount
                 \ifnum\tempcountb<100 0\fi
                 \ifnum\tempcountb<10 0\fi
                 \number\tempcountb}}

\def\headedbox        %% #1=heading text #2=vbox
    #1#2{\vtop{\vbox to 0cm{\vss\setbox0=\hbox to 0cm{\tt #1\hss} \dp0=0cm 
                            \box0 \vskip2pt}
               #2}}
\def\boxit#1{\vbox{\hrule
                   \hbox{\vrule #1\vrule}
                   \hrule}}

%%%%%%%%%%%%%%%%%%%%%%%%%%%%%%%%%%%%%%%%%%%%%%%%%%%%%%%%%%%%%%%%%%%%
%%%%%%%%%%%%%%%%%%%%%% baseline measurement %%%%%%%%%%%%%%%%%%%%%%%%
\newskip\liftskip
%% what baselineskips are needed?
\def\baselinedist#1{\ifcase#1\or 5\or 6\or 7\or 8\or 9\or 10\or
           11\or 12\or 13\or 14\or 15\or 16\or 17\or
           18\or 20\or 22\or 25\or 30\fi}

%% first one step in a ladder of baselineskips
\def\baselinebox      %% #1=size #2=number label
    #1#2{\vbox to #1pt{\vfil
                 \moveright .5\hsize\vbox{\hrule height \linewidth 
                                          width .5\hsize}
                 \vfil\vbox to 0cm{\vss\hskip2pt\tiny\number#2\vskip\liftskip}
                 \hrule height \linewidth}}

%% next a whole ladder of baselineskips
\def\baselinescale    %% #1=number of steps #2=point size
    #1#2{\ifnum#2<7 \liftskip=0pt \else \liftskip=2pt \fi
         \setbox\scale=\vbox{}%
         \baselinescalei 1{#1}{#2}%
         \edef\tcs{{#2}{\box\scale}}\expandafter\headedbox\tcs}
\def\baselinescalei   %% #1=label #2=number of steps #3=point size
    #1#2#3{\ifnum#1>#2 \def\tcs{0}\let\next=\gobblethree
           \else\tempcount=#1 \advance\tempcount by 1
                \setbox\scale=\vtop{\hsize=\scalewidth
                                    \advance\hsize by -\linewidth
                                    \unvbox\scale
                                    \baselinebox{#3}{#1}}%
                \let\next\baselinescalei \edef\tcs{{\number\tempcount}}\fi
           \expandafter\next\tcs{#2}{#3}}

%% finally a grid of ladders of baselineskips
\def\baselinegrid     %% #1=number of baseline scales #2=storage box
    #1#2{\setbox\grid=\hbox{}%
         \setbox#2=\vtop{\hrule height 0pt depth \linewidth
             \baselinegridi 1{#1}
             \hrule height \linewidth}}
\def\baselinegridi    %% #1=current number #2=total number of baseline scales
    #1#2{\ifnum#1>#2 \hbox{\vsize=\gridheight \unhbox\grid
                           \vrule width \linewidth}
         \else\tempcount=\gridheight \tempskip=\baselinedist{#1}pt
              \divide\tempcount by \tempskip \edef\tcs{{\number\tempcount}}
              \setbox\grid=
                  \hbox{\vsize=\gridheight \unhbox\grid
                        \vrule width \linewidth
                        \expandafter\baselinescale\tcs{\baselinedist{#1}}}
              \tempcount=#1 \advance\tempcount by 1 \edef\tcs{{\number\tempcount}}
              \expandafter\baselinegridi\tcs{#2}\fi}


%%%%%%%%%%%%%%%%%%%%%%%%%%%%%%%%%%%%%%%%%%%%%%%%%%%%%%%%%%%%%%%%%%%%
%%%%%%%%%%%%%%%%%%%%% magnified font sizes %%%%%%%%%%%%%%%%%%%%%%%%%
\newcount\honcor
\long\def\bungeltext#1#2{
    \hrule height \linewidth
    \kern1mm
    \vbox{\hsize=#1\relax \rightskip=0cm plus 1fil
          \lineskiplimit=0pt \lineskip=0pt \baselineskip=6pt
          \tiny #2\par}}

\def\magstep#1{\ifcase#1 1000\or 1095\or 1200\or 1440\or 1728\or 
                         2074\or 2488\fi}
\def\tienstep#1{\ifcase#1 10\or 10.95\or 12\or 14.4\or 17.28\or
                         20.74\or 24.88\fi}
\def\zevenstep#1{\ifcase#1 7\or 7.67\or 8.4\or 10.08\or 12.10\or
                         14.52\or 17.42\fi}
\def\vijfstep#1{\ifcase#1 5\or 5.48\or 6\or 7.2\or 8.64\or
                         10.37\or 12.44\fi}

\def\magname#1{\ifcase#1 0\or half\or 1\or 2\or 3\or 4\or 5\fi}
\def\cmmagbox#1{{\divide\tempdimen by 1000
                 \multiply\tempdimen by \magstep#1%
                 \tempdimenb=.01mm \tempdimenc=\tempdimen 
                 \divide\tempdimenc by \tempdimenb
                 \tempcount=\tempdimenc
                 \edef\next{\the\tempcount}%
                 \divide\honcor by 1000
                 \multiply \honcor by \magstep#1\relax
                 \tempdimenc=.01pt \divide\honcor by \tempdimenc
                 \edef\mixt{\the\honcor}%
                 \edef\next{\noexpand\magbox{\the\tempdimen}{\magname#1}%
                             {\noexpand\hundred{\next}}
                             {\noexpand\corrout#1}}\next}}
\def\magbox  %% #1=height #2=first text #3=second #4=third
    #1#2#3#4{\vtop{\hbox{\vrule width\linewidth
                       \vbox{\hrule width \magboxwidth height \linewidth 
                             \kern #1}}
                   \bungeltext\magboxwidth
                        {#2\par #3\par #4\par}
                  }}

\def\magsteps{\vtop{\hbox{\cmmagbox0\cmmagbox1\cmmagbox2%
                          \cmmagbox3\cmmagbox4\cmmagbox5\cmmagbox6%
                          \vrule width \linewidth depth 0cm}}}

\def\magrow           %% #1=name #2=unit numerator #3=unit denominator
                      %% #4=unit size #5=design size #6=store box
  #1#2#3#4#5#6{\tempdimen=#4 \divide\tempdimen by #3
               \multiply\tempdimen by #2
               \tempdimenb=#5pt \honcor=\tempdimenb
               \ifnum#5=10 \let\corrout\tienstep
                     \else\ifnum#5=7 \let\corrout\zevenstep
                     \else \let\corrout\vijfstep \fi\fi
               \magboxwidth=\scalewidth \setbox\tempboxa=\magsteps
               \setbox\tempboxb=
                       \vtop{\vbox{\hsize=1.2cm 
                                   \lineskiplimit=0pt \lineskip=0pt
                                   \baselineskip=6pt
                                   \def\\{\hfil\break} \tiny #1
                                   \par \kern1mm}
                             \setbox0=\hbox{\tiny point size }
                             \bungeltext{\wd0}{magstep\par 
                                               mm \par point size\par}}
               \tempdimen=\wd\tempboxa \advance\tempdimen by \wd\tempboxb
               \advance\tempdimen by 2mm
               \setbox#6=
                 \hbox to \tempdimen{\box\tempboxa \kern2mm \box\tempboxb}}
   
\def\magsizebox      %% #1=text
    #1{\hbox to 1.5\scalewidth{\hfil \tiny #1\hfil}}

%%%%%%%%%%%%%%%%%%%%%%%% general magrows %%%%%%%%%%%%%%%%%%%%%%%%%%%

\def\gmagsteps        %% #1=from #2=by #3=to
    #1#2#3{\tempcount=#1\relax
           {\multiply\tempdimen by \tempcount \tempcountb=\tempdimen
            \tempdimenc=.01mm \divide\tempcountb by \tempdimenc
            \edef\next{\the\tempcountb}%
            \edef\next{\noexpand\magbox{\the\tempdimen}{\the\tempcount}%
                                       {\noexpand\hundred{\next}}{}}%
            \next}%
           \advance\tempcount by #2%
           \ifnum\tempcount=#3\relax\vrule width \linewidth depth 0cm
           \else\edef\next{\noexpand\gmagsteps
                             {\number\tempcount}{#2}{#3}}\next\fi}

\def\gmagrow          %% #1=name #2=unit numerator #3=unit denominator
                      %% #4=unit size #5=from #6=by #7=to #8=store box
    #1#2#3#4#5#6#7#8{\tempdimen=#4 \divide\tempdimen by #3
       \multiply\tempdimen by #2 
       \setbox#8=\hbox{\magboxwidth\scalewidth \gmagsteps{#5}{#6}{#7}
                       \ {}
                       \vtop{\kern0cm
                             \setbox0=\hbox{\tiny point size }
                             \bungeltext{\wd0}{point size\\mm}}}}

%%%%%%%%%%%%%%%%%%%%%%%%%%%%%%%%%%%%%%%%%%%%%%%%%%%%%%%%%%%%%%%%%%%%
%%%%%%%%%%%%%%%%%%%%%%%%%%%%% rulers %%%%%%%%%%%%%%%%%%%%%%%%%%%%%%%
\newif\ifleftruler \newdimen\stapje \newdimen\Stap \newdimen\hstapje
\newcount\halfway \newcount\alltheway
\newbox\onerulerbox
\newdimen\vollebreedte \newdimen\vollebreedteplus
\newdimen\overschot \overschot=.75mm

\def\doalltheway
    #1#2#3{#3{#1}\relax
           \ifnum#1=#2\relax\else
           \tempcount=#1 \advance\tempcount by 1
           \expandafter\doalltheway\expandafter
                {\number\tempcount}{#2}{#3}\fi}

\def\centimeterruler{\Stap=1cm \leftrulerfalse
                     \vollebreedte=1cm\relax
                     \def\rulename{centimeter}%
                     \let\ruledigitfont=\rm
                     \alltheway=10 \doruler{cm}}
\def\inchruler{\Stap=1in \leftrulertrue 
               \vollebreedte=1cm\relax
               \def\rulename{inch}%
               \let\ruledigitfont=\rm
               \alltheway=8 \doruler{inch}}
\def\picaruler{\Stap=12pt \leftrulerfalse 
               \vollebreedte=1cm\relax
               \def\rulename{pica}%
               \let\ruledigitfont=\tiny
               \alltheway=6 \doruler{pica}}
\def\rulerpiece
    #1{\vbox to \stapje{\vfil \rulerhalfline \vfil \rulerline{#1}}}
\def\rulerline %% streepje over aantal mm; halfweg volle breedte
    #1{\ifnum#1=\halfway \tempskipb=\vollebreedteplus
       \else \tempskipb=#1\hstapje
             \advance\tempskipb by \overschot \fi
       \hbox to \vollebreedteplus
            {\maybeleft \vrule width\tempskipb height\linewidth
             \mayberight }}
\def\rulerhalfline
    {\tempskipb=.5\hstapje \advance\tempskipb by \overschot
     \hbox to \vollebreedteplus
          {\maybeleft\vrule width\tempskipb height\linewidth
           \mayberight}}
\def\rulerbox         %% #1=label
    #1{\vbox to 0cm{\kern.75mm\relax %was .5\stapje
                    \hbox to \vollebreedteplus
                                {\mayberight
                                 \kern.75mm\ruledigitfont 
                                 #1\kern.75mm
                                 \maybeleft}
                    \vss}
       \copy\onerulerbox}
\def\makerulerbox{\global\setbox\onerulerbox=
       \vbox to \Stap{\vbox to 0cm
                {\hbox{\ifleftruler\kern\overschot\fi
                       \kern.5\vollebreedte
                       \vrule depth \Stap width\linewidth}
                 \vss}
                       \doalltheway1{\alltheway}{\rulerpiece}
                       \vss}}
\def\doruler  %% #1=unit name #2=length #3=storage box 
    #1#2#3{\ifleftruler \let\maybeleft\relax \let\mayberight\hfil
           \else        \let\maybeleft\hfil  \let\mayberight\relax
           \fi
       \bgroup \halfway=\alltheway \divide\halfway by 2\relax
               \stapje=\Stap \divide\stapje by \alltheway
               \hstapje=\vollebreedte \divide\hstapje by \alltheway
       \vollebreedteplus=\vollebreedte
       \advance\vollebreedteplus by \overschot
       \makerulerbox
       \setbox\ruler=\vbox{\hrule width \vollebreedteplus height\linewidth}%
       \doruleri 0{#2}\relax
       \global\setbox\ruler=
           \hbox{\ifleftruler\else\vrule width\linewidth\fi
                 \box\ruler
                 \ifleftruler \vrule width\linewidth \fi}%
       \xdef\tcs{{#1}{\box\ruler}}\egroup
       \setbox#3\expandafter\headedbox\tcs}
 \def\doruleri
    #1#2{\ifnum#1>#2 \let\next=\gobbletwo \def\tcs{0}%
         \else \setbox\ruler=
                   \vtop{\unvbox\ruler\rulerbox{#1}}%
               \let\next=\doruleri
               \tempcount=#1\relax \advance\tempcount by 1 
               \edef\tcs{{\number\tempcount}}\fi
         \expandafter\next\tcs{#2}}

%%%%%%%%%%%%%%%%%%%%%%%%%% paper size %%%%%%%%%%%%%%%%%%%%%%%%%%%%%
\def\papersize#1#2{\vtop
    {\kern0cm\vbox to #2{
     \vbox to #1{\vfil\hrule width 2mm height 0cm depth \linewidth}
                 \vfil\hrule width 2mm height 0cm depth \linewidth}}}

%%%%%%%%%%%%%%%%%%%%%%%%%%%%%%%%%%%%%%%%%%%%%%%%%%%%%%%%%%%%%%%%%%%
%%%%%%%%%%%%%%%%%%%%%%%%% rule thickness %%%%%%%%%%%%%%%%%%%%%%%%%%
\newdimen\rulelength \rulelength=.75\scalewidth
\newdimen\interruleskip \interruleskip=.5\scalewidth

\def\ruleboxh         %% #1=rule thickness #2=label
    #1#2{\setbox\tempbox=
                  \vbox{\hsize=\rulewidth 
                        \vfill \vskip10pt \hbox{\tiny \rulenum{#2}}
                        \vskip1pt \hrule height #1 width \rulewidth}%
         \ifdim\ht\tempbox<\ruleboxheight
               \vbox to \ruleboxheight{\unvbox\tempbox}%
         \else\box\tempbox\fi}
\def\ruleboxv         %% #1=rule thickness #2=label
    #1#2{\vtop{\kern0cm
               \hbox{\hbox to 5mm{\tiny \hfil \rulenum{#2}\kern1mm}%
                     \vrule height \rulelength width #1}
               \kern \interruleskip}}
\let\rulebox=\ruleboxv
\def\rulestackhead    %% #1=minimal label #2=label step
                      %% #3=minimal dimen #4=dimen step
                      %% #5=maximal label #6=storage box #7=header
    #1#2#3#4#5#6#7{\setbox\scale=\vbox{}%
             \rulestacki{#1}{#2}{#3}{#4}{#5}%
             \setbox#6=\headedbox{#7}{\box\scale}}
\def\rulestack        %% #1=minimal label #2=label step
                      %% #3=minimal dimen #4=dimen step
                      %% #5=maximal label #6=storage box
    #1#2#3#4#5#6{\setbox\scale=\vbox{}%
             \rulestacki{#1}{#2}{#3}{#4}{#5}%
             \setbox#6=\box\scale}
\def\rulestacki       %% #1=curr label #2=label step 
                      %% #3=curr dimen #4=dimen step #5=total dimen
    #1#2#3#4#5{\ifnum#1>#5%
                     \setbox\scale=
                        \vtop{\unvbox\scale \setbox0=\lastbox}%
                     \let\next=\gobblefive \def\tcs{0}%
               \else \setbox\scale=
                        \vtop{\unvbox\scale
                              \rulebox{#3}{#1}}%
                     \tempskip=#3\relax \advance\tempskip by #4\relax
                     \tempcount=#1\relax \advance\tempcount by #2\relax
                     \let\next=\rulestacki
                     \edef\tcs{{\number\tempcount}}\fi
               \expandafter\next\tcs{#2}{\tempskip}{#4}{#5}}
                     

%%%%%%%%%%%%%%%%%%%%%%%%%% conversion table %%%%%%%%%%%%%%%%%%%%%%%%
\def\num{\ss\enspace\hfill}
\def\me#1{\ifcase#1\relax\or pt\or pc\or dd\or cc\or bp\or mm\or in\fi}
\def\table{\halign{\strut\hbox to 1cm{\tt##\hfil}\vrule\enspace
                   &\num##&\num##&\num##&\num##&\num##&\num##&\num##\cr
    \hfill&\tt \me1&\tt \me2&\tt \me3&\tt \me4&\tt \me5&\tt \me6&\tt \me7\cr
   \noalign{\hrule}
    \me1&1&.0833&&&&.351&.0138\cr
    \me2&12&1&&&&4.218&.166\cr
    \me3&&&1&.0833&&.376&.0148\cr
    \me4&&&12&1&&4.513&.1777\cr
    \me5&&&&&1&.383&.0139\cr
    \me6&2.845&.237&2.659&.222&2.835&1&.0394\cr
    \me7&72.27&6.0225&67.54&5.628&72&25.4&1\cr
          }}
%%%%%%%%%%%%%%%%%%%%%%%%%% the whole sheet %%%%%%%%%%%%%%%%%%%%%%%%%

\def\opskip{\vskip.5mm\relax}
\let\neerskip\opskip

\output{}

\begingroup\hsize=12cm \vsize=8in
\parindent=0pt \parskip=6pt
\rightskip=0cm plus 5cm \normalbaselines

\hbox{}\vskip3cm
\big The \TeX\ ruler

\manual
This is the \TeX\ ruler, a measuring device for
typographical designers working with \TeX. It includes ordinary
rulers in inches, centimeters, and picas.
It can be used to measure font sizes, baseline distances,
and rule weights. The design is based on a ruler
by Inge~Eijkhout, implementation in \TeX\ is by 
Victor~Eijkhout.

The \TeX\ ruler is shareware, distributed under the GNU Public License.
If you want to show your appreciation, send a contribution
to the address below.

How to format and print this file

This file will give you three pages. Most likely the second one
cannot be handled by your printer, but I've included it for
completeness. At the top of the source file, 3 fonts are
declared. Please change the CM fonts to something more pretty,
eg Helvetica.

How to use the \TeX\ ruler

Rulers. Use them like ordinary rulers. It is suggested
that you print the \TeX\ ruler on transparent plastic and
cut the edges along the rulers.

Baseline distances. Put the top line along the baseline
of some line of text. Then see in what column the other
lines of text line up.

Font sizes: cap height. Take an uppercase letter
such as `K' or `H'. Find the line with its font name,
or use the first line. See what box on that line
it fits in best, with the character completely between
the lines. Read out the magnification and size of the
font. Alternately check on all lines which box fits
best, then read out the size.

General remark: all lines have a width of .05 millimeter. 
This is less than the dot width of a 300dpi
laser printer, so beware of some inaccuracies.
Also your printer driver may introduce 
inaccuracies.

\leavevmode
{\obeylines \parskip=0pt %
Victor Eijkhout
Department of Computer Science
University of Tennessee
Knoxville, TN 37996
USA\par {\tt victor@eijkhout.net}}

\vfil
Copyright 1990-99 by Victor Eijkhout
\eject
\endgroup

\endlinechar-1

\def\BigSheet{\leavevmode
\leftrulertrue \inchruler{11}{\storea}
\box\storea
\papersize{8.5in}{11in}
%
\vtop{
  \baselinegrid{18}\storea \box\storea
  \kern1mm
  \moveright2mm
  \vbox{\hbox{\ss Baseline distances}
        \neerskip \hrule}
  \kern5mm
  %
 \moveright2mm
 \hbox{
 \vtop{\vbox{\hrule \opskip \hbox{\ss Cap heights}}
       \kern-3mm
  \def\a{\kern2mm}
  \magrow{cmr10\\cmbx10\\cmss10\\cmmi10}{250}{36}{1pt}{10}\storec
  \magrow{cmtt10\\cmff10}{220}{36}{1pt}{10}\stored
  \magrow{cmr7\\cmbx7\\cmmi7}{175}{36}{1pt}7\storee
  \magrow{cmr5\\cmbx5\\cmmi5}{125}{36}{1pt}5\storef
  \vbox{\a\box\storec \a\box\stored
        \a\box\storee \a\box\storef}
  %
  \gmagrow{cmr10}{25}{36}{1pt}51{14}\storec
  \gmagrow{cmr10}{25}{36}{1pt}{15}1{24}\stored
  \gmagrow{cmr10}{25}{36}{1pt}{26}2{44}\storee
  \gmagrow{cmr10}{25}{36}{1pt}{46}2{64}\storef
  \vbox{\a\box\storec \a\box\stored
        \a\box\storee \a\box\storef}
  %
  \leftrulertrue \picaruler{19}{\storea}
  \setbox\storeb=\vtop{\kern0cm 
                       \def\a{}%{\moveright1cm}
                       \a\vbox{\hrule \opskip
                               \hbox{\strut\ss Conversion of units}}
                       \kern2mm
                       \halign{\vrule height8pt depth3pt width0pt
                               \ss##\ &${}={}$\hfil\ \ss##&$\,$##\cr
                               1{\tt in}& 72.27&{\tt pt}\cr
                               1{\tt in}& 72&{\tt bp}\cr
                               1{\tt in}& 25.4&{\tt mm}\cr
                               1157{\tt dd}& 1238&{\tt pt}\cr
                               1{\tt pt}& 65536&{\tt sp}\cr}
                       \kern2mm \table
                       \kern8mm
                       \hbox{\hskip1cm \big The \TeX\ Ruler}
                       \kern2pt
                       \hbox{\hskip1cm \tiny Design:\ {}
                             Inge and Victor Eijkhout 1990--96}}
  \kern4mm
  \hbox{\box\storea \kern2mm \box\storeb}
  }%end of \vtop
 \kern-1cm
 \vtop{\vbox{\hrule \opskip \hbox{\ss Rule weights}}
       \kern5mm
   \let\rulenum=\decimalb
   \rulestackhead{5}{5}{0.05mm}{0.05mm}{100}\storeg{mm}
   \rulestack    {100}{25}{1mm}{0.25mm}{525}\storeh
   \let\rulenum=\decimala
   \rulestackhead{5}{5}{.5pt}{.5pt}{151}\storei{pt}
   \hbox{\box\storeg \kern3mm \box\storeh \kern3mm \box\storei}
  }%end of second \vtop
 }%end of \hbox
  }
%
\kern2mm\relax
\leftrulerfalse \centimeterruler {29}{\storea}
\papersize{21cm}{29.7cm}
\box\storea
}

%%%%%%%%%%%%%%%%%%%%%%%%%%%%%%
\def\smallsheet{\leavevmode
\leftrulertrue \inchruler{9}{\storea}
\box\storea \kern2mm
%
\vtop{
  \baselinegrid{18}\storea \box\storea
  \kern1mm
  \moveright2mm
  \vbox{\hbox{\ss Baseline distances}
        \neerskip \hrule}
  \kern5mm
  %
 \moveright2mm
 \hbox{
 \vtop{\vbox{\hrule \opskip \hbox{\ss Cap heights}}
       \kern-3mm
  \def\a{\kern2mm}
  \magrow{cmr10\\cmbx10\\cmss10\\cmmi10}{250}{36}{1pt}{10}\storec
  \magrow{cmtt10\\cmff10}{220}{36}{1pt}{10}\stored
  \magrow{cmr7\\cmbx7\\cmmi7}{175}{36}{1pt}7\storee
  \magrow{cmr5\\cmbx5\\cmmi5}{125}{36}{1pt}5\storef
  \vbox{\a\box\storec \a\box\stored
        \a\box\storee \a\box\storef}
%
  \leftrulertrue \picaruler{19}{\storea}%\showbox\storea
  \setbox\storeb=\vtop{\kern0cm 
                       \def\a{}%{\moveright1cm}
                       \a\vbox{\hrule \opskip
                               \hbox{\strut\ss Conversion of units}}
                       \kern2mm
                       \halign{\vrule height8pt depth3pt width0pt
                               \ss##\ &${}={}$\hfil\ \ss##&$\,$##\cr
                               1{\tt in}& 72.27&{\tt pt}\cr
                               1{\tt in}& 72&{\tt bp}\cr
                               1{\tt in}& 25.4&{\tt mm}\cr
                               1157{\tt dd}& 1238&{\tt pt}\cr
                               1{\tt pt}& 65536&{\tt sp}\cr}
                       \kern2mm \table
                       \kern8mm
                       \hbox{\hskip1cm \big The \TeX\ Ruler}
                       \kern2pt
                       \hbox{\hskip1cm \tiny Design:\ {}
                             Inge and Victor Eijkhout 1990--96}}
  \kern4mm
  \hbox{\box\storea \kern2mm \box\storeb}
  }%end of \vtop
 \kern-1cm
 \vtop{\vbox{\hrule \opskip \hbox{\ss Rule weights}}
       \kern5mm
   \let\rulenum=\decimalb
   \rulestackhead{5}{5}{0.05mm}{0.05mm}{50}\storeg{mm}
   \rulestack    {50}{10}{.5mm}{0.1mm}{160}\storeh
   \let\rulenum=\decimala
   \rulestackhead{5}{5}{.5pt}{.5pt}{101}\storei{pt}
   \hbox{\box\storeg \kern3mm \box\storeh \kern3mm \box\storei}
  }%end of second \vtop
 }%end of \hbox
  }
%
\kern2mm\relax
\leftrulerfalse \centimeterruler {23}{\storea}
\box\storea
}


\count0=2
\BigSheet
\vfill\eject
\count0=3
\smallsheet

\end
%%%%%%%%%%%%%%%%%%%%%%%%%%%%%%%%%%%%%%%%%%%%%%%%%%%%%%%%%%%%%%%%
