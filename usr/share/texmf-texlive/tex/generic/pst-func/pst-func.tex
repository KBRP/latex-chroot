%%
%% This is file `pst-func.tex',
%%
%% IMPORTANT NOTICE:
%%
%% Package `pst-func.tex'
%%
%% Herbert Voss <voss@pstricks.de>
%%
%% This program can be redistributed and/or modified under the terms
%% of the LaTeX Project Public License Distributed from CTAN archives
%% in directory macros/latex/base/lppl.txt.
%%
%% DESCRIPTION:
%%   `pst-func' is a PSTricks package to plot special functions
%%
%% For a ChangeLog go the the end
%%
\csname PSTfuncLoaded\endcsname
\let\PSTfuncLoaded\endinput
% Requires PSTricks, pst-node, pst-xkey
\ifx\PSTricksLoaded\endinput\else\input pstricks.tex\fi
\ifx\PSTnodesLoaded\endinput\else\input pst-plot.tex\fi
\ifx\PSTricksAddLoaded\endinput\else\input pstricks-add.tex\fi
\ifx\PSTXKeyLoaded\endinput\else\input pst-xkey.tex \fi
%
\edef\PstAtCode{\the\catcode`\@} \catcode`\@=11\relax
% interface to the `xkeyval' package
\pst@addfams{pst-func}

\def\fileversion{0.46}
\def\filedate{2006/09/06}
\message{`PST-func' v\fileversion, \filedate\space (hv)}
%
\pstheader{pst-func.pro}
%
\define@key[psset]{pst-func}{xShift}{\def\psk@xShift{#1}}
\psset[pst-func]{xShift=0}
%
\define@key[psset]{pst-func}{cosCoeff}{\def\psk@cosCoeff{#1}}
\define@key[psset]{pst-func}{sinCoeff}{\def\psk@sinCoeff{#1}}
\psset[pst-func]{cosCoeff=0,sinCoeff=1} % coeff=a0 a1 a2 a3 ...
%
\def\psFourier{\@ifnextchar[{\psFourier@i}{\psFourier@i[]}}
\def\psFourier@i[#1]#2#3{{%
  \pst@killglue
  \psset{#1}
  \psplot{#2}{#3}{%
      /type (cos) def
      /Fourier {
        aload length /n exch def
        n -1 roll 2 div n 1 roll % a0/2
        n 1 sub -1 0 {
          /i exch def
          i x mul 180 mul 3.141592 div
          type (sin) eq {sin}{cos} ifelse
          mul n 1 roll
        } for
        n 1 sub -1 1 { pop add } for
      } def
      [\psk@cosCoeff] Fourier
      /type (sin) def
      [0 \psk@sinCoeff] Fourier add
    }%
}\ignorespaces}
%
\define@key[psset]{pst-func}{coeff}{\def\psk@coeff{#1}}
\define@key[psset]{pst-func}{Abbreviation}{\def\psk@Deriviation{#1}}% compatibility
\define@key[psset]{pst-func}{Derivation}{\def\psk@Derivation{#1}}
\define@boolkey[psset]{pst-func}[Pst@]{markZeros}[true]{}
\define@key[psset]{pst-func}{epsZero}{\def\psk@epsZero{#1}}
\define@key[psset]{pst-func}{dZero}{\def\psk@dZero{#1}}
\define@key[psset]{pst-func}{zeroLineTo}{\def\psk@zeroLineTo{#1}}
\define@key[psset]{pst-func}{zeroLineColor}{\pst@getcolor{#1}\psk@zeroLineColor}
\newdimen\psk@zeroLineWidth
\define@key[psset]{pst-func}{zeroLineWidth}{\pssetlength\psk@zeroLineWidth{#1}}
\define@key[psset]{pst-func}{zeroLineStyle}{%
  \@ifundefined{psls@#1}%
    {\@pstrickserr{Line style `#1' not defined}\@eha}%
    {\edef\psk@zeroLineStyle{#1}}%
}
\psset[pst-func]{%
       coeff=0 1,      % coeff=a0 a1 a2 a3 ...
       Derivation=0, % 0 is the original function
       markZeros=false,% no dots for the zeros
       epsZero=0.1,    % the distance between two zero points   
       dZero=0.1,      % the distance of the x value for scanning the function
       zeroLineTo=-1,  % a line to the value of the lineTo's Derivation (-1= none)
       zeroLineStyle=dashed,%
       zeroLineWidth=0.5\pslinewidth,%
       zeroLineColor=black}%
%
\def\psPolynomial{\pst@object{psPolynomial}}
\def\psPolynomial@i#1#2{{%
  \begin@OpenObj
  \@nameuse{beginplot@\psplotstyle}%
  \gdef\psplot@init{}%
  \@nameuse{testqp@\psplotstyle}%
  \addto@pscode{%
    tx@FuncDict begin
    /coeff [ \psk@coeff ] def
    /x0 #1 def /x1 #2 def
    /dx x1 x0 sub \psk@plotpoints\space div def
    /Derivation \psk@Derivation\space def
    \ifPst@markZeros 
      gsave
      \pst@number\psk@zeroLineWidth SLW
      \pst@usecolor\psk@zeroLineColor
      \psk@epsZero\space \psk@dZero\space FindZeros 
      pstZeros aload length {
        /xZero exch def
        xZero \pst@number\psxunit mul /xPixel exch def
        \psk@dotsize
        \@nameuse{psds@\psk@dotstyle}%
        xPixel 0 Dot 
        \psk@zeroLineTo\space 0 ge { % line to function \psk@lineTo
          xPixel 0 moveto
          xZero coeff \psk@zeroLineTo\space FuncValue 
          \pst@number\psyunit mul xPixel exch L
          \@nameuse{psls@\psk@zeroLineStyle}
        } if
      } repeat
      grestore
    \fi
    /x x0 def
    /xy {
      x \psk@xShift\space sub coeff Derivation FuncValue \pst@number\psyunit mul 
      x \pst@number\psxunit mul exch
    } def
    xy moveto
  }%
  \if@pst%	lines and dots
    \psPolynomial@ii%
  \else%	curves
    \psPolynomial@iii%
  \fi%
  \end@OpenObj
}\ignorespaces}
%
\def\psPolynomial@ii{%
  \addto@pscode{%
     xy \@nameuse{beginqp@\psplotstyle}
    \psk@plotpoints {
      xy \@nameuse{doqp@\psplotstyle}
      /x x dx add def
    } repeat
    xy \@nameuse{doqp@\psplotstyle}
    end
  }%
  \@nameuse{endqp@\psplotstyle}%
}
\def\psPolynomial@iii{%    curves
  \addto@pscode{%
    mark
    /n 2 def
    \psk@plotpoints {
      xy 
      n 2 roll
      /n n 2 add def
      /x x dx add def
    } repeat
    /x x1 def
    xy
    n 2 roll
    end
  }%
  \@nameuse{endplot@\psplotstyle}%
}
%
% Bessel 2004-06-08
% Manuel Luque, Herbert Voss
% Look at the end for some more documentation about the algorithm
%
\define@key[psset]{pst-func}{constI}{\def\psk@constI{#1 }}
\define@key[psset]{pst-func}{constII}{\def\psk@constII{#1 }}
\psset{constI=1,constII=0}
%
\def\psBessel{\@ifnextchar[{\psBessel@i}{\psBessel@i[]}}
\def\psBessel@i[#1]#2#3#4{{%%%  #2 = n
  \pst@killglue
  \psset{plotpoints=500}%
  \psset{#1}%
  \parametricplot{#3}{#4}{%
    /J1 0 def
    /k { 57.29577951 mul } def
    /xBessel t k def
    0 0.1 180 {
      /tB exch k def
      /J1 J1 0.1 xBessel
        tB sin mul tB #2\space mul sub cos mul add def
    } for
    t J1 180 div \psk@constI mul \psk@constII add
  }%
}\ignorespaces}
%
\define@key[psset]{pst-func}{sigma}{\def\psk@sigma{#1 }}
\define@key[psset]{pst-func}{mue}{\def\psk@mue{#1 }}
\psset[pst-func]{sigma=0.5,mue=0}
%
\def\psGauss{\@ifnextchar[{\psGauss@i}{\psGauss@i[]}}
\def\psGauss@i[#1]#2#3{{%
    \pst@killglue%
    \psset{plotpoints=200}%
    \psset{#1}%
    \psplot{#2}{#3}{%
        Euler x \psk@mue sub dup mul 2 div \psk@sigma dup mul div neg exp 
	1.0 \psk@sigma div TwoPi sqrt div mul%
    }%
}\ignorespaces}
%
\define@key[psset]{pst-func}{Simpson}{\def\psk@Simpson{#1 }}
\psset[pst-func]{Simpson=5}
%
\def\psGaussI{\pst@object{psGaussI}}
\def\psGaussI@i#1#2{%
  \begin@SpecialObj%
  \addto@pscode{
    /a #1  def
    /dx #2 #1 sub \psk@plotpoints\space div def
    /b a dx add def
    /scx { \pst@number\psxunit mul } def 
    /scy { \pst@number\psyunit mul } def
    tx@FuncDict begin 
    /Sigma 1 \psk@sigma div TwoPi sqrt div def
    /SFunc {% x on Stack
      Euler exch \psk@xShift\space sub dup mul 2 div Sigma dup mul div neg exp Sigma mul 
    } def 
    end
    a scx 0 moveto
    \psk@plotpoints 1 sub {
      a b \psk@Simpson % a b M on Styack
      tx@FuncDict begin Simpson I end % y value on stack
      scy b scx exch lineto 
      /b b dx add def
    } repeat
    stroke
  }%
  \end@SpecialObj%
}
%
\def\psSi{\pst@object{psSi}}
\def\psSi@i#1#2{%
  \begin@SpecialObj%
  \addto@pscode{
    /x #1  def
    /dx #2 #1 sub \psk@plotpoints\space div def
    /scx { \pst@number\psxunit mul } def
    /scy { \pst@number\psyunit mul } def
    x scx x tx@FuncDict begin Si end scy moveto
    \psk@plotpoints 1 sub {
      x dup scx exch tx@FuncDict begin Si end scy lineto
      /x x dx add def
    } repeat
    stroke
  }%
  \end@SpecialObj%
}
\def\pssi{\pst@object{pssi}}
\def\pssi@i#1#2{%
  \begin@SpecialObj%
  \addto@pscode{
    /x #1  def
    /dx #2 #1 sub \psk@plotpoints\space div def
    /scx { \pst@number\psxunit mul } def
    /scy { \pst@number\psyunit mul } def
    x scx x tx@FuncDict begin si end scy moveto
    \psk@plotpoints 1 sub {
      x dup scx exch tx@FuncDict begin si end scy lineto
      /x x dx add def
    } repeat
    stroke
  }%
  \end@SpecialObj%
}
%
\def\psCi{\pst@object{psCi}}
\def\psCi@i#1#2{%
  \begin@SpecialObj%
  \addto@pscode{
    /x #1  def
    /dx #2 #1 sub \psk@plotpoints\space div def
    /scx { \pst@number\psxunit mul } def
    /scy { \pst@number\psyunit mul } def
    x scx x tx@FuncDict begin Ci end scy moveto
    \psk@plotpoints 1 sub {
      x dup scx exch tx@FuncDict begin Ci end scy lineto
      /x x dx add def
    } repeat
    stroke
  }%
  \end@SpecialObj%
}
\def\psci{\pst@object{psci}}
\def\psci@i#1#2{%
  \begin@SpecialObj%
  \addto@pscode{
    /x #1  def
    /dx #2 #1 sub \psk@plotpoints\space div def
    /scx { \pst@number\psxunit mul } def
    /scy { \pst@number\psyunit mul } def
    x scx x tx@FuncDict begin ci end scy moveto
    \psk@plotpoints 1 sub {
      x dup scx exch tx@FuncDict begin ci end scy lineto
      /x x dx add def
    } repeat
    stroke
  }%
  \end@SpecialObj%
}
%
\define@key[psset]{pst-func}{PSfont}{\def\psk@PSfont{/#1 }}
\define@key[psset]{pst-func}{valuewidth}{\def\psk@valuewidth{#1 }}
\define@key[psset]{pst-func}{fontscale}{\def\psk@fontscale{#1 }}
\psset[pst-func]{PSfont=Times-Roman,fontscale=10,valuewidth=10}
%
\def\psPrintValue{\pst@object{psPrintValue}}
\def\psPrintValue@i#1{%
  \begin@ClosedObj
  \addto@pscode{  
     gsave \psk@PSfont findfont \psk@fontscale scalefont setfont 
     #1 \psk@valuewidth string cvs 0 0 moveto show grestore
  }%
  \end@ClosedObj%
}
%
% Integrals 2006-01-16
% Jose-Emilio Vila-Forcen, Herbert Voss
%
\def\psCumIntegral{\pst@object{psCumIntegral}}
\def\psCumIntegral@i#1#2#3{%
  \begin@SpecialObj%
  \addto@pscode{
    /a #1  def
    /dx #2 #1 sub \psk@plotpoints\space div def
    /b a dx add def
    /scx { \pst@number\psxunit mul } def
    /scy { \pst@number\psyunit mul } def
    tx@FuncDict begin /SFunc { #3 } def end
    a scx 0 moveto
    \psk@plotpoints 1 sub {
      a b \psk@Simpson % a b M on Styack
      tx@FuncDict begin Simpson I end % y value on stack
      scy b scx exch lineto
      /b b dx add def
    } repeat
    stroke
  }%
  \end@SpecialObj%
}
%
\def\psIntegral{\pst@object{psIntegral}}
\def\psIntegral@i#1#2(#3,#4)#5{%
  \begin@SpecialObj%
  \addto@pscode{
    /a #3  def
    /dx #4 #3 sub \psk@plotpoints\space div def
    /b #4 def
    /aa #1 def
    /dd #2 #1 sub \psk@plotpoints\space div def
    /t aa dd add def
    /scx { \pst@number\psxunit mul } def
    /scy { \pst@number\psyunit mul } def
    tx@FuncDict begin /SFunc { t #5 } def end
      a b \psk@Simpson % a b M on Stack
      tx@FuncDict begin Simpson I end % y value on stack
      scy t scx exch moveto
      /t t dd add def
    \psk@plotpoints 1 sub {
      a b \psk@Simpson % a b M on Stack
      tx@FuncDict begin Simpson I end % y value on stack
      scy t scx exch lineto
      /t t dd add def
    } repeat
    stroke
  }%
  \end@SpecialObj%
}
%
\def\psConv{\@ifnextchar[{\psConv@i}{\psConv@i[]}}
\def\psConv@i[#1]#2#3(#4,#5)#6#7{%
  \psIntegral[#1]{#2}{#3}(#4,#5){pop pop x #6\space x t neg add #7\space mul}%
}%
%
\define@boolkey[psset]{pst-func}[Pst@]{printValue}[true]{}
\define@key[psset]{pst-func}{barwidth}{\def\psFunc@barwidth{#1 }}% a factor, not a dimen
\psset[pst-func]{printValue=false,barwidth=1}
%
\def\psBinomial{\pst@object{psBinomial}}
\def\psBinomial@i#1#2{%
  \begin@SpecialObj%
  \addto@pscode{
    /scx { \pst@number\psxunit mul } def
    /scy { \pst@number\psyunit mul } def
    /N #1 def
    /p #2 def
    /dx \psFunc@barwidth 2 div def
    /q 1 p sub def
    /kOld dx neg def
    kOld scx 0 moveto   % starting point
%    /Y 1 def   % start value
    0 1 N {             % N times
      /k exch def       % save loop variable
      k 0 eq
        { /Y q N exp def }
        { /Y Y N k sub 1 add mul k div p mul q div def }
      ifelse % recursive definition
      kOld scx 0 L kOld scx Y scy L k dx add scx Y scy L
      \ifPst@markZeros k dx add scx 0 L \fi
      \ifPst@printValue
        gsave \psk@PSfont findfont \psk@fontscale scalefont setfont
        Y \psk@valuewidth string cvs
        k scx \psk@fontscale 2 div add
        Y scy \pst@number\pslabelsep add moveto
        90 rotate show grestore
      \fi
      /kOld kOld 1 add def
    } for
  }%
  \psk@fillstyle
  \pst@stroke
  \end@SpecialObj%
}
%
\def\psBinomialN{\pst@object{psBinomialN}}
\def\psBinomialN@i#1#2{%
  \pst@killglue
  \begin@SpecialObj%
  \def\cplotstyle{curve}%
  \ifx\psplotstyle\cplotstyle \@nameuse{beginplot@\psplotstyle}\fi%
  \addto@pscode{
    \ifx\psplotstyle\cplotstyle /Curve true def \else /Curve false def \fi
    /scx { \pst@number\psxunit mul } def 
    /scy { \pst@number\psyunit mul } def
    /N #1 def 
    /p #2 def				% probability
    /q 1 p sub def
    /E N p mul def
    /sigma E q mul sqrt def 		% variant
    /dx 1.0 sigma div 2 div def
    /xOld dx neg E sub sigma div def
    /xEnd xOld neg dx add scx def
    Curve 
      { /Coors [xOld dx sub scx 0] def }% saves the coordinates for curve
      { xOld scx 0 moveto }	% starting point
    ifelse 
   0 1 N {				% N times
      /k exch def			% save loop variable
      k 0 eq 
        { /Y q N exp def }
        { /Y Y N k sub 1 add mul k div p mul q div def }
      ifelse % recursive definition
      /x k E sub sigma div dx add def
      /y Y sigma mul def		% normalize
      Curve 
        { x dx sub scx y scy Coors aload length 2 add array astore /Coors exch def}
        { xOld scx y scy L x scx y scy L
          \ifPst@markZeros x scx 0 L \fi % 
        } ifelse
      \ifPst@printValue 
        gsave \psk@PSfont findfont \psk@fontscale scalefont setfont 
        y \psk@valuewidth string cvs 
        x dx sub scx \psk@fontscale 2 div add 
        y scy \pst@number\pslabelsep add moveto 
        90 rotate show grestore
      \fi
      /xOld x def
    } for
    Curve { [ xEnd 0 Coors aload pop} if % showpoints on top of the stack
  }%
  \ifx\psplotstyle\cplotstyle \@nameuse{endplot@\psplotstyle} \else%
    \psk@fillstyle%
    \pst@stroke%
  \fi%
  \end@SpecialObj%
  \ignorespaces%
}
%
% Superellipese / Lamefunction
\define@key[psset]{pst-func}{radiusA}{\pst@getlength{#1}\pst@radiusA}
\define@key[psset]{pst-func}{radiusB}{\pst@getlength{#1}\pst@radiusB}
\psset[pst-func]{radiusA=1,radiusB=1}
%
\def\psLame{\pst@object{psLame}}
\def\psLame@i#1{%
  \addbefore@par{plotpoints=200}
  \pst@killglue
  \begin@SpecialObj 
  \parametricplot{0}{360}{%
     t cos dup mul 1 #1\space div exp \pst@radiusA \pst@number\psxunit div mul 
     t 90 gt { t 270 lt { neg } if } if
     t sin dup mul 1 #1\space div exp \pst@radiusB \pst@number\psyunit div mul 
     t 180 gt { neg } if }
  \end@SpecialObj 
}
%
% For polar plots
%\define@boolkey[psset]{pst-func}[PstAdd@]{polarplot}[true]{}
%\define@boolkey[psset]{pst-func}[PstAdd@]{algebraic}[true]{}
%\psset[pst-func]{polarplot=false,algebraic=false}
%
\def\psplotImp{\pst@object{psplotImp}}% 20060420
\def\psplotImp@i(#1,#2)(#3,#4)#5{%
  \pst@killglue
  \begingroup
  \begin@SpecialObj%
  \addto@pscode{
    /xMin #1 def
    /xMax #3 def
    /yMin #2 def
    /yMax #4 def
    \ifPst@polarplot 
      /@PolarAlgPlot (#5) tx@addDict begin AlgParser end cvx def
      /Func {
        /phi y x atan def
        /r x y Pyth def 
        \ifPst@algebraic @PolarAlgPlot \else #5 \fi } def 
    \else
      /Func \ifPst@algebraic (#5) tx@addDict begin AlgParser end cvx \else { #5 } \fi  def
    \fi
    /xPixel xMax xMin sub \pst@number\psxunit mul round cvi def
    /yPixel yMax yMin sub \pst@number\psxunit mul round cvi def
    /dx xMax xMin sub xPixel div def
    /dy yMax yMin sub yPixel div def
    /setpixel { dy div exch dx div exch \pst@number\pslinewidth 2 div 0 360 arc fill } bind def
%
    /VZ true def % suppose that F(x,y)>=0
    /x xMin def /y yMin def Func 0.0 lt { /VZ false def } if % erster Wert
    xMin dx 1.5 div xMax {
      /x exch def
      yMin dy 1.5 div yMax {
	/y exch def
	Func 0 lt 
	    { VZ { x y setpixel /VZ false def} if }
	    { VZ {}{ x y setpixel /VZ true def } ifelse } ifelse
      } for
    } for
%
    /x xMin def /y yMin def Func 0.0 lt { /VZ false def } if % erster Wert
    yMin dy 1.5 div yMax {
      /y exch def
      xMin dx 1.5 div xMax {
	/x exch def
	Func 0 lt 
	    { VZ { x y setpixel /VZ false def} if }
	    { VZ {}{ x y setpixel /VZ true def } ifelse } ifelse
      } for
    } for
  }%
  \end@SpecialObj%
  \endgroup
  \ignorespaces
}
%
\catcode`\@=\PstAtCode\relax
%
%% END: pst-func.tex
\endinput
%
