%M \setuphead[section]  [page=]

%D \module
%D   [       file=t-mathsets,
%D        version=2007-02-25,
%D          title=Math Sets,
%D       subtitle=\CONTEXT\ port of \filename{braket.sty},
%D         author={Aditya Mahajan},
%D          email={adityam [at] umich [dot] edu},
%D           date=\currentdate]
  
%M \usemodule[mathsets] 

%D \section Introduction
%D
%D I write a lot of probability expressions, which look like
%D \startformula
%D \mfunction{E} \left\{ \sum_{y} f(X,y) \,\middle|\, Z \right\}
%D \stopformula
%D The delimiters should scale properly, and so should the {\em
%D conditional} sign \type{|}. Moreover the spacing around the
%D conditional sign should be correct. This ensures that the resultant
%D \TEX\ code is almost unreadable.  In \LATEX\ I used to use Donald
%D Arseneau's \filename{braket.sty} to typeset should expressions.
%D \CONTEXT\ does not have anything similar. So, this is a port of
%D \filename{braket.sty} functionality to \CONTEXT. I have not ported
%D everything, only the features that I use. 
%D
%D \section Usage
%D
%D To use this module add
%D \starttyping
%D \usemodule[mathsets]
%D \stoptyping
%D on the top of your file. Now, a new {\em set} can be defined as follows:
%D \startbuffer[EXP]
%D \definemathset[EXP] [text=\mfunction{E}]
%D \definemathset[PR]  [text={\mfunction{Pr}},left=(,right=)]
%D \stopbuffer
%D After which you can use 
%D \startbuffer
%D \startformula
%D    \EXP{f(X) | Y} = \sum_{x} f(x) \PR{x|Y}
%D \stopformula
%D \stopbuffer
%D \typebuffer
%D \getbuffer[EXP] \getbuffer
%D
%D We can also run the example specified in \filename{braket.sty}
%D documentation.
%D
%D \startbuffer
%D \definemathset[BRAKET][left=\langle,right=\rangle]
%D \definemathset[SET]
%D
%D \startformula
%D   \BRAKET{ \phi | \frac{\partial^2}{\partial t^2} | \psi }
%D   \SET{ x\in {\bf R}^2 | 0<{|x|}<5 }
%D \stopformula
%D \stopbuffer
%D \typebuffer \getbuffer
%D Notice that the \type{|} protected by \type<<{|}>> did not get
%D expanded in the second expression.


%D \section Implementation
%D

\unprotect

%D Since two letter codes are reserved for system modules, and \CONTEXT\
%D seems to be running out of those, I choose a more verbose variable to
%D store options.

\definesystemvariable {mathset}   % Math Set

%D \macros{setupmathset}
%D To specify the default values of left, middle, and right delimiters

\def\setupmathset
  {\dosingleargument\getparameters[\??mathset]}

\def\definemathset
  {\dodoubleargument\dodefinemathset}

%D \macros{definemathset}
%D To define new math delimiters

\let\currentmathset\empty
\let\currentmathsetgrouplevel\empty

\def\mathsetmiddle
  {\ifnum\currentmathsetgrouplevel=\currentgrouplevel
      \expandafter\firstoftwoarguments
    \else
      \expandafter\secondoftwoarguments
    \fi
   {\egroup\,\middle\mathsetparameter\c!middle\,\bgroup}
   {\mathsetparameter\c!middle}}

\def\mathsetparameter#1%
  {\executeifdefined{\??mathset\currentmathset#1}{\executeifdefined{\??mathset#1}\empty}}

\def\dodefinemathset[#1][#2]%
  {\getparameters[\??mathset#1][#2]
   \setvalue{#1}{\dododefinemathset[#1]}}

%D Since \type{|} is already active, we do not have to make it active
%D again.

\def\dododefinemathset[#1]#2#%
  {\begingroup
   \def\currentmathset{#1}
   \edef\currentmathsetgrouplevel{\the\numexpr\currentgrouplevel+2\relax}
   \mathcode`\|32768
   \let|\mathsetmiddle
   \def\mathsetarguments{#2}
   \dodododefinemathset}

%D The extra group in the definition of \type{dodododefinemathset} is
%D so that such expressions turn out correct
%D \getbuffer[EXP]
%D \startformula
%D \EXP{ \left(\frac {a}{b}\right) | 
%D      \left( \frac{a} {\frac{b}{\sum c}} \right) }
%D \stopformula

\def\dodododefinemathset#1%
  {\doifsomething{\mathsetparameter\c!text}
    {\mathop{\mathsetparameter\c!text\mathsetarguments}}
   \left\mathsetparameter\c!left{#1}\right\mathsetparameter\c!right
   \endgroup}

\setupmathset
  [  left={\{},
    right={\}},
   middle=\vert]

\protect

