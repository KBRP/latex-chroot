%D \module
%D   [       file=supp-pat,
%D        version=2005.02.12,
%D          title=\CONTEXT\ Language Macros,
%D       subtitle=Loading (Generic) Patterns,
%D         author=Hans Hagen,
%D           date=\currentdate,
%D      copyright={PRAGMA / Hans Hagen \& Ton Otten}]
%C
%C This module is part of the \CONTEXT\ macro||package and is
%C therefore copyrighted by \PRAGMA. See mreadme.pdf for
%C details.

%D We support:
%D
%D \starttyping
%D \patterns {
%D  ...[namedglyphs]...
%D }
%D \stoptyping
%D
%D as well as:
%D
%D \starttyping
%D <patterns>
%D  ...&namedglyphs;...
%D </patterns>
%D \stoptyping
%D
%D The \XML\ variant has very limited \XML\ parsing capabilities and
%D is meant as an experiment.

\ifx\writestatus\undefined \else

    \writestatus{loading}{Context Language Macros / Loading Generic Patterns}

\fi

%D \CONTEXT:

\ifx\startmapping\undefined \else \ifx\startencoding\undefined \else

    \def\fetchpatterntoken#1{\csname#1\endcsname}

\fi \fi

%D First we define the \ETEX\ variant. We cannot do that inside
%D the if statement because normal \TEX\ gets confused over the
%D csname test (which it does not see as if but it does see the
%D else's and fi's).

\def\etexfetchpatterntoken#1%
  {\csname
     \ifcsname @pattok@#1\endcsname
       @pattok@#1%
     \else\ifcsname#1\endcsname
       #1%
     \else
       @pattok@default%
     \fi\fi
  \endcsname}

\ifx \fetchpatterntoken \undefined

    \ifx \eTeXversion \undefined

        \def\fetchpatterntoken#1%
          {\csname
             \expandafter\ifx\csname @pattok@#1\endcsname\relax
               \expandafter\ifx\csname#1\endcsname\relax
                 @pattok@default%
               \else
                 #1%
               \fi
             \else
                @pattok@#1%
             \fi
          \endcsname}

    \else

        \let\fetchpatterntoken\etexfetchpatterntoken

    \fi

\fi

\def\definepatterntoken #1 #2 %
  {\expandafter\def\csname @pattok@#1\endcsname{#2}}

%D \TEX\ based:

\bgroup \catcode`\[=\active

  \gdef\enablepatterntokens
    {\catcode`\[=\active \let[\handlepatterntoken}

\egroup

\def\handlepatterntoken#1]%
  {\fetchpatterntoken{#1}}

%D \XML\ based:

\bgroup \catcode`\<=\active \catcode`\&=\active

  \gdef\enablepatternxml
    {\catcode`\<=\active
     \catcode`\&=\active
     \let&\handlepatternentity
     \let<\handlepatternxml}

\egroup

\def\handlepatternentity#1;%
  {\fetchpatterntoken{#1}}

%D We use a very simple \XML\ parser, and we only trigger on the first
%D two characters of the tag name. Consider this feature a playground
%D (and a bit of a joke).

\def\handlepatternxml#1#2% we only distinguish the first two chars
  {\csname @patxml@\ifcsname @patxml@#1#2\endcsname#1#2\else de\fi\endcsname}

\def\definepatternxml #1#2#3 #4%
  {\expandafter\def\csname @patxml@#1#2\endcsname{#4}}

\def\fetchpatternxml#1#2>%
  {\bgroup
   \long\def\next##1>{\egroup}%
   \catcode`\<=2 % \@@endgroup
   \afterassignment\next
   #1\bgroup}

\def\gobblepatternxml
  {\bgroup
   \catcode`\%=11
   \dogobblepatternxml}

\long\def\dogobblepatternxml#1>%
  {\egroup}

%D Here are the default definitions. We don't catch empty elements.

\definepatternxml patterns     {\fetchpatternxml\patterns}
\definepatternxml hyphenations {\fetchpatternxml\hyphenation}
\definepatternxml pattern      {\fetchpatternxml\patterns}
\definepatternxml hyphenation  {\fetchpatternxml\hyphenation}
\definepatternxml default      {\gobblepatternxml}

\definepatterntoken default x

\ifx\pattern     \undefined \let\pattern     \patterns    \fi
\ifx\hyphenations\undefined \let\hyphenations\hyphenation \fi

%D An example of usage in another format than \CONTEXT:
%D
%D \starttyping
%D \bgroup
%D
%D   \lccode"E4="E4  \definepatterntoken adiaeresis ^^e4
%D   \lccode"F6="F6  \definepatterntoken odiaeresis ^^f6
%D   \lccode"FC="FC  \definepatterntoken ediaeresis ^^fc
%D   \lccode"FF="FF  \definepatterntoken ssharp     ^^ff
%D
%D   \enablepatterntokens
%D   \enablepatternxml
%D
%D   \input lang-de.pat
%D   \input lang-de.hyp
%D
%D \egroup
%D \stoptyping
%D
%D This is a minimal example for EC encoding, and of course
%D a couple of more language related settings need to take
%D place (like uc mappings).

\endinput
