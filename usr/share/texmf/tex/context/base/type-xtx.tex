%D \module
%D   [       file=type-xtx,
%D        version=2004.11.15, % prereleased earlier
%D          title=\CONTEXT\ Typescript Macros,
%D       subtitle=\XETEX's font treasures,
%D         author=Adam T. Lindsay,
%D           date=\currentdate,
%D      copyright={Adam T. Lindsay / PRAGMA}]
%C
%C This module is part of the \CONTEXT\ macro||package and is
%C therefore copyrighted by \PRAGMA. See mreadme.pdf for
%C details.

%D Here are some fonts definitions that can get you started with
%D \XETEX (for more details see Adam's MyWay documents).
%D
%D Most typescripts in this file are mostly independent of the other
%D typescript files. Generally, you can speed things up a lot by
%D eliminating all but one of \CONTEXT's typescript files:
%D
%D \starttyping
%D \usetypescriptfiles[reset]     % HH: watch out, new feature, since
%D \usetypescriptfiles[type-siz]  % I disliked the low level redef.
%D \stoptyping
%D
%D The exceptions are the \quotation {legacy} fonts Times, Palatino,
%D Courier, and Helvetica, which also depend on \type {type-syn}.
%D
%D These following six typescripts call the basic four variants on any
%D given font, given the name of the \quotation {Regular} variant in the
%D name slot. These typescripts default to a Unicode encoding,
%D accepts sizes \quotation {default} and \quotation {dtp}, and are
%D activated with the identifiers \quotation {Xserif}, \quotation {Xsans},
%D and \quotation {Xmono}. They can have relative scaling within the
%D typeface. Any of the following work:
%D
%D \starttyping
%D \definetypeface[basic][rm][Xserif][Baskerville]
%D \definetypeface[basic][ss][Xsans] [Optima Regular][default][encoding=uc,rscale=.87]
%D \definetypeface[basic][tt][Xmono] [Courier]       [default]
%D \stoptyping
%D
%D Activate the typeface with:
%D
%D \starttyping
%D \setupbodyfont[basic]
%D \stoptyping

%D This file is hacked by Taco Hoekwater in an attempt to figure out the right approach
%D to font loading in \XeTeX.  (jun19,2007)

%D The General \XeTeX\ low-level font syntax is (at least) as follows.
%D
%D Named font:
%D \starttyping
%D \font\x = "<fontname><engine-options>:<featurelist>" <at or scaled>
%D \stoptyping
%D \type{<fontname>} = Font name as seen in a system font menu or the output of fc-list
%D
%D \type{<engine-options>} = \type{/B} or \type{/I} or \tupe{/BI}, and||or \type{/S=<X>}.
%D That last one selects an optical scaled variant for size \type{<X>}
%D (it is a bare number, the unit is points).
%D
%D \type{<featurelist>} = comma- or semicolon- separated list of font features.
%D
%D Opentype features are selected using \type{+<tag>}, and deselected using \type{-<tag>},
%D except that key||value pairs are used for \type{script=<tag>} and \type{language=<tag>}
%D
%D AAT features are always key||value pairs, often including spaces.
%D
%D \XETEX's own features are key||value pairs, and can be applied to both OpenType
%D and AAT fonts:
%D \type{mapping=<font map>} for glyph remapping
%D \type{color=RRGGBB[TT]} for color (hex numbers, with  optional transparancy),
%D \type{letterspace=<x>} to add \type{<x>/<fontsize>}  intercharacter spacing.
%D
%D Full example showing all parts of the syntax for an OTF font:
%D \starttyping
%D \font\f= "Warnock Pro/I/S=5:+smcp,-liga,mapping=tex-text,script=latn"
%D \stoptyping

%D Non-installed (filename-based) fonts :
%D
%D \starttyping
%D \font\x = "[<fontname>]:<featurelist>" <at or scaled>
%D \stoptyping
%D
%D Here, there are no \type{<engine-options>}, because there is no
%D font discovery engine available to be queried.


%D TH: This \type{\xetexcolon} definition seems needed because the name/file \
%D parser otherwise drops the rest of the argument into oblivion.

\unexpanded\def\xetexcolon{:}

\starttypescriptcollection[xetex]

\starttypescript[Xserif][all][name]

%D TH: I removed all single quotes because they don't seem to add anything. And I added
%D the \type{name:} everywhere, because passing that information on \type{\typescripttwo}
%D doesn't work either (maybe  \type{\typescripttwo} is expanded incorrectly)
%D
%D And even if that would have worked, \type{file:} will not work properly anyway
%D in this case since tricks like \type{/I} will never, ever work for local fonts,
%D so there is really only one choice.

%D HH: todo, define feature set switch mapping=tex-tex

\definefontsynonym[Dummy]          [name:\typescripttwo\xetexcolon mapping=tex-text]   [encoding=uc]
\definefontsynonym[DummyItalic]    [name:\typescripttwo/I\xetexcolon mapping=tex-text] [encoding=uc]
\definefontsynonym[DummyBold]      [name:\typescripttwo/B\xetexcolon mapping=tex-text] [encoding=uc]
\definefontsynonym[DummyBoldItalic][name:\typescripttwo/BI\xetexcolon mapping=tex-text][encoding=uc]

\definefontsynonym[Serif]           [Dummy]
\definefontsynonym[SerifBold]       [DummyBold]
\definefontsynonym[SerifItalic]     [DummyItalic]
\definefontsynonym[SerifBoldItalic] [DummyBoldItalic]
\definefontsynonym[SerifSlanted]    [DummyItalic]
\definefontsynonym[SerifBoldSlanted][DummyBoldItalic]
\definefontsynonym[SerifCaps]       [Dummy]

\stoptypescript

\starttypescript[Xsans][all][name]

\definefontsynonym[DummySans]          [name:\typescripttwo\xetexcolon mapping=tex-text]   [encoding=uc]
\definefontsynonym[DummySansItalic]    [name:\typescripttwo/I\xetexcolon mapping=tex-text] [encoding=uc]
\definefontsynonym[DummySansBold]      [name:\typescripttwo/B\xetexcolon mapping=tex-text] [encoding=uc]
\definefontsynonym[DummySansBoldItalic][name:\typescripttwo/BI\xetexcolon mapping=tex-text][encoding=uc]

\definefontsynonym[Sans]           [DummySans]
\definefontsynonym[SansBold]       [DummySansBold]
\definefontsynonym[SansItalic]     [DummySansItalic]
\definefontsynonym[SansBoldItalic] [DummySansBoldItalic]
\definefontsynonym[SansSlanted]    [DummySansItalic]
\definefontsynonym[SansBoldSlanted][DummySansBoldItalic]
\definefontsynonym[SansCaps]       [DummySans]

\stoptypescript

\starttypescript[Xmono][all][name]

\definefontsynonym[DummyMono]          [name:\typescripttwo]   [encoding=uc]
\definefontsynonym[DummyMonoItalic]    [name:\typescripttwo/I] [encoding=uc]
\definefontsynonym[DummyMonoBold]      [name:\typescripttwo/B] [encoding=uc]
\definefontsynonym[DummyMonoBoldItalic][name:\typescripttwo/BI][encoding=uc]

\definefontsynonym[Mono]           [DummyMono]
\definefontsynonym[MonoBold]       [DummyMonoBold]
\definefontsynonym[MonoItalic]     [DummyMonoItalic]
\definefontsynonym[MonoBoldItalic] [DummyMonoBoldItalic]
\definefontsynonym[MonoSlanted]    [DummyMonoItalic]
\definefontsynonym[MonoBoldSlanted][DummyMonoBoldItalic]
\definefontsynonym[MonoCaps]       [DummyMono]

\stoptypescript

\starttypescript[Xserif][default][size]
  \definebodyfont
  [4pt,5pt,6pt,7pt,8pt,9pt,10pt,11pt,12pt,14.4pt,17.3pt] [rm]
  [default]
\stoptypescript

\starttypescript[Xsans][default][size]
  \definebodyfont
    [4pt,5pt,6pt,7pt,8pt,9pt,10pt,11pt,12pt,14.4pt,17.3pt]
    [ss] [default]
\stoptypescript

\starttypescript [Xmono][default][size]
  \definebodyfont
    [4pt,5pt,6pt,7pt,8pt,9pt,10pt,11pt,12pt,14.4pt,17.3pt]
    [tt] [default]
\stoptypescript

\starttypescript[Xserif][dtp][size]
  \definebodyfont
    [5pt,6pt,7pt,8pt,9pt,10pt,11pt,12pt,13pt,14pt,16pt,18pt,22pt,28pt]
    [rm] [default]
\stoptypescript

\starttypescript[Xsans][dtp][size]
  \definebodyfont
    [5pt,6pt,7pt,8pt,9pt,10pt,11pt,12pt,13pt,14pt,16pt,18pt,22pt,28pt]
    [ss] [default]
\stoptypescript

\starttypescript[Xmono][dtp][size]
  \definebodyfont
    [5pt,6pt,7pt,8pt,9pt,10pt,11pt,12pt,13pt,14pt,16pt,18pt,22pt,28pt]
    [tt] [default]
\stoptypescript

%D The following are \quotation {legacy} named fonts. Times, Palatino,
%D and Helvetica are familiar to most users of modern \TEX\
%D systems. These versions are accessed via the Unicode encoding
%D enabled by \XETEX. There is no attempt to match metrics with
%D the actual legacy fonts. These are simply familiar names.

%D These typescripts, unlike others in this file, depend on those in
%D \type{type-pre}.

\starttypescript[serif][times][uc]

\definefontsynonym[Times-Roman]     [name:Times Roman\xetexcolon mapping=tex-text]       [encoding=uc]
\definefontsynonym[Times-Italic]    [name:Times Italic\xetexcolon mapping=tex-text]      [encoding=uc]
\definefontsynonym[Times-Bold]      [name:Times Bold\xetexcolon mapping=tex-text]        [encoding=uc]
\definefontsynonym[Times-BoldItalic][name:Times Bold Italic\xetexcolon mapping=tex-text;][encoding=uc]

\stoptypescript

%D Book Antiqua is Mac OS X's Palatino clone.

\starttypescript[serif][palatino][uc]

\definefontsynonym[Palatino]            [name:Book Antiqua\xetexcolon mapping=tex-text]            [encoding=uc]
\definefontsynonym[Palatino-Italic]     [name:Book Antiqua Italic\xetexcolon mapping=tex-text]     [encoding=uc]
\definefontsynonym[Palatino-Bold]       [name:Book Antiqua Bold\xetexcolon mapping=tex-text]       [encoding=uc]
\definefontsynonym[Palatino-BoldItalic] [name:Book Antiqua Bold Italic\xetexcolon mapping=tex-text][encoding=uc]

\definefontsynonym[Palatino-Slanted]    [Palatino-Italic]
\definefontsynonym[Palatino-BoldSlanted][Palatino-BoldItalic]
\definefontsynonym[Palatino-Caps]       [Palatino]

\stoptypescript

%D The default Helvetica doesn't have an oblique variant, so we'll
%D go ahead and name Helvertica Neue here.

\starttypescript[sans][helvetica][uc]

\definefontsynonym[Helvetica]            [name:Helvetica Neue\xetexcolon mapping=tex-text]            [encoding=uc]
\definefontsynonym[Helvetica-Oblique]    [name:Helvetica Neue Italic\xetexcolon mapping=tex-text]     [encoding=uc]
\definefontsynonym[Helvetica-Bold]       [name:Helvetica Neue Bold\xetexcolon mapping=tex-text]       [encoding=uc]
\definefontsynonym[Helvetica-BoldOblique][name:Helvetica Neue Bold Italic\xetexcolon mapping=tex-text][encoding=uc]

\stoptypescript

%D Courier, as delivered on MacOSX 10.3, doesn't have an oblique
%D variant, either. Unfortunately, none of the default Mono fonts in
%D MacOSX have oblique|/|italic versions!

\starttypescript[mono][courier][uc]

\definefontsynonym[Courier]            [name:Courier\xetexcolon mapping=tex-text]     [encoding=uc]
\definefontsynonym[Courier-Oblique]    [Courier]
\definefontsynonym[Courier-Bold]       [name:Courier Bold\xetexcolon mapping=tex-text][encoding=uc]
\definefontsynonym[Courier-BoldOblique][Courier-Bold]

\stoptypescript

%D The following fonts go beyond the usual four variants that
%D are accessible via the above wildcard typescripts, so they
%D get a more expanded treatment here\xetexcolon

\starttypescript[serif][hoefler][uc]

\definefontsynonym[Hoefler]      [name:Hoefler Text\xetexcolon mapping=tex-text;%
                   Ligatures=Diphthongs]  [encoding=uc]
\definefontsynonym[HoeflerItalic][name:Hoefler Text Italic\xetexcolon mapping=tex-text;%
                   Ligatures=Diphthongs]  [encoding=uc]
\definefontsynonym[HoeflerBlack] [name:Hoefler Text Black\xetexcolon mapping=tex-text;%
                   Ligatures=Diphthongs]  [encoding=uc]
\definefontsynonym[HoeflerBlackItalic][name:Hoefler Text Black Italic\xetexcolon mapping=tex-text;%
                   Ligatures=Diphthongs]  [encoding=uc]
\definefontsynonym[HoeflerSmCap] [name:Hoefler Text\xetexcolon mapping=tex-text;%
                   Ligatures=Diphthongs;%
                   Letter Case=Small Caps][encoding=uc]
\stoptypescript

\starttypescript[serif][hoefler][name]

\definefontsynonym[Serif]           [Hoefler]
\definefontsynonym[SerifBold]       [HoeflerBlack]
\definefontsynonym[SerifItalic]     [HoeflerItalic]
\definefontsynonym[SerifBoldItalic] [HoeflerBlackItalic]
\definefontsynonym[SerifSlanted]    [HoeflerItalic]
\definefontsynonym[SerifBoldSlanted][HoeflerBlackItalic]
\definefontsynonym[SerifCaps]       [HoeflerSmCap]

\stoptypescript

\starttypescript[sans][lucidagrande][uc]

\definefontsynonym[LucidaGrande]    [name:Lucida Grande\xetexcolon mapping=tex-text]     [encoding=uc]
\definefontsynonym[LucidaGrandeBold][name:Lucida Grande Bold\xetexcolon mapping=tex-text][encoding=uc]

\stoptypescript

\starttypescript[sans][lucidagrande][name]

\definefontsynonym[Sans]           [LucidaGrande]
\definefontsynonym[SansBold]       [LucidaGrandeBold]
\definefontsynonym[SansItalic]     [LucidaGrande]
\definefontsynonym[SansBoldItalic] [LucidaGrandeBold]
\definefontsynonym[SansSlanted]    [LucidaGrande]
\definefontsynonym[SansBoldSlanted][LucidaGrandeBold]
\definefontsynonym[SansCaps]       [LucidaGrande]

\stoptypescript

\starttypescript[sans][optima][uc]
\definefontsynonym[Optima]          [name:Optima Regular\xetexcolon mapping=tex-text]    [encoding=uc]
\definefontsynonym[OptimaItalic]    [name:Optima Italic\xetexcolon mapping=tex-text]     [encoding=uc]
\definefontsynonym[OptimaBold]      [name:Optima Bold\xetexcolon mapping=tex-text]       [encoding=uc]
\definefontsynonym[OptimaBoldItalic][name:Optima Bold Italic\xetexcolon mapping=tex-text][encoding=uc]
\definefontsynonym[OptimaBlack]     [name:Optima ExtraBlack\xetexcolon mapping=tex-text] [encoding=uc]
\stoptypescript

\starttypescript[sans][optima][name]

\definefontsynonym[Sans]           [Optima]
\definefontsynonym[SansBold]       [OptimaBold]
\definefontsynonym[SansItalic]     [OptimaItalic]
\definefontsynonym[SansBoldItalic] [OptimaBoldItalic]
\definefontsynonym[SansSlanted]    [OptimaItalic]
\definefontsynonym[SansBoldSlanted][OptimaBoldItalic]
\definefontsynonym[SansCaps]       [Optima]

\stoptypescript

\starttypescript[sans][gillsans,gillsanslt][uc]

\definefontsynonym[GillSans]           [name:Gill Sans\xetexcolon mapping=tex-text]             [encoding=uc]
\definefontsynonym[GillSansItalic]     [name:Gill Sans Italic\xetexcolon mapping=tex-text]      [encoding=uc]
\definefontsynonym[GillSansBold]       [name:Gill Sans Bold\xetexcolon mapping=tex-text]        [encoding=uc]
\definefontsynonym[GillSansBoldItalic] [name:Gill Sans Bold Italic\xetexcolon mapping=tex-text] [encoding=uc]
\definefontsynonym[GillSansLight]      [name:Gill Sans Light\xetexcolon mapping=tex-text]       [encoding=uc]
\definefontsynonym[GillSansLightItalic][name:Gill Sans Light Italic\xetexcolon mapping=tex-text][encoding=uc]

\stoptypescript

\starttypescript[sans][gillsans][name]

\definefontsynonym[Sans]           [GillSans]
\definefontsynonym[SansBold]       [GillSansBold]
\definefontsynonym[SansItalic]     [GillSansItalic]
\definefontsynonym[SansBoldItalic] [GillSansBoldItalic]
\definefontsynonym[SansSlanted]    [GillSansItalic]
\definefontsynonym[SansBoldSlanted][GillSansBoldItalic]
\definefontsynonym[SansCaps]       [GillSans]

\stoptypescript

\starttypescript[sans][gillsanslt][name]

\definefontsynonym[Sans]           [GillSansLight]
\definefontsynonym[SansBold]       [GillSans]
\definefontsynonym[SansItalic]     [GillSansLightItalic]
\definefontsynonym[SansBoldItalic] [GillSansItalic]
\definefontsynonym[SansSlanted]    [GillSansLightItalic]
\definefontsynonym[SansBoldSlanted][GillSansItalic]
\definefontsynonym[SansCaps]       [GillSansLight]

\stoptypescript

\starttypescript[serif,handwriting][zapfino][uc]

\definefontsynonym[ZapfinoOne]  [name:Zapfino\xetexcolon mapping=tex-text]                  [encoding=uc]
\definefontsynonym[ZapfinoTwo]  [name:Zapfino\xetexcolon mapping=tex-text;%
                                 Stylistic Variants=First variant glyph set] [encoding=uc]
\definefontsynonym[ZapfinoThree][name:Zapfino\xetexcolon mapping=tex-text;%
                                 Stylistic Variants=Second variant glyph set][encoding=uc]
\definefontsynonym[ZapfinoFour] [name:Zapfino\xetexcolon mapping=tex-text;%
                                 Stylistic Variants=Third variant glyph set] [encoding=uc]
\stoptypescript

\starttypescript[handwriting][zapfino][name]

\definefontsynonym[Handwriting][ZapfinoOne]

\stoptypescript

\starttypescript[serif][zapfino][name]

\definefontsynonym[Serif]           [ZapfinoOne]
\definefontsynonym[SerifBold]       [ZapfinoThree]
\definefontsynonym[SerifItalic]     [ZapfinoTwo]
\definefontsynonym[SerifBoldItalic] [ZapfinoTwo]
\definefontsynonym[SerifSlanted]    [ZapfinoThree]
\definefontsynonym[SerifBoldSlanted][ZapfinoThree]
\definefontsynonym[SerifCaps]       [ZapfinoOne]

\stoptypescript

\starttypescript[serif,calligraphy][applechancery][uc]

\definefontsynonym[AppleChanceryOne]    [name:Apple Chancery\xetexcolon mapping=tex-text;%
                   Number Case=Old Styles]                [encoding=uc]
\definefontsynonym[AppleChanceryTwo]    [name:Apple Chancery\xetexcolon mapping=tex-text;%
                   Number Case=Old Styles;%
                   Design Complexity=Elegant Design Level][encoding=uc]
\definefontsynonym[AppleChanceryThree]  [name:Apple Chancery\xetexcolon mapping=tex-text;%
                   Number Case=Old Styles;%
                   Design Complexity=Flourishes Set A]    [encoding=uc]
\definefontsynonym[AppleChanceryFour]   [name:Apple Chancery\xetexcolon mapping=tex-text;%
                   Number Case=Old Styles;%
                   Design Complexity=Flourishes Set B]    [encoding=uc]
\definefontsynonym[AppleChanceryCaps]   [name:Apple Chancery\xetexcolon mapping=tex-text;%
                   Number Case=Old Styles;%
                   Letter Case=Small Caps]                [encoding=uc]
\definefontsynonym[AppleChanceryCapsTwo][name:Apple Chancery\xetexcolon mapping=tex-text;%
                   Number Case=Old Styles;%
                   Letter Case=Small Caps;%
                   Design Complexity=Flourishes Set B]    [encoding=uc]
\stoptypescript

\starttypescript[calligraphy][applechancery][name]

\definefontsynonym[Calligraphy][AppleChanceryOne]

\stoptypescript

\starttypescript[serif][applechancery][name]

\definefontsynonym[Serif]           [AppleChanceryOne]
\definefontsynonym[SerifBold]       [AppleChanceryThree]
\definefontsynonym[SerifItalic]     [AppleChanceryTwo]
\definefontsynonym[SerifBoldItalic] [AppleChanceryFour]
\definefontsynonym[SerifSlanted]    [AppleChanceryThree]
\definefontsynonym[SerifBoldSlanted][AppleChanceryFour]
\definefontsynonym[SerifCaps]       [AppleChanceryCaps]

\stoptypescript

% MS Office 2004 for Mac has impressive Unicode coverage in
% many of its fonts.

\starttypescript[serif][timesnewroman][uc]

\definefontsynonym[MSTimes]          [name:Times New Roman\xetexcolon mapping=tex-text]            [encoding=uc]
\definefontsynonym[MSTimesItalic]    [name:Times New Roman Italic\xetexcolon mapping=tex-text]     [encoding=uc]
\definefontsynonym[MSTimesBold]      [name:Times New Roman Bold\xetexcolon mapping=tex-text]       [encoding=uc]
\definefontsynonym[MSTimesBoldItalic][name:Times New Roman Bold Italic\xetexcolon mapping=tex-text][encoding=uc]

\stoptypescript

\starttypescript[serif][timesnewroman][name]

\definefontsynonym[Serif]           [MSTimes]
\definefontsynonym[SerifBold]       [MSTimesBold]
\definefontsynonym[SerifItalic]     [MSTimesItalic]
\definefontsynonym[SerifBoldItalic] [MSTimesBoldItalic]
\definefontsynonym[SerifSlanted]    [MSTimesItalic]
\definefontsynonym[SerifBoldSlanted][MSTimesBoldItalic]
\definefontsynonym[SerifCaps]       [MSTimes]

\stoptypescript

\starttypescript[sans][arial][uc]

\definefontsynonym[Arial]          [name:Arial\xetexcolon mapping=tex-text]            [encoding=uc]
\definefontsynonym[ArialItalic]    [name:Arial Italic\xetexcolon mapping=tex-text]     [encoding=uc]
\definefontsynonym[ArialBold]      [name:Arial Bold\xetexcolon mapping=tex-text]       [encoding=uc]
\definefontsynonym[ArialBoldItalic][name:Arial Bold Italic\xetexcolon mapping=tex-text][encoding=uc]

\stoptypescript

\starttypescript[sans][arial][name]

\definefontsynonym[Sans]           [Arial]
\definefontsynonym[SansBold]       [ArialBold]
\definefontsynonym[SansItalic]     [ArialItalic]
\definefontsynonym[SansBoldItalic] [ArialBoldItalic]
\definefontsynonym[SansSlanted]    [ArialItalic]
\definefontsynonym[SansBoldSlanted][ArialBoldItalic]
\definefontsynonym[SansCaps]       [Arial]

\stoptypescript

%D MS Office comes with an installation of the Lucida family in
%D TrueType form. It's nice, except\dots\ no math, no slanted, no caps
%D and some other auxiliary fonts.

\starttypescript [serif] [lucida] [uc]

  \definefontsynonym [LucidaBright]              [name:Lucida Bright\xetexcolon mapping=tex-text]         [encoding=uc]
  \definefontsynonym [LucidaBright-Demi]         [name:Lucida Bright Demibold\xetexcolon mapping=tex-text][encoding=uc]
  \definefontsynonym [LucidaBright-DemiItalic]   [name:Lucida Bright Demibold\xetexcolon mapping=tex-text][encoding=uc]
  \definefontsynonym [LucidaBright-Italic]       [name:Lucida Bright\xetexcolon mapping=tex-text]         [encoding=uc]

  \definefontsynonym [LucidaBrightSmallcaps]     [LucidaBright]
  \definefontsynonym [LucidaBrightSmallcaps-Demi][LucidaBright-Demi]
  \definefontsynonym [LucidaBright-Oblique]      [LucidaBright-Italic]

\stoptypescript

\starttypescript [sans] [lucida] [uc]
  \definefontsynonym [LucidaSans]           [name:Lucida Sans Regular\xetexcolon mapping=tex-text]        [encoding=uc]
  \definefontsynonym [LucidaSans-Demi]      [name:Lucida Sans Demibold Roman\xetexcolon mapping=tex-text] [encoding=uc]
  \definefontsynonym [LucidaSans-DemiItalic][name:Lucida Sans Demibold Italic\xetexcolon mapping=tex-text][encoding=uc]
  \definefontsynonym [LucidaSans-Italic]    [name:Lucida Sans Italic\xetexcolon mapping=tex-text]         [encoding=uc]

  \definefontsynonym [LucidaSans-Bold]      [LucidaSans-Demi]
  \definefontsynonym [LucidaSans-BoldItalic][LucidaSans-DemiItalic]

\stoptypescript

\starttypescript [mono] [lucida] [uc]

  \definefontsynonym [LucidaSans-Typewriter]           [name:Lucida Sans Typewriter Regular]     [encoding=uc]
  \definefontsynonym [LucidaSans-TypewriterBold]       [name:Lucida Sans Typewriter Bold]        [encoding=uc]
  \definefontsynonym [LucidaSans-TypewriterBoldOblique][name:Lucida Sans Typewriter Bold Oblique][encoding=uc]
  \definefontsynonym [LucidaSans-TypewriterOblique]    [name:Lucida Sans Typewriter Oblique]     [encoding=uc]

\stoptypescript

\starttypescript [calligraphy] [lucida] [uc]

  \definefontsynonym[LucidaCalligraphy-Italic][name:Lucida Calligraphy Italic\xetexcolon mapping=tex-text][encoding=uc]

\stoptypescript

% No casual that I know of

\starttypescript[handwriting][lucida][uc]

  \definefontsynonym[LucidaHandwriting-Italic][name:Lucida Handwriting Italic\xetexcolon mapping=tex-text][encoding=uc]

\stoptypescript

\starttypescript[fax][lucida][uc]

  \definefontsynonym[LucidaFax]           [name:Lucida Fax Regular\xetexcolon mapping=tex-text]        [encoding=uc]
  \definefontsynonym[LucidaFax-Demi]      [name:Lucida Fax Demibold\xetexcolon mapping=tex-text]       [encoding=uc]
  \definefontsynonym[LucidaFax-DemiItalic][name:Lucida Fax Demibold Italic\xetexcolon mapping=tex-text][encoding=uc]
  \definefontsynonym[LucidaFax-Italic]    [name:Lucida Fax Italic\xetexcolon mapping=tex-text]         [encoding=uc]

\stoptypescript

%D Gentium is from SIL, the fine makers of \XETEX, and it's not only
%D very complete with Roman and Italic Unicode support, but very
%D attractive.

\starttypescript[serif][gentium][uc]

\definefontsynonym[Gentium]      [name:Gentium\xetexcolon mapping=tex-text]       [encoding=uc]
\definefontsynonym[GentiumItalic][name:Gentium Italic\xetexcolon mapping=tex-text][encoding=uc]

\stoptypescript

\starttypescript[serif][gentium][name]

\definefontsynonym[Serif]           [Gentium]
\definefontsynonym[SerifBold]       [Gentium]
\definefontsynonym[SerifItalic]     [GentiumItalic]
\definefontsynonym[SerifBoldItalic] [GentiumItalic]
\definefontsynonym[SerifSlanted]    [GentiumItalic]
\definefontsynonym[SerifBoldSlanted][GentiumItalic]
\definefontsynonym[SerifCaps]       [Gentium]

\stoptypescript

\stoptypescriptcollection

\endinput
