%D \module
%D   [       file=spec-fdf,
%D        version=1998.05.18,
%D          title=\CONTEXT\ \PDF\ Macros,
%D       subtitle=Support Macros,
%D         author=Hans Hagen,
%D           date=\currentdate,
%D      copyright={PRAGMA / Hans Hagen \& Ton Otten}]
%C
%C This module is part of the \CONTEXT\ macro||package and is
%C therefore copyrighted by \PRAGMA. See mreadme.pdf for
%C details.

%D The name of this module is a bit strange but it started with fields
%D so we keep the name.

%D When dealing with resources, we share the resource dictionaries
%D between all xforms. This is inefficent in the sense that when no
%D resources are used, redundant entries take space, but on the other
%D hand we save redundant dictionaries so it's a nice compromise. Maybe
%D that in \LUATEX\ I will reimplement most of the code here anyway.

%D We need to check if we can use \type {\driverreferenced}
%D object in more places.

%D Initialization of fields is tricky. If a field has no
%D value, it is kind of not there. If ResetForm is used, the
%D default is assigned, but pushbuttons are spoiled. Adding a
%D \type {/MK} dictionary helps, but gives ugly down
%D appearances (displaced with background). What a mess.
%D Also, in order to get at least something, the \type {/AS}
%D key should be provided.

% to do : /IF << /SW /N >> == no scaling / clipping of widget

\unprotect

%D \macros
%D   {PDFobjref}
%D
%D Just a shortcut.

% Watch out, \def\PDFobjref#1{\purenumber#1 0 R} also works, but not when
% #1 == \the\whatever

\def\PDFobjref#1{\purenumber{#1} 0 R}

%D \macros
%D   {PDFswapdir}

\let\PDFswapdir\empty \def\PDFswapdir{\ifcase\inlinedirection\or\or-\fi}

% the pdf spec changed cq. viewers started behaving differently / 5+

\chardef\overcomePDFpage\plusone   % page numbers/ beware: optimizers remove this one
\chardef\overcomePDFpage\plustwo   % page:number
%chardef\overcomePDFpage\plusthree % pdftex page ref feature

\ifx\pdfpageref\undefined \else \chardef\overcomePDFpage\plusthree \fi

%D \macros
%D   {setPDFdestination}
%D
%D \PDF\ destinations should obey the specifications laid down
%D in the \PDF\ reference manual. The next macro strips illegal
%D characters from the destination name.
%D
%D The \ACROBAT\ programs are not bug free. By setting the next
%D switches, we will at least try to prevent problems.

\newif\ifovercomePDFbugs  \overcomePDFbugsfalse % dest sort problem / 3-
\newif\ifovercomePDFspace \overcomePDFspacetrue % dest sort problem / 3-

\let\setPDFdestination\gobbleoneargument % a MK specific definition

%D \macros
%D   {sanitizePDFstring}
%D
%D This macro at least tries to convert a arbitrary string
%D into a sequence of characters valid for \PDF\ bookmarks and
%D alike.

\def\sanitizePDFstring#1\to#2{}

%D \macros
%D   {doPDFdestination,
%D    doPDFaction,
%D    doPDFannotation,
%D    doPDFannotationobject,
%D    doPDFdictionaryobject,
%D    doPDFarrayobject,
%D    doPDFaddtocatalog,
%D    doPDFaddtoinfo,
%D    doPDFpageattribute,
%D    doPDFpageresource,
%D    doPDFpagesattribute,
%D    doPDFbookmark,
%D    defaultobjectreference,
%D    doPDFgetobjectreference}
%D
%D This module deals with \PDF\ support, including fill||in
%D forms. Before we present the largely unreadable bunch of
%D macros, we introduce the here||not||defined low level
%D interface macros. These must be provided by the special
%D drivers \type{pdf} (\ACROBAT) and \type{tpd} (\PDFTEX).
%D
%D \starttyping
%D \doPDFdestination            #1         name
%D \doPDFaction                 #1#2#3     width height action
%D \doPDFannotation             #1#2#3     width height data
%D \doPDFannotationobject       #1#2#3#4#5 class name width height data
%D \doPDFdictionaryobject       #1#2#3     class name data
%D \doPDFarrayobject            #1#2#3     class name data
%D \doPDFaddtocatalog           #1
%D \doPDFaddtoinfo              #1
%D \doPDFpageattribute          #1
%D \doPDFpageresource           #1
%D \doPDFpagesattribute         #1
%D \doPDFbookmark               #1#2#3#4#5 level n text page open
%D
%D \defaultobjectreference      #1#2       class name
%D \doPDFgetobjectreference     #1#2#3     class name \PDFobjectreference
%D \doPDFgetobjectpagereference #1#2#3     class name \PDFobjectreference
%D \stoptyping
%D
%D The keywords reflect their use. For the moment we stick to
%D keywords, because that way at we get an indication of what
%D we're doing.

\startspecials[fdf]

%D Common:

% \def\doPDFgetobjectreference#1#2#3%
%  {\def#3{..}}

\def\doPDFgetobjectpage#1#2#3%
  {\dogetobjectreferencepage{#1}{#2}#3%
   \ifx#3\empty\def#3{1}\fi}

\def\doPDFgetobjectpagereference#1#2#3%
  {\dogetobjectreferencepage{#1}{#2}#3%
   \ifx#3\empty
     \doPDFgetpagereference\realfolio#3%
   \else
     \doPDFgetpagereference#3#3% we assume that #3 gets expanded
   \fi}

% \def\doPDFgetpagereference#1#2%%%%%%%%%%%%%%%
%   {\def#2{...}}

%D Due to the fact that \PDFTEX\ has a different concept of
%D page attributes, we need:

\let\doPDFresetpageattributes\relax
\let\doPDFresetpageresources \relax

\appendtoksonce
  \doPDFresetpageattributes
  \doPDFresetpageresources
\to \everyaftershipout

\ifx\PDFcode\undefined
  \ifx\pdfliteral\undefined
    \def\PDFcode#1{\message{[ignored pdfliteral: #1]}}
  \else
    \let\PDFcode\pdfliteral
  \fi
\fi

%D For special (\METAPOST) effects, we need to build
%D resource dictionaries. Here is the framework.

\let\docuPDFextgstates\empty
%let\pagePDFextgstates\empty

\def\checkPDFextgstates
  {\ifx\docuPDFextgstates\empty \else
     \ifnum\realpageno=\lastpage\relax
       \doPDFdictionaryobject{FDF}{docuextgstates}{\docuPDFextgstates}%
     \fi
     \doPDFgetobjectreference{FDF}{docuextgstates}\PDFobjectreference
     \doPDFpageresource{/ExtGState \PDFobjectreference}%
   \fi}

\appendtoksonce
  \checkPDFextgstates
\to \everyshipout

\def\appendtoPDFdocumentextgstates#1%
  {\xdef\docuPDFextgstates{\docuPDFextgstates\space#1}}

% this can save a pass, but we need to handle dpx then first; for
% that we need a 'direct special'
%
% \def\checkPDFextgstates
%   {\ifx\docuPDFextgstates\empty \else
%      \doreservePDFobject{FDF}{docuextgstates}%
%      \gdef\checkPDFextgstates
%        {\doPDFgetobjectreference{FDF}{docuextgstates}\PDFobjectreference
%         \doPDFpageresource{/ExtGState \PDFobjectreference}}%
%      \checkPDFextgstates
%    \fi}
%
% \appendtoksonce
%    \ifx\docuPDFextgstates\empty \else
%      \doPDFreserveddictionaryobject{FDF}{docuextgstates}\docuPDFextgstates
%    \fi
% \to \everybye
%
% \appendtoksonce
%   \checkPDFextgstates
% \to \everyshipout

%D Another special mechanism (needed for color separation):

\let\docuPDFcolorspaces\empty

\def\checkPDFcolorspaces
  {\ifx\docuPDFcolorspaces\empty \else
     \ifnum\realpageno=\lastpage\relax
       \doPDFdictionaryobject{FDF}{colorspaces}{\docuPDFcolorspaces}%
     \fi
     \doPDFgetobjectreference{FDF}{colorspaces}\PDFobjectreference
     \doPDFpageresource{/ColorSpace \PDFobjectreference}%
   \fi}

\appendtoksonce
  \checkPDFcolorspaces
\to \everyshipout

\def\appendtoPDFdocumentcolorspaces#1%
  {\xdef\docuPDFcolorspaces{\docuPDFcolorspaces\space#1}}

%D And another one (used to be in spec-pdf)

\let\docuPDFshades\empty

\def\checkPDFshades
  {\ifx\docuPDFshades\empty \else
     \ifnum\realpageno=\lastpage\relax
       \doPDFdictionaryobject{FDF}{docushades}{\docuPDFshades}%
     \fi
     \doPDFgetobjectreference{FDF}{docushades}\PDFobjectreference
     \doPDFpageresource{/Shading \PDFobjectreference}%
   \fi}

\appendtoksonce
  \checkPDFshades
\to \everyshipout

\def\appendtoPDFdocumentshades#1%
  {\xdef\docuPDFshades{\docuPDFshades\space#1}}

%D \macros
%D   {doPDFsetupscreen,doPDFsetupidentity}
%D
%D Opposite to \DVI\ drivers, \PDF\ ones must know which what
%D page dimensions they are dealing. We also use the
%D opportunity to launch full screen (1) or show bookmarks (2).

\let\currentPDFpagemode   \empty % document catalog
\let\currentPDFviewerprefs\empty % document catalog

\let\currentPDFcropbox    \empty % page attributes
\let\currentPDFbleedbox   \empty % page attributes
\let\currentPDFartbox     \empty % page attributes
\let\currentPDFtrimbox    \empty % page attributes

\def\doPDFsetupscreen#1#2#3#4#5#6% watch the extra argument
  {\bgroup
%   \!!widtha#4%
%   \advance\!!widtha#2%
%   \!!heighta-#5%
%   \!!heightb#1% extra argument
%   \advance\!!heightb -#3%
%   \advance\!!heighta \!!heightb
%   % sometimes whole values give better results
%   % \PointsToWholeBigPoints{#2}\left
%   % \PointsToWholeBigPoints\!!heighta\bottom
%   % \PointsToWholeBigPoints\!!widtha \width
%   % \PointsToWholeBigPoints\!!heightb\height
%   % but since pdf/x does not round when checking if
%   % the boxes fit inside the media box ...
%   \PointsToBigPoints{#2}\left
%   \PointsToBigPoints\!!heighta\bottom
%   \PointsToBigPoints\!!widtha \width
%   \PointsToBigPoints\!!heightb\height
%   \xdef\currentPDFcropboxspec
%     {[\left\space\bottom\space\width\space\height]}%
%   \global\let\currentPDFtrimboxspec\currentPDFcropboxspec
%
%   \xdef\currentPDFpagemode
%     {/PageMode \ifcase#6
%        /UseNone\or/FullScreen\or/UseOutlines\else/UseNone\fi}%
%
   \xdef\currentPDFpagemode
     {\ifnum#6=4
        /PageLayout /TwoColumnRight
      \else
        /PageMode \ifcase#6
        /UseNone\or/FullScreen\or/UseOutlines\else/UseNone\fi
      \fi}%
   \xdef\currentPDFviewerprefs % space after #6 needed, else \relax
     {\ifcase#6 \or\or\else /ViewerPreferences << /FitWindow true >>\fi}%
   \egroup}

% not that good if we switch drivers

\def\addPDFdocumentinfo
  {\doPDFaddtocatalog{\currentPDFpagemode\currentPDFviewerprefs}%
   \doPDFaddtocatalog{/Version \ifdim\PDFversion00\points>100\points 1.\fi\PDFversion}%
   \doPDFaddtoinfo{/Trapped /False}%
   \doPDFaddtoinfo{/ConTeXt.Version (\contextversion)}%
   \doPDFaddtoinfo{/ConTeXt.Time    (\number\normalyear.\twodigits\normalmonth.\twodigits\normalday\space \twodigits\currenthour:\twodigits\currentminute)}%
   \doPDFaddtoinfo{/ConTeXt.Jobname (\jobname)}%
   \doPDFaddtoinfo{/ConTeXt.Url     (www.pragma-ade.com)}}

\appendtoksonce % hack to prevent duplicates
    \addPDFdocumentinfo
\to \everyfirstshipout

\def\PDFversion{1.5}

\appendtoksonce
  \def\PDFversion{1.5}%
  \let\addPDFdocumentinfo\relax
\to \everyresetspecials

\def\doPDFsetupwhateverbox#1#2#3#4#5#6% watch the extra arguments
  {\bgroup
   \!!widtha \dimexpr#5+#3\relax
   \!!heightb\dimexpr#2-#4\relax
   \!!heighta\dimexpr\!!heightb-#6\relax
   % sometimes whole values give better results
   % \PointsToWholeBigPoints{#3}\left
   % \PointsToWholeBigPoints\!!heighta\bottom
   % \PointsToWholeBigPoints\!!widtha \width
   % \PointsToWholeBigPoints\!!heightb\height
   % but since pdf/x does not round when checking if
   % the boxes fit inside the media box ...
   \PointsToBigPoints{#3}\left
   \PointsToBigPoints\!!heighta\bottom
   \PointsToBigPoints\!!widtha \width
   \PointsToBigPoints\!!heightb\height
   \xdef#1{[\left\space\bottom\space\width\space\height]}%
   \egroup}

\def\doPDFsetupartbox  {\doPDFsetupwhateverbox\currentPDFartbox  }
\def\doPDFsetupcropbox {\doPDFsetupwhateverbox\currentPDFcropbox }
\def\doPDFsetupbleedbox{\doPDFsetupwhateverbox\currentPDFbleedbox}
\def\doPDFsetuptrimbox {\doPDFsetupwhateverbox\currentPDFtrimbox }

\gdef\currentPDFtrimbox{\currentPDFcropbox} % default, needed for pdf/x

\def\flushPDFwhateverbox#1#2%
  {\doifsomething{#1}{\doPDFpageattribute{/#2Box #1}}}

\appendtoksonce
  \flushPDFwhateverbox\currentPDFartbox  {Art}%
  \flushPDFwhateverbox\currentPDFcropbox {Crop}%
  \flushPDFwhateverbox\currentPDFbleedbox{Bleed}%
  \flushPDFwhateverbox\currentPDFtrimbox {Trim}%
\to \everyshipout

\def\doPDFsetupidentity#1#2#3#4#5#6%
  {\bgroup
   \enablePDFdocencoding
   \edef\!!stringa{#5}%
   \ifx\!!stringa\empty \ifx\pdfdate\undefined\else
     \edef\!!stringa{D:\pdfdate}%
   \fi \fi
   \expanded{\doPDFaddtoinfo
     {/Title (#1)
      /Subject (#2)
      /Author (#3)
      /Creator (#4)
      /ModDate (\!!stringa)
      /ID (\jobname.\!!stringa) % needed for pdf/x
      /Keywords (#6)}}%
   \egroup}

%D \macros
%D   {doPDFsetupopenaction,doPDFsetupcloseaction,
%D    doPDFsetupopenpageaction,doPDFsetupclosepageaction}
%D
%D Setting the open and close actions is kind of fuzzy
%D because action chains are derived from the reference
%D mechanism.

%D Starting with version~5 viewers, when the open actions
%D started yto give problems, for testing purposes we
%D decided use indirect actions.

% \definespecial\dosetupopenaction     {\doPDFsetupopenaction}
% \definespecial\dosetupcloseaction    {\doPDFsetupcloseaction}
% \definespecial\dosetupopenpageaction {\doPDFsetupopenpageaction}
% \definespecial\dosetupclosepageaction{\doPDFsetupclosepageaction}

\let\lastPDFaction\empty

%D We can safe a couple of references by moving this code
%D to the specific drivers.
%D
%D The following code used to work okay, but as with any
%D update of Acrobat Viewers, upward compatibility was
%D just a dream.

\definespecial\dosetupopenaction {\doPDFaddtocatalog{/OpenAction  <<\lastPDFaction>>}}
\definespecial\dosetupcloseaction{\doPDFaddtocatalog{/CloseAction <<\lastPDFaction>>}}

% todo: /AA << dictionary in catalog >>

% \globalletempty\PDFdocumentclose
% \globalletempty\PDFwillsave
% \globalletempty\PDFdidsave
% \globalletempty\PDFwillprint
% \globalletempty\PDFdidprint

% \definespecial\dosetupdocumentcloseaction {\global\let\PDFdocumentclose\lastPDFaction}
% \definespecial\dosetupwillsaveaction      {\global\let\PDFwillsave     \lastPDFaction}
% \definespecial\dosetupdidsaveaction       {\global\let\PDFdidsave      \lastPDFaction}
% \definespecial\dosetupwillprintaction     {\global\let\PDFwillprint    \lastPDFaction}
% \definespecial\dosetupdidprintaction      {\global\let\PDFdidprint     \lastPDFaction}

% \def\checkPDFdocumentactions
%   {\iflocation
%      \doPDFpageattribute
%        {/AA <<\ifx\PDFdocumentclose\empty \else /DC <<\PDFdocumentclose>> \fi
%               \ifx\PDFwillsave     \empty \else /WS <<\PDFwillsave     >> \fi
%               \ifx\PDFdidsave      \empty \else /DS <<\PDFdidsave      >> \fi
%               \ifx\PDFwillprint    \empty \else /WP <<\PDFwillprint    >> \fi
%               \ifx\PDFdidprint     \empty \else /DP <<\PDFdidprint     >> \fi>>}%
%      % \globalletempty\PDFdocumentclose
%      % \globalletempty\PDFwillsave
%      % \globalletempty\PDFdidsave
%      % \globalletempty\PDFwillprint
%      % \globalletempty\PDFdidprint
%      \global\let\checkPDFdocumentactions\relax
%    \fi}

% \appendtoksonce
%   \checkPDFdocumentactions
% \to \everylastshipout

%\def\doPDFsetupopenaction%
%  {\doPDFdictionaryobject{FDF}{local:openaction}\lastPDFaction
%   \doPDFgetobjectreference{FDF}{local:openaction}\PDFobjectreference
%   \doPDFaddtocatalog{/OpenAction \PDFobjectreference}}
%
%\def\doPDFsetupcloseaction%
%  {\doPDFdictionaryobject{FDF}{local:closeaction}\lastPDFaction
%   \doPDFgetobjectreference{FDF}{local:closeaction}\PDFobjectreference
%   \doPDFaddtocatalog{/CloseAction \PDFobjectreference}}

\let\PDFopenpageaction \empty
\let\PDFclosepageaction\empty

\definespecial\dosetupopenpageaction {\global\let\PDFopenpageaction \lastPDFaction}
\definespecial\dosetupclosepageaction{\global\let\PDFclosepageaction\lastPDFaction}

\def\checkPDFpageactions
  {\iflocation % important since direct -)
     \donefalse
     \ifx\PDFopenpageaction \empty\!!doneafalse\else\donetrue\!!doneatrue\fi
     \ifx\PDFclosepageaction\empty\!!donebfalse\else\donetrue\!!donebtrue\fi
     \ifdone
       \doPDFpageattribute
         {/AA <<\if!!donea/O <<\PDFopenpageaction >> \fi
                \if!!doneb/C <<\PDFclosepageaction>> \fi>>}%
     \fi
     \global\let\PDFopenpageaction \empty
     \global\let\PDFclosepageaction\empty
   \fi}

\appendtoksonce
  \checkPDFpageactions
\to \everyshipout

%D \macros
%D   {doPDFstartthisislocation}
%D
%D Next we define the macros that deal with hyperreferencing,
%D graphic inclusion and general document features. These are
%D the olderst ones. I won't comment much because one needs
%D knowledge of \PDF\ itself, and explaning \PDF\ is beyond
%D this documentation.

\def\doPDFstartthisislocation#1%
  {\bgroup
   \setPDFdestination{#1}%
  %\doifsomething{\PDFdestination}
  %  {\doPDFdestination{\PDFdestination}}%
   \ifx\PDFdestination\empty \else
     \doPDFdestination{\PDFdestination}%
   \fi
   \egroup}

%D \macros
%D   {doPDFstartgotolocation,
%D    doPDFstartgotorealpage,
%D    doPDFstartgotoJS}
%D
%D The goto macros use the switch \type{\ifsecondaryreference}
%D to determine if actions should be linked.

\def\locationfilesuffix{pdf}

% \def\preparePDFlocationfile#1#2%
%   {\setreferencefilename#1\to#2%
%    \expanded{\doifnotinstring{.\locationfilesuffix}{#2}}
%      {\edef#2{#2.\locationfilesuffix}}}
%
% \def\preparePDFlocationfile#1\to#2%
%   {\setreferencefilename#1\to#2%
%    \expanded{\doifnotinstring{.pdf}{#2}}{\edef#2{#2.pdf}}}

\def\doPDFstartgotolocation#1#2#3#4#5#6%
  {\bgroup
   \doifelsenothing{#3}
     {\setPDFdestination{#5}%
      \doifelsenothing\PDFdestination
        {\let\action\empty}
        {\doifelsenothing{#4}
           {\let\PDFfile\empty}
           {\expanded{\beforesplitstring#4}\at.\to\PDFfile
            \doifparentfileelse\PDFfile % {#4}
              {\let\PDFfile\empty}
             %{\setreferencefilename#4.\locationfilesuffix\to\PDFfile
              {\@EA\setreferencefilename\PDFfile.\locationfilesuffix\to\PDFfile
               \edef\PDFfile
                 {R /F (\PDFfile)\ifgotonewwindow\space/NewWindow true \fi}}}%
         \edef\action%
           {/S /GoTo\PDFfile\space /D (\PDFdestination)}}}
     {\doifelsenothing{#4}
        {\let\PDFfile\empty
         \let\PDFdestination\empty}
        {\setreferencefilename/#4\to\PDFfile
         \setPDFdestination{#5}%
         \doifsomething\PDFdestination
           {\edef\PDFdestination{\URLhash\PDFdestination}}}%
      \edef\action{/S /URI /URI (#3\PDFfile\PDFdestination)}}%
   \ifx\action\empty\else
     \ifsecondaryreference
       \savesecondaryPDFreference\action
     \else
       \getsecondaryPDFreferences
       \doPDFaction{\PDFswapdir#1}{#2}{\action \secondaryPDFreferences}%
     \fi
   \fi
   \egroup}

\def\PDFgotonewwindow{\ifgotonewwindow\space/NewWindow true \fi}

% optimization in tpd driver
%
% \edef\PDFdestination{(page:\the\scratchcounter)}%
%
%   ==>
%
% \advance\scratchcounter 1
% \edef\PDFdestination{[\pdfpageref \PDFobjref\scratchcounter\PDFpageviewwrd]}%
%
% \doPDFgetpagedestination#1#2% pagenumber macro % % fuzzy hack

\def\doPDFstartgotorealpage#1#2#3#4#5% watch the R append trick
  {\bgroup
   \doifelsenothing{#3}% #1 = url
     {\scratchcounter0#5\relax
      \ifnum\scratchcounter>0
        \doifelsenothing{#4}
          {\let\PDFfile\empty}
          {\expanded{\beforesplitstring#4}\at.\to\PDFfile
           \doifparentfileelse\PDFfile % {#4}
             {\let\PDFfile\empty}
            %{\setreferencefilename#4.\locationfilesuffix\to\PDFfile
             {\@EA\setreferencefilename\PDFfile.\locationfilesuffix\to\PDFfile
              \edef\PDFfile{R /F (\PDFfile)\PDFgotonewwindow}}}%
        \ifx\PDFfile\empty
          \ifcase\overcomePDFpage
          \or % pdf starts numbering at zero
            \advance\scratchcounter \minusone
            \edef\PDFdestination{[\the\scratchcounter\space\PDFpageviewwrd]}%
          \or % pdf starts numbering at zero
            \advance\scratchcounter \minusone
            \edef\PDFdestination{(page:\the\scratchcounter)}%
          \or % pdftex starts numbering at one
            \edef\PDFdestination{[\pdfpageref\scratchcounter\space0 R \PDFpageviewwrd]}%
          \fi
        \else % across files it's a page number / pdf starts numbering at zero
          \advance\scratchcounter \minusone
          \edef\PDFdestination{[\the\scratchcounter\space\PDFpageviewwrd]}%
        \fi
        \edef\action{/S /GoTo\PDFfile\space /D \PDFdestination}%
      \else
        \let\action\empty
      \fi}
     {\doifelsenothing{#4}
        {\let\PDFfile\empty}
        {\setreferencefilename/#4\to\PDFfile}%
      \edef\action{/S /URI /URI (#3\PDFfile)}}%
   \ifx\action\empty\else
     \ifsecondaryreference
       \savesecondaryPDFreference\action
     \else
       \getsecondaryPDFreferences
       \doPDFaction{\PDFswapdir#1}{#2}{\action \secondaryPDFreferences}%
     \fi
   \fi
   \egroup}

\let\lastfakedPDFpage\!!zerocount

\def\fakePDFpagedestination % as in pdf, we start numbering at zero
  {\iflocation \ifarrangingpages \ifnum\overcomePDFpage=\plustwo \else
     \ifnum\lastfakedPDFpage<\realpageno
       \bgroup
         \xdef\lastfakedPDFpage{\realfolio}%
         \advance\realpageno \minusone % is \expanded needed ?
         \expanded{\doPDFdestination{page:\realfolio}}%
       \egroup
     \fi
   \fi \fi \fi}

\appendtoksonce
  \fakePDFpagedestination
\to \everyshipout

\def\doPDFstartgotoJS#1#2#3%
  {\bgroup
   \doPSsanitizeJScode#3\to\sanitizedJScode
   \edef\action
     {/S /JavaScript /JS (\sanitizedJScode)}%
   \ifsecondaryreference
     \savesecondaryPDFreference\action
   \else
     \getsecondaryPDFreferences
     \doPDFaction{\PDFswapdir#1}{#2}{\action \secondaryPDFreferences}%
   \fi
   \egroup}

%D \macros
%D   {doPDFstartexecutecommand}
%D
%D At the cost of much auxiliary placeholders, we can pretty
%D fast convert the command asked for. This is how the \PDF\
%D code looks like.

\def\PDFmoviecode#1#2#3%
  {/Movie
   /T (\ifcase#1movie \else sound \fi\ifx\argumentA\empty#2\else\argumentA\fi)
   /Operation /\ifcase#3Play\or Stop\or Pause\or Resume\fi\space}

\def\PDFexecutestartmovie  {\PDFmoviecode0\currentmovie0}
\def\PDFexecutestopmovie   {\PDFmoviecode0\currentmovie1}
\def\PDFexecutepausemovie  {\PDFmoviecode0\currentmovie2}
\def\PDFexecuteresumemovie {\PDFmoviecode0\currentmovie3}

\def\PDFexecutestartsound  {\PDFmoviecode1\currentsound0}
\def\PDFexecutestopsound   {\PDFmoviecode1\currentsound1}
\def\PDFexecutepausesound  {\PDFmoviecode1\currentsound2}
\def\PDFexecuteresumesound {\PDFmoviecode1\currentsound3}

\def\PDFformcode#1%
  {\doFDFiffieldset{#1}{/Field [\doFDFgetfieldset{#1}]}}

% bit 3 = html
% bit 6 = xml
% bit 4 = get

\ifx\PDFsubmitfiller\undefined \let\PDFsubmitfiller\empty \fi

\chardef\PDFformmethod=1 % 0=GET 1=POST

\def\PDFformflag#1#2{\ifcase\PDFformmethod#1\else#2\fi}

\def\PDFexecuteimportform  {/Named /N /AcroForm:ImportFDF}
\def\PDFexecuteexportform  {/Named /N /AcroForm:ExportFDF}
\def\PDFexecuteresetform   {/ResetForm  \PDFformcode\argumentA}
\def\PDFexecutesubmitform  {/SubmitForm \PDFformcode\argumentB
                            /Flags \ifcase\submitoutputformat\space
                                         \PDFformflag{12} {4} % 0=unknown
                                   \or   \PDFformflag{12} {4} % 1=HTML
                                   \or   \PDFformflag {8} {0} % 2=FDF
                                   \or   \PDFformflag{40}{32} % 3=XML
                                   \else \PDFformflag{12} {4} % ?=unknown
                                   \fi
                            /F (\argumentA)\PDFsubmitfiller}

% urifill permits url substitution

\def\PDFexecutehide        {/Hide /T (\argumentA) /H true}
\def\PDFexecuteshow        {/Hide /T (\argumentA) /H false}

\def\PDFexecutefirst       {/Named /N /FirstPage}
\def\PDFexecuteprevious    {/Named /N /PrevPage}
\def\PDFexecutenext        {/Named /N /NextPage}
\def\PDFexecutelast        {/Named /N /LastPage}
\def\PDFexecutebackward    {/Named /N /GoBack}
\def\PDFexecuteforward     {/Named /N /GoForward}
\def\PDFexecuteprint       {/Named /N /Print}
\def\PDFexecuteexit        {/Named /N /Quit}
\def\PDFexecuteclose       {/Named /N /Close}
\def\PDFexecutesave        {/Named /N /Save}
\def\PDFexecutesavenamed   {/Named /N /SaveAs}
\def\PDFexecuteopennamed   {/Named /N /Open}
\def\PDFexecutehelp        {/Named /N /HelpUserGuide}
\def\PDFexecutetoggle      {/Named /N /FullScreen}
\def\PDFexecutesearch      {/Named /N /Find}
\def\PDFexecutesearchagain {/Named /N /FindAgain}
\def\PDFexecutegotopage    {/Named /N /GoToPage}
\def\PDFexecutequery       {/Named /N /AcroSrch:Query}
\def\PDFexecutequeryagain  {/Named /N /AcroSrch:NextHit}
\def\PDFexecutefitwidth    {/Named /N /FitWidth}
\def\PDFexecutefitheight   {/Named /N /FitHeight}

\let\PDFobjectclass\empty
\let\PDFobjectname \empty

\def\doPDFstartexecutecommand#1#2#3#4%
  {\doifdefined{PDFexecute#3}
     {\bgroup
      \@EA\dogetcommalistelement\@EA1\@EA\from#4\to\argumentA
      \@EA\dogetcommalistelement\@EA2\@EA\from#4\to\argumentB
      \edef\argument{#4}%
      \edef\action%
        {/S \getvalue{PDFexecute#3}}%
      \ifsecondaryreference
        \savesecondaryPDFreference\action
      \else
        \getsecondaryPDFreferences
%        \ifx\PDFobjectclass\empty
%          \let\next\doPDFaction
%        \else
%          \edef\next{\doPDFactionobject{\PDFobjectclass}{\PDFobjectname}}%
%          \globalletempty\PDFobjectclass
%          \globalletempty\PDFobjectname
%        \fi
%        \next
        \doPDFaction{\PDFswapdir#1}{#2}{\action \secondaryPDFreferences}%
      \fi
      \egroup}}

%D \macros
%D   {doPDFstartrunprogram}
%D
%D Running programs is possible, but is non that portable, and
%D therefore dangerous.

\def\doPDFstartrunprogram#1#2#3#4% new: #3 => #3#4
  {\bgroup
  %\edef\string{#3}%
  %\@EA\beforesplitstring\string\at{ }\to\program
  %\@EA\aftersplitstring \string\at{ }\to\parameters
  %\edef\action%
  %  {/S /Launch /F (\program) /P (\parameters) /D (.)}%
   \edef\action
     {/S /Launch /F (#3) /P (#4) /D (.)}%
   \ifsecondaryreference
     \savesecondaryPDFreference\action
   \else
     \getsecondaryPDFreferences
     \doPDFaction{\PDFswapdir#1}{#2}{\action \secondaryPDFreferences}%
   \fi
   \egroup}

%D \macros
%D   {doPDFstartgotoprofile}
%D
%D Far from perfect, but nevertheless present, is the profile
%D handler. We want to misuse article threads for reder
%D profiles.

\def\doPDFstartgotoprofile#1#2#3% to be done: file
  {\bgroup
   \setPDFdestination{#3}%
   \doifsomething\PDFdestination
     {\edef\action
        {/S /Thread /D (\PDFdestination)}%
      \ifsecondaryreference
        \savesecondaryPDFreference\action
      \else
        \getsecondaryPDFreferences
        \doPDFaction{\PDFswapdir#1}{#2}{\action \secondaryPDFreferences}%
      \fi}%
   \egroup}

%D \macros
%D   {doPDFsetpagetransition}
%D
%D This array holds a reasonable selection of transitions
%D (watch out: \type{replace} is not in this list). Most of
%D the transitions look awful anyway. By the way, \CONTEXT\ is
%D able to select transitions randomly.

% some day, when 1.5 is on linux and apple, we will add:
%
% \def\pagetransitions
%   {{split,in,vertical},{split,in,horizontal},
%    {split,out,vertical},{split,out,horizontal},
%    {blinds,horizontal},{blinds,vertical},
%    {box,in},{box,out},
%    {wipe,east},{wipe,west},{wipe,north},{wipe,south},
%    dissolve,
%    {glitter,east},{glitter,south},
%    {fly,in,east},{fly,in,west},{fly,in,north},{fly,in,south},
%    {fly,out,east},{fly,out,west},{fly,out,north},{fly,out,south},
%    {push,east},{push,west},{push,north},{push,south},
%    {cover,east},{cover,west},{cover,north},{cover,south},
%    {uncover,east},{uncover,west},{uncover,north},{uncover,south},
%    fade}

\def\pagetransitions
  {{split,in,vertical},{split,in,horizontal},
   {split,out,vertical},{split,out,horizontal},
   {blinds,horizontal},{blinds,vertical},
   {box,in},{box,out},
   {wipe,east},{wipe,west},{wipe,north},{wipe,south},
   dissolve,
   {glitter,east},{glitter,south}}

%D Again, we use macros as placeholders for \PDF\ key||value
%D pairs.

\def\PDFpagesplit    {/S /Split    }
\def\PDFpageblinds   {/S /Blinds   }
\def\PDFpagebox      {/S /Box      }
\def\PDFpagewipe     {/S /Wipe     }
\def\PDFpagedissolve {/S /Dissolve }
\def\PDFpageglitter  {/S /Glitter  }
\def\PDFpagereplace  {/S /R        }

\def\PDFpagefly      {/S /Fly      } % 1.5
\def\PDFpagepush     {/S /Push     } % 1.5
\def\PDFpagecover    {/S /Cover    } % 1.5
\def\PDFpageuncover  {/S /Uncover  } % 1.5
\def\PDFpagefade     {/S /Fade     } % 1.5

\def\PDFpagehorizontal {/Dm /H  }
\def\PDFpagevertical   {/Dm /V  }
\def\PDFpagein         {/M  /I  }
\def\PDFpageout        {/M  /O  }
\def\PDFpageeast       {/Di   0 }
\def\PDFpagenorth      {/Di  90 }
\def\PDFpagewest       {/Di 180 }
\def\PDFpagesouth      {/Di 270 }

\def\dodoPDFsetpagetransition#1%
  {\doifdefined{PDFpage#1}
     {\edef\PDFpagetransitions{\PDFpagetransitions\getvalue{PDFpage#1}}}}

\def\doPDFsetpagetransition#1#2%
  {\let\PDFpagetransitions\empty
   \processcommalist[#1]\dodoPDFsetpagetransition
   \doPDFpageattribute
    %{\ifnum#2>0 /Dur #2 \fi
     {\ifnum0<0#2 /Dur #2 \fi
      \ifx\PDFpagetransitions\empty\else/Trans <<\PDFpagetransitions>>\fi}}
%      \ifx\PDFpagetransitions\empty\else/Trans <</Type /Trans \PDFpagetransitions>>\fi}}

%D \macros
%D   {doPDFinsertmov}
%D
%D Most of the annotations we use here are of type {\em
%D link}, but here is another one: the {\em movie} annotation.
%D The driver module must implement \type {setcurrentmovie}.

%D Great: this will become an obsolete pdf feature; why did we have to
%D keep up with the bugs ... and by the time acrobat gets better in
%D handling it have to drop it.

\let\currentmovie\s!unknown

\def\doPDFinsertmov
  {\bgroup
   \xdef\currentmovie{\@@DriverImageLabel}%
   \PointsToBigPoints\@@DriverImageWidth \width
   \PointsToBigPoints\@@DriverImageHeight\height
   \let\pdf@@options\empty
   \let\pdf@@actions\empty
   \donefalse
   \expanded{\processallactionsinset[\@@DriverImageOptions]}
     [\v!controls=>\donetrue,
        \v!repeat=>\edef\pdf@@actions{\pdf@@actions /Mode /Repeat },
       \v!preview=>\edef\pdf@@options{\pdf@@options /Poster true  }]%
   \edef\pdf@@actions{\pdf@@actions /ShowControls \ifdone true\else false\fi}%
   \doPDFannotation\@@DriverImageWidth\@@DriverImageHeight
     {/Subtype /Movie
      /Border [0 0 0]
      /T (movie \currentmovie)
      /Movie << /F (\@@DriverImageFile) /Aspect [\width\space\height] \pdf@@options >>
      /A << \pdf@@actions >>}%
   \egroup}

%D \macros
%D   {doPDFinsertsoundtrack}
%D
%D In \PDF\ sounds can be embedded like movies.

\ifx\everygoto\undefined \newtoks\everygoto \fi

\let\currentsound\s!unknown

\def\doPDFinsertsoundtrack#1#2#3%
  {\bgroup
   \xdef\currentsound{#2}%
   \let\pdf@@actions\empty
   \@EA\processallactionsinset\@EA
     [#3]
     [\v!repeat=>\edef\pdf@@actions{\pdf@@actions /Mode /Repeat }]%
\collectdriverresource
%    \flushatshipout % since it can be buried in a chained box
     {\doPDFannotation{0pt}{0pt}
        {/Subtype /Movie
         /Border [0 0 0]
         /T (sound \currentsound)
         /Movie <</F (#1)>>%
         \ifx\pdf@@actions\empty\else/A << \pdf@@actions >>\fi}}%
   \egroup}

%D \macros
%D   {doPDFattachfile}

\def\doPDFfilestreamobject#1#2#3#4%
  {}

\def\doPDFfilestreamidentifier#1%
  {0}

\def\doPDFgetfilestreamreference#1#2%
  {0 0 R}

\def\doPDFattachfile#1#2#3#4#5#6#7#8%
  {\bgroup % title width height color symbol file
   \edef\PDFfile{#8}\convertcommand\PDFfile\to\PDFfile
   % beware: the symbol may (indirectly) use the file
   % reference when typesetting the object number;
   \presetPDFsymbolappearance{#5}{#6}{#2}{#3}{#4}% sets width/height
   \startPDFsymbolappearance
     \doPDFembedfile\PDFfile{#7}{#8}%
     \doPDFgetembeddedfilereference\PDFfile\PDFobjectreference
     \setFDFlayer\@@DriverAttachmentLayer
     \doPDFannotation{\width}{\totalheight}
       {/Subtype /FileAttachment
        /FS \PDFobjectreference\space
        /Contents (#1)
        \PDFsymbol
        \FDFlayer
        \PDFattributes}%
   \stopPDFsymbolappearance
   \egroup}

% semi-public

\def\doPDFembedfile#1#2#3% symbolic name | filename | user name
  {\edef\PDFfile{#1}\convertcommand\PDFfile\to\PDFfile
   \doifnotflagged{a:\PDFfile}%
     {\doPDFfilestreamobject{PDFEF}{\PDFfile}{#2}{#3}%
      \doglobal\setflag{a:\PDFfile}}}

\def\doPDFgetembeddedfilereference#1#2%
  {\edef\PDFfile{#1}\convertcommand\PDFfile\to\PDFfile
   \doPDFgetobjectreference{PDFEF}\PDFfile#2}

\def\doPDFgetembeddedfilestreamreference#1#2%
  {\edef\PDFfile{#1}\convertcommand\PDFfile\to\PDFfile
   \doPDFgetfilestreamreference\PDFfile#2} % == \doPDFgetobjectreference{PDFFS}\PDFfile#2

\definespecial \doattachfile {\doPDFattachfile}

% requested by Jens-Uwe Morawski: permits usage of pdftosrc
% in viewers that don't support attachments:
%
% \definesymbol
%   [ObjectNumber]
% % [object number {\PDFattachmentnumber[xx]}] % named
%   [object number \PDFattachmentnumber]       % current
%
% \useattachment[test][xx][test.tex]
% \setupattachments[symbol=ObjectNumber]
% \attachment[test]

\def\PDFattachmentnumber
  {\dosingleargument\doPDFattachmentnumber}

\def\doPDFattachmentnumber[#1]%
  {\iffirstargument
     \doPDFfilestreamidentifier{#1}%
   \else
     \doPDFfilestreamidentifier\PDFfile
   \fi}

%D \macros
%D   {...}
%D
%D Rather preliminary. We have to wait till the complete specs
%D show up. As usual, we cannot really check it (Acrobat 6.0
%D has a bug that inhibits us to make a test file). Half a day
%D of testing made clear that trying to control the plugin fails
%D in most cases (we need plugin specs -). We also miss a feature
%D to let acrobat wait with proceeding (action processing) till
%D the media clip is ready.

% aiff audio/aiff
% au   audio/basic
% avi  video/avi
% mid  audio/midi
% mov  video/quicktime
% mp3  audio/x-mp3 (mpeg)
% mp4  audio/mp4
% mp4  video/mp4
% mpeg video/mpeg
% smil application/smil
% swf  application/x-shockwave-flash

% beware, this is preliminary code, should be improved

\def\PDFrenderingspecs#1{\executeifdefined{PDFMR:#1}\empty}

\def\PDFexecutestartrendering  {/Rendition /OP 0 \PDFrenderingspecs\argumentA}
\def\PDFexecutestoprendering   {/Rendition /OP 1 \PDFrenderingspecs\argumentA}
\def\PDFexecutepauserendering  {/Rendition /OP 2 \PDFrenderingspecs\argumentA}
\def\PDFexecuteresumerendering {/Rendition /OP 3 \PDFrenderingspecs\argumentA}

% todo : sub files
%
% \doPDFembedfile{pier-39.png}{pier-39.png}{pier-39.png}%
% \doPDFgetembeddedfilestreamreference{pier-39.png}\xPDFobjectreference
% \edef\xxxx{/RF [(pier-39.png) \xPDFobjectreference]}%

% todo: alternative renderings
%
% object_1  -> <</Type /Rendition /S /MR /C << /Type /MediaClip ... >> >>
% object_2  -> <</Type /Rendition /S /MR /C << /Type /MediaClip ... >> >>
%
% rendering -> <</Type /Rendition /S /MS [objref_1 objref_2]>>

% todo: embedded files (too buggy)
%
%     \let\PDFattribute\empty
%      % /D \PDFobjectreference
%      % test one, no error, but ignored
%      \doifinset\v!file{#4}
%        {\doPDFembedfile{#3}{#3}{#3}%
%         \doPDFgetembeddedfilestreamreference{#3}\PDFobjectreference
%         \edef\PDFattribute{/EF \PDFobjectreference}}%
%      % official, does not work either
%      \doifinset\v!file{#4}
%        {\doPDFembedfile{#3}{#3}{#3}%
%         \doPDFgetembeddedfilereference{#3}\PDFobjectreference}
%      % do we play the game as follows

\definespecial\doinsertrendering#1#2#3#4% tag mime file options
  {\ifundefined{PDFMR:#1}%
     \doifinstringelse{://}{#3}\donetrue\donefalse % evt url as keyword
     \doPDFdictionaryobject{PDFMF}{#1}
       {/Type /Rendition
        /S    /MR
        % does not work: /SP << /Type /MediaScreenParam /BE << /B [1 0 0] /O 0.5 >> >>
        /C << /Type /MediaClip
              /S /MCD
              /N (#1)
              /Alt [() (file not found)] % language id + message
              /D << /Type /Filespec
                    /F (#3)
                    \ifdone/FS /URL\fi >>
              /CT (#2) >>}%
     % common code
     \doifobjectreferencefoundelse{PDFMS}{#1}
       {\doPDFgetobjectreference{PDFMS}{#1}\PDFobjectreferenceB}
       {\doPDFgetobjectreference{PDFMU}{#1}\PDFobjectreferenceB}%
     \doPDFgetobjectreference{PDFMF}{#1}\PDFobjectreferenceA
     \setxvalue{PDFMR:#1}% needed /AA actions in /Screen
       {/R  \PDFobjectreferenceA
        /AN \PDFobjectreferenceB}%
     \doifobjectreferencefoundelse{PDFMS}{#1}\donothing
       {\dodoinsertrenderingwindow{PDFMU}{#1}\zeropoint\zeropoint{#4}}%
   \fi}

\definespecial\doinsertrenderingobject#1#2#3#4% tag class objectname options
  {\ifundefined{PDFMR:#1}%
     \doPDFgetobjectreference{#2}{#3}\PDFobjectreference
     \doPDFdictionaryobject{PDFMF}{#1}
       {/Type /Rendition
        /S    /MR
        /C << /Type /MediaClip
              /S /MCD
              /N (#1)
              /D \PDFobjectreference>>}%
     % common code
     \doifobjectreferencefoundelse{PDFMS}{#1}
       {\doPDFgetobjectreference{PDFMS}{#1}\PDFobjectreferenceB}
       {\doPDFgetobjectreference{PDFMU}{#1}\PDFobjectreferenceB}%
     \doPDFgetobjectreference{PDFMF}{#1}\PDFobjectreferenceA
     \setxvalue{PDFMR:#1}% needed /AA actions in /Screen
       {/R  \PDFobjectreferenceA
        /AN \PDFobjectreferenceB}%
     \doifobjectreferencefoundelse{PDFMS}{#1}\donothing
       {\dodoinsertrenderingwindow{PDFMU}{#1}\zeropoint\zeropoint{#4}}%
   \fi}

\definespecial\doinsertrenderingwindow
  {\dodoinsertrenderingwindow{PDFMS}}

\def\dodoinsertrenderingwindow#1#2#3#4#5%
  {\vbox to #4 \bgroup
     \checkPDFscreenactions{#2}{#5}%
     \doPDFgetobjectpagereference{PDFMF}{#2}\PDFobjectreferenceA
     \doPDFgetobjectreference    {PDFMF}{#2}\PDFobjectreferenceB
     \vss
     \hbox to #3 \bgroup
       \doPDFannotationobject{#1}{#2}{#3}{#4}
         {/Subtype /Screen
          /P \PDFobjectreferenceA
          /A \PDFobjectreferenceB
          \PDFattributes
          /Border [0 0 0]}%
       \hss
     \egroup
   \egroup}

\global\let\PDFrenderingopenpageaction \empty
\global\let\PDFrenderingclosepageaction\empty

\def\checkPDFscreenactions#1#2%
  {\let\PDFattributes\empty
   \iflocation % important since direct -)
     % the action can either (already) be set by the window handler
     % or (normally when no window [i.e a zero dimensions one] is present) by keyword
     \doifinset\v!auto{#2}
       {% brrr, here instead of in navigation module, must move and become special
        % now two sided dependency
        \let\checkrendering\gobbleoneargument
        \ifx\PDFrenderingopenpageaction \empty
          \handlereferenceactions{\v!StartRendering{#1}}\dosetuprenderingopenpageaction
        \fi
        \ifx\PDFrenderingclosepageaction\empty
          \handlereferenceactions{\v!StopRendering {#1}}\dosetuprenderingclosepageaction
        \fi
        }%
     \donefalse
     \ifx\PDFrenderingopenpageaction \empty\!!doneafalse\else\donetrue\!!doneatrue\fi
     \ifx\PDFrenderingclosepageaction\empty\!!donebfalse\else\donetrue\!!donebtrue\fi
     \ifdone
       \edef\PDFattributes
         {/AA <<\if!!donea/PO <<\PDFrenderingopenpageaction >> \fi
                \if!!doneb/PC <<\PDFrenderingclosepageaction>> \fi>>}%
     \fi
     \global\let\PDFrenderingopenpageaction \empty
     \global\let\PDFrenderingclosepageaction\empty
   \fi}

\definespecial\dosetuprenderingopenpageaction {\global\let\PDFrenderingopenpageaction \lastPDFaction}
\definespecial\dosetuprenderingclosepageaction{\global\let\PDFrenderingclosepageaction\lastPDFaction}

%D \macros
%D   {doPDFinsertbookmark}
%D
%D Well, here is the dreadfull bookmark, rather useless because
%D only standard encoding is possible, no typography is done,
%D and a maximum of 32~characters is advized.

\def\doPDFinsertbookmark#1#2#3#4#5% level sublevels text page open=1
  {\bgroup
   % todo, unicode
   \sanitizePDFdocencoding#3\to\bookmarktext % uses scratchcounter
   \stripstring\bookmarktext
  %\sanitizePDFstring#3\to\bookmarktext
   \doPDFbookmark{#1}{#2}{\bookmarktext}{#4}{#5}%
   \egroup}

%D The next section of this module is dedicated to form
%D support. These macros are complicated by the fact that
%D cloning is possible.

%D \macros
%D   {FDFflag...,FDFplus...}
%D
%D The \type{/FT} key determines the type of field: text,
%D button or choice. The latter two come in several disguises,
%D which are set by flipping bits in the \type{/Ff}. Other bits
%D are used to set states. Personally I hate this bitty way of
%D doing things. The next six bit determine the field sub type:

\def\FDFflagMultiLine          {4096} % 13
\def\FDFflagNoToggleToOff     {16384} % 15
\def\FDFflagRadio             {32768} % 16
\def\FDFflagPushButton        {65536} % 17
\def\FDFflagPopUp            {131072} % 18
\def\FDFflagEdit             {262144} % 19

% bugged anyway, so we need to drop it:

\def\FDFflagRadiosInUnison {33554432} % 26

%D A few more (pdf 1.4) flags, what the spell check one: for
%D obscure reasons for Adobe downward compatibility means
%D enabling features that harm old applications like testing.

\def\FDFflagDoNotSpellCheck {4194304} % 23
\def\FDFflagDoNotScroll     {8388608} % 24

%D The next bits (watch how strange the bits are organized)
%D take care of the states:

\def\FDFflagReadOnly              {1} %  1
\def\FDFflagRequired              {2} %  2
\def\FDFflagNoExport              {4} %  3
\def\FDFflagPassword           {8192} % 14
\def\FDFflagSort             {524288} % 20
\def\FDFflagFileSelect      {1048576} % 21

%D There is a second, again bitset oriented, \type{/F} flag:

\def\FDFplusInvisible             {1} %  1
\def\FDFplusHidden                {2} %  2
\def\FDFplusPrintable             {4} %  3

%def\FDFplusNoView               {32} %  6
%def\FDFplusToggleNoView        {256} %  9

\def\FDFplusAutoView            {256} % {288} %  6+9

%D \macros
%D   {setFDFswitches}
%D
%D The non||type bits are mapped onto user||interface
%D swithes, to be used later on:

\def\@@FDFflag{FDFflag}
\def\@@FDFplus{FDFplus}

\letvalue    {\@@FDFflag\v!readonly}=\FDFflagReadOnly
\letvalue    {\@@FDFflag\v!required}=\FDFflagRequired
\letvalue   {\@@FDFflag\v!protected}=\FDFflagPassword
\letvalue      {\@@FDFflag\v!sorted}=\FDFflagSort
\letvalue {\@@FDFflag\v!unavailable}=\FDFflagNoExport
\letvalue     {\@@FDFflag\v!nocheck}=\FDFflagDoNotSpellCheck
\letvalue       {\@@FDFflag\v!fixed}=\FDFflagDoNotScroll
\letvalue        {\@@FDFflag\v!file}=\FDFflagFileSelect

\letvalue      {\@@FDFplus\v!hidden}=\FDFplusHidden
\letvalue   {\@@FDFplus\v!printable}=\FDFplusPrintable

\letvalue        {\@@FDFplus\v!auto}=\FDFplusAutoView

%D A set of switches is collected into the flags we mentioned
%D before by the next macro (we don't handle negations yet,
%D but do take care of redundancy):

\def\FDFflag{0}
\def\FDFplus{0}

\def\setFDFswitches[#1]%
  {\bgroup
   \!!counta\zerocount
   \!!countb\zerocount
   \def\docommand##1%
     {\doifsomething{##1}
        {\advance\!!counta 0\getvalue{\@@FDFflag##1}%
         \setvalue{\@@FDFflag##1}{0}%
         \advance\!!countb 0\getvalue{\@@FDFplus##1}%
         \setvalue{\@@FDFplus##1}{0}}}%
   \processcommacommand[#1]\docommand
   \xdef\FDFflag{\the\!!counta}%
   \xdef\FDFplus{\the\!!countb}%
   \egroup}

%D \macros
%D   {setFDFvalues}
%D
%D Menu items are passed as an array of \type{(string)}'s and
%D the content of this array is build with:

\let\FDFvalues      \empty
\let\FDFfirstvalues \empty
\let\FDFsecondvalues\empty
\let\FDFkidlist     \empty
\let\FDFdefaultindex\!!zerocount
\let\FDFdefaultvalue\empty

% Why do we need to tweak this mechanism each time acrobat updates ...
% it would make sense to have version specific sections in pdf files
% since my guess is that it never will be done right since each year
% new programmers have new ideas about what is supposed to happen with
% kids. So .. best is not to trust this feature esp not for radio
% widgets. (new flags, different interpretation of AS etc etc)

\def\setFDFvalues[#1][#2]% #1 = list (item=>value) #2 = default
  {\let\FDFvalues      \empty
   %when radio opt works ok
   %\let\FDFfirstvalues \empty
   %\let\FDFsecondvalues\empty
   \let\FDFkidlist     \empty
   %\let\FDFdefaultindex\!!zerocount
   %\let\FDFdefaultvalue\empty
   %\scratchcounter\zerocount
   \def\dodocommand##1=>##2=>##3\end
     {\addtocommalist{##1}\FDFkidlist
      %\edef\FDFfirstvalues{\FDFfirstvalues(##1)}%
      %\doif{##1}{#2}{\edef\FDFdefaultindex{\the\scratchcounter}}%
      %\advance\scratchcounter\plusone
      \doifelsenothing{##2}
        {\doif{##1}{#2}{\edef\FDFdefaultvalue{##1}}%
         %\edef\FDFsecondvalues{\FDFsecondvalues(##1)}%
         \edef\FDFvalues{\FDFvalues [(##1)(##1)] }}
        {\doif{##1}{#2}{\edef\FDFdefaultvalue{##2}}%
         %\edef\FDFsecondvalues{\FDFsecondvalues(##2)}%
         \edef\FDFvalues{\FDFvalues [(##2)(##1)] }}}% ! ##1 is shown
   \def\docommand##1%
     {\dodocommand##1=>=>\end}%
   \expanded{\processcommalist[#1]}\docommand}

%D This macro accepts comma separated \type{visual=>result}
%D pairs.

%D \macros
%D   {setFDFalignment}
%D
%D Text and line fields can be entered and showed in three
%D alternative alingments, indicated by a digit:

\def\FDFalign{0}

\def\setFDFalignment[#1]%
  {\processaction
     [#1]
     [  \v!left=>\edef\FDFalign{2},    % raggedleft
      \v!middle=>\edef\FDFalign{1},    % raggedcenter
       \v!right=>\edef\FDFalign{0}]}   % raggedright

%D \macros
%D   {setFDFattributes}
%D
%D The weak part of (at least version 2.1 \PDF) is that only
%D default fonts are handled well. Another restriction is that
%D the encoding vector must be the standard \PDF\ document one.
%D Although the \PDF\ reference explictly states that one could
%D use the normal text operators, leading is not yet handled.
%D
%D For the moment the current \CONTEXT\ font is mapped onto
%D one best suitable default font. The color attribute is
%D less problematic and is directly derived from the \CONTEXT\
%D color.

\def\FDFattributes{/Helv 12 Tf 0 g 14.4 TL}

\def\FDFrm  {TiRo} \def\FDFss  {Helv} \def\FDFtt  {Cour}
\def\FDFrmtf{TiRo} \def\FDFsstf{Helv} \def\FDFtttf{Cour}
\def\FDFrmbf{TiBo} \def\FDFssbf{HeBo} \def\FDFttbf{CoBo}
\def\FDFrmit{TiIt} \def\FDFssit{HeOb} \def\FDFttit{CoOb}
\def\FDFrmsl{TiIt} \def\FDFsssl{HeOb} \def\FDFttsl{CoOb}
\def\FDFrmbi{TiBI} \def\FDFssbi{HeBO} \def\FDFttbi{CoBO}
\def\FDFrmbs{TiBI} \def\FDFssbs{HeBO} \def\FDFttbs{CoBO}

\let\FDFusedfonts=\FDFsstf

\def\setFDFattributes[#1,#2,#3,#4]% style, color, backgroundcolor, framecolor
  {\bgroup % nog interlinie: n TL
   \setbox\scratchbox\hbox
     \bgroup
       \doconvertfont{#1}{}%
       \PointsToBigPoints\bodyfontsize\size % x/xx, so better the actual size
       \doifdefinedelse{FDF\fontstyle\fontalternative}
         {\xdef\FDFattributes{\getvalue{FDF\fontstyle\fontalternative}}}
         {\doifdefinedelse{FDF\fontstyle}
            {\xdef\FDFattributes{\getvalue{FDF\fontstyle}}}
            {\xdef\FDFattributes{\FDFrm}}}%
       \doglobal\addtocommalist\FDFattributes\FDFusedfonts
       \xdef\FDFattributes% move up with "x.y Ts"
         {/\FDFattributes\space\size\space Tf\space\PDFcolor{#2}}%
       \doifelsenothing{#3}
         {\global\let\FDFsurroundings\empty}
         {\xdef\FDFsurroundings{/BG \FDFcolor{#3}}}%
       \doifsomething{#4}
         {\xdef\FDFsurroundings{\FDFsurroundings\space /BC \FDFcolor{#4}}}%
       \ifx\FDFsurroundings\empty \else
         \xdef\FDFsurroundings{/MK << \FDFsurroundings\space>>}%
       \fi
     \egroup
   \egroup}

%D \macros
%D   {setFDFactions}
%D
%D Depending on the type of the field, one can assign
%D \JAVASCRIPT\ code to a mouse event or keystroke. The next
%D preparation macro shows what events are handled.

\let\FDFactions\empty

%\def\setFDFactions[#1,#2,#3,#4,#5,#6,#7,#8]%
%  {\global\let\FDFactions\empty
%   \setFDFaction D#1%  mousedown
%   \setFDFaction U#2%  mouseup
%   \setFDFaction E#3%  enterregion
%   \setFDFaction X#4%  exitregion
%   \setFDFaction K#5%  afterkeystroke
%   \setFDFaction F#6%  formatresult
%   \setFDFaction V#7%  validateresult
%   \setFDFaction C#8%  calculatewhatever
%   \ifx\FDFactions\empty\else
%     \xdef\FDFactions{/AA << \FDFactions >>}% since 1.3 no longer inherited
%   \fi}
%% {\setFDFaction A#2}% mouseup % changed for version 5

% \def\setFDFactions[#1,#2,#3,#4,#5,#6,#7,#8,%
%   {\global\let\FDFactions\empty
%    \setFDFaction D#1%  mousedown
%   %\setFDFaction U#2%  mouseup
%    \setFDFaction E#3%  enterregion
%    \setFDFaction X#4%  exitregion
%    \setFDFaction K#5%  afterkeystroke
%    \setFDFaction F#6%  formatresult
%    \setFDFaction V#7%  validateresult
%    \setFDFaction C#8%  calculatewhatever
%    \setFDFactionsmore#2,}

% \def\setFDFactionsmore#1,#2,#3]%
%   {\setFDFaction{Fo}#2%  focusin
%    \setFDFaction{Bl}#3%  focusout % was I (now pdf ref manual explicitly talks about lowercase l)
%    \ifx\FDFactions\empty\else
%      \xdef\FDFactions{/AA << \FDFactions >>}% since 1.3 no longer inherited
%    \fi
%    \setFDFaction A#1}% mouseup

\def\setFDFactions[#1,#2,#3,#4,#5,#6,#7,#8,%
  {\global\let\FDFactions\empty
   \setFDFaction D#1%  mousedown
   \setFDFaction U#2%  mouseup
   \setFDFaction E#3%  enterregion
   \setFDFaction X#4%  exitregion
   \setFDFaction K#5%  afterkeystroke
   \setFDFaction F#6%  formatresult
   \setFDFaction V#7%  validateresult
   \setFDFaction C#8%  calculatewhatever
   \setFDFactionsmore}

\def\setFDFactionsmore#1,#2]%
  {\setFDFaction{Fo}#1%  focusin
   \setFDFaction{Bl}#2%  focusout % was I (now pdf ref manual explicitly talks about lowercase l)
   \ifx\FDFactions\empty\else
     \xdef\FDFactions{/AA << \FDFactions >>}% since 1.3 no longer inherited
   \fi}

% todo, when new var scheme is implemented
%
%   \setFDFaction{PO}\@@DriverFieldPageOpen
%   \setFDFaction{PC}\@@DriverFieldPageClose
%   \setFDFaction{PV}\@@DriverFieldPageVisible
%   \setFDFaction{PI}\@@DriverFieldPageInVisible

%D The event handler becomes something:
%D
%D \starttyping
%D /AA << /D << /S ... >> ... /C << /S ... >>
%D /A << /S /JavaScript /JS (...) >>
%D \stoptyping

% \def\setFDFaction#1#2%
%   {\bgroup
%    \global\let\sanitizedJScode\empty
%    \def\setFDFaction##1%
%      {\doifreferencefoundelse{##1}
%         {\doifelse{\currentreferencespecial}{JS} % filter non-js
%            {\presetJScode
%               \currentreferenceoperation
%               \currentreferencearguments
%             \doPSsanitizeJScode\JScode\to\JScode
%             \xdef\sanitizedJScode{\sanitizedJScode\space\JScode}}
%            {\illegalreference{##1}}}
%         {\unknownreference{##1}}}%
%    \@EA\processcommalist\@EA[#2]\setFDFaction % one level expansion
%    \ifx\sanitizedJScode\empty \else
%      \xdef\FDFactions%
%        {\FDFactions /#1 << /S /JavaScript /JS (\sanitizedJScode) >> }%
%    \fi
%    \egroup}
%
% acrobat 5 supports other that JS actions too

\def\setFDFaction#1#2%
  {\bgroup
   \def\docommand{\xdef\FDFactions{\FDFactions /#1 << \lastPDFaction >> }}%
   \@EA\handlereferenceactions\@EA{#2}\docommand % one level expansion
   \egroup}

%D \macros
%D   {testFDFactions}
%D
%D This rather confusion prone series of script can be tested
%D with:
%D
%D \starttyping
%D \testFDFactions
%D \stoptyping
%D
%D which simply redefined the previous macro to one that prints
%D a message to the console.

\def\testFDFactions
  {\def\setFDFaction##1##2%
     {\doPSsanitizeJScode console.show();console.println("executing:##1"); \to\sanitizedJScode
      \edef\FDFactions{\FDFactions /##1 << /S /JavaScript /JS (\sanitizedJScode) >> }}}

%D \macros
%D   {doFDFregistercalculationset}
%D
%D There is at most one calculation order list, which defines
%D the order in which fields are calculated.

\let\PDFcalculationset\empty

\def\doFDFregistercalculationset#1%
  {\def\PDFcalculationset{#1}}

%D \macros
%D   {registerFDFobject,everylastshipout}
%D
%D Officially one needs to embed some general datastructures
%D that tell the viewer what fields are present in the file, as
%D well as what resources they use. The next mechanism does that
%D job automatically when one registers the field.

\def\flushFDFnames
  {\ifbuildFDFdictionary
     \ifx\FDFcollection\empty\else
       \ifbuildFDFencodingvector
         \doPDFdictionaryobject{FDF}{local:encodingvector}{\FDFencodingvector}%
       \fi
       \defineFDFfonts
       \doPDFarrayobject{FDF}{local:fields}{\FDFcollection}%
       \doPDFgetobjectreference{FDF}{local:fields}\PDFobjectreference
       % The /NeedAppearances is pretty important because
       % otherwise Acrobat 5 blows up on cloned radio widgets
       \doPDFdictionaryobject{FDF}{local:acroform}
         {/Fields \PDFobjectreference\space
          /NeedAppearances true
          \doFDFiffieldset\PDFcalculationset{/CO [\doFDFgetfieldset\PDFcalculationset]}
          /DR << /Font << \FDFfonts >> >>
          /DA (/Helv 10 Tf 0 g)}%
       \doPDFgetobjectreference{FDF}{local:acroform}\PDFobjectreference
       \doPDFaddtocatalog
         {/AcroForm \PDFobjectreference}%
       \global\let\FDFcollection\empty
       \global\let\flushFDFnames\relax
     \fi
   \fi}

\let\FDFcollection\empty

\def\registerFDFobject#1%
  {\ifbuildFDFdictionary
     \ifx\flushFDFnames\relax
       \writestatus{FDF}{second run needed for field list (#1)}%
     \fi
     \doPDFgetobjectreference{FDF}{#1}\PDFobjectreference
     \xdef\FDFcollection{\FDFcollection\space\PDFobjectreference}%
   \fi}

\appendtoksonce \flushFDFnames \to \everylastshipout % test \everybye / was \prependtoksonce

%D \macros
%D   {defineFDFfonts,
%D    ifbuildFDFdictionary,
%D    ifbuildFDFencodingvector}
%D
%D Another datastruture concerns the fonts used. We only
%D define the fonts we use.

\newif\ifbuildFDFdictionary     \buildFDFdictionarytrue
\newif\ifbuildFDFencodingvector \buildFDFencodingvectortrue

\def\defineFDFfonts
  {\let\FDFfonts\empty
   \processcommacommand[\FDFusedfonts]\defineFDFfont}

\def\defineFDFfont#1%
  {\ifbuildFDFencodingvector
     \doPDFgetobjectreference{FDF}{local:encodingvector}\PDFobjectreference
   \fi
   \doPDFdictionaryobject{FDF}{local:#1}
     {/Type /Font
      /Subtype /Type1
      /Name /#1
      \ifbuildFDFencodingvector /Encoding \PDFobjectreference\space\fi
      /BaseFont /\getvalue{FDFname#1}}%
   \doPDFgetobjectreference{FDF}{local:#1}\PDFobjectreference
   \edef\FDFfonts{\FDFfonts \space/#1 \PDFobjectreference}}

%D Another list of constants:

\def\FDFnameTiRo {Times-Roman}
\def\FDFnameTiBo {Times-Bold}
\def\FDFnameTiIt {Times-Italic}
\def\FDFnameTiBI {Times-BoldItalic}
\def\FDFnameHelv {Helvetica}
\def\FDFnameHeBo {Helvetica-Bold}
\def\FDFnameHeOb {Helvetica-Oblique}
\def\FDFnameHeBO {Helvetica-BoldOblique}
\def\FDFnameCour {Courier}
\def\FDFnameCoBo {Courier-Bold}
\def\FDFnameCoOb {Courier-Oblique}
\def\FDFnameCoBO {Courier-BoldOblique}

%D And a big one: (should be run time loaded (spec-run or
%D so)).

\def\FDFencodingvector
  {/Type /Encoding
   /Differences
     [ 24 /breve /caron /circumflex /dotaccent /hungarumlaut /ogonek
          /ring /tilde
       39 /quotesingle
       96 /grave
      128 /bullet /dagger /daggerdbl /ellipsis /emdash /endash /florin
          /fraction /guilsinglleft /guilsinglright /minus /perthousand
          /quotedblbase /quotedblleft /quotedblright /quoteleft
          /quoteright /quotesinglbase /trademark /fi /fl /Lslash /OE
          /Scaron /Ydieresis /Zcaron /dotlessi /lslash /oe /scaron
          /zcaron
      164 /currency
      166 /brokenbar
      168 /dieresis /copyright /ordfeminine
      172 /logicalnot /.notdef /registered /macron /degree /plusminus
          /twosuperior /threesuperior /acute /mu
      183 /periodcentered /cedilla /onesuperior /ordmasculine
      188 /onequarter /onehalf /threequarters 192 /Agrave /Aacute
          /Acircumflex /Atilde /Adieresis /Aring /AE /Ccedilla /Egrave
          /Eacute /Ecircumflex /Edieresis /Igrave /Iacute /Icircumflex
          /Idieresis /Eth /Ntilde /Ograve /Oacute /Ocircumflex /Otilde
          /Odieresis /multiply /Oslash /Ugrave /Uacute /Ucircumflex
          /Udieresis /Yacute /Thorn /germandbls /agrave /aacute
          /acircumflex /atilde /adieresis /aring /ae /ccedilla /egrave
          /eacute /ecircumflex /edieresis /igrave /iacute /icircumflex
          /idieresis /eth /ntilde /ograve /oacute /ocircumflex /otilde
          /odieresis /divide /oslash /ugrave /uacute /ucircumflex
          /udieresis /yacute /thorn /ydieresis ]}

%D \macros
%D   {currentFDFmode,currentFDFparent,currentFDFkids,currenrFDFroot}
%D
%D There are three more quasi global interfacing variables
%D that need to be set.

\let\currentFDFmode  =\fieldlonermode
\let\currentFDFkids  =\empty
\let\currentFDFparent=\empty
\let\currentFDFroot  =\empty

%D \macros
%D   {dosetfieldstatus}
%D
%D And here comes the special that deals with them.

\definespecial\dosetfieldstatus#1#2#3#4%
  {\chardef\currentFDFmode #1%
   \edef\currentFDFparent {#2}%
   \edef\currentFDFkids   {#3}%
   \edef\currentFDFroot   {#4}}

%D \macros
%D   {dosetuppageview}
%D
%D Because this command will seldom be called, we can permit
%D slow action processing. We need three settings, one for
%D direct \PDF\ inclusion, the other as \PDFTEX\ keyword, an
%D a last one for form. All determine in what way the
%D screen is adapted when going to a destination. Watch the
%D space.

\def\PDFpageviewkey{fit}
\def\PDFpageviewwrd{/Fit}
\def\PDFpageview   {/View [\PDFpageviewwrd] }
\def\PDFpagexyzspec{0 0 0} % hack, pdftex does handle this

\def\dosetuppageview#1% watch the v-h swapping here
  {\processaction
     [#1]
     [       \v!fit=>\def\PDFpageviewkey  {fit}\def\PDFpageviewwrd{/Fit},
           \v!width=>\def\PDFpageviewkey {fith}\def\PDFpageviewwrd{/FitH},
          \v!height=>\def\PDFpageviewkey {fitv}\def\PDFpageviewwrd{/FitV},
        \v!minwidth=>\def\PDFpageviewkey{fitbh}\def\PDFpageviewwrd{/FitBH},
       \v!minheight=>\def\PDFpageviewkey{fitbv}\def\PDFpageviewwrd{/FitBV},
        \v!standard=>\def\PDFpageviewkey{xyz \PDFpagexyzspec}\def\PDFpageviewwrd{/XYZ \PDFpagexyzspec},
         \s!unknown=>\def\PDFpageviewkey  {fit}\def\PDFpageviewwrd{/Fit}]%
   \edef\PDFpageview{/View [\PDFpageviewwrd]}}

%D \macros
%D   {setFDFkids}
%D
%D Clones as well as radiofields (which themselves can have
%D cloned components) need a list of kids. The next macro
%D builds one.

\def\setFDFkids[#1][#2]% tag commalist
  {\let\FDFkids\empty
   \def\docommand##1%
     {\doPDFgetobjectreference{FDF}{#1##1}\PDFobjectreference
      \edef\FDFkids{\FDFkids\PDFobjectreference\space}}%
   \@EA\processcommalist\@EA[#2]\docommand
   \ifx\FDFkids\empty\else\edef\FDFkids{/Kids [\FDFkids]}\fi
%    \edef\FDFkids{/Kids [\FDFkids]}%
   }

%D \macros
%D   {doFDFpresetlinefield,doFDFpresettextfield,
%D    doFDFpresetchoicefield,doFDFpresetpopupfield,doFDFpresetcombofield,
%D    doFDFpresetpushfield,doFDFpresetcheckfield,
%D    doFDFpresetfield,doFDFpresetradiorecord}
%D
%D I would say: read the \PDF\ reference manual first and see
%D what happens here next. Lucky us that they have so much in
%D common.

\def\doFDFpresetlinefield#1#2#3#4#5#6#7#8#9%
  {\bgroup
   \setFDFlayer\@@DriverFieldLayer
   \setFDFswitches[#7]%
   \setFDFattributes[#6]%
   \setFDFalignment[#8]%
   \setFDFactions[#9]%
   \expanded{\escapePSstring#4}\to\FDFtext
   \ifcase\currentFDFmode
     \doPDFannotationobject{FDF}{#1}{#2}{#3}
       {/Subtype /Widget /T (#1) /FT /Tx
        /MaxLen \ifcase0#5 1000 \else#5 \fi
       %/DV (#4) /V (#4) % value added
        /DV (\FDFtext) /V (\FDFtext) % value added
        /Ff \FDFflag\space
        /F \FDFplus\space
        /DA (\FDFattributes)
        \FDFlayer\space
        \FDFsurroundings\space
        /Q \FDFalign\space
        \FDFactions}%
     \registerFDFobject{#1}%
   \or
     \setFDFkids[kids:][\currentFDFkids]%
     \doPDFdictionaryobject{FDF}{#1}
       {/T (#1) /FT /Tx
        /MaxLen \ifcase0#5 1000 \else#5 \fi
        \FDFkids\space
       %/DV (#4) /V (#4) % value added
        /DV (\FDFtext) /V (\FDFtext) % value added
        /Ff \FDFflag\space
        /F \FDFplus\space
        /DA (\FDFattributes)
        \FDFlayer\space
        \FDFsurroundings\space
        /Q \FDFalign\space
        \FDFactions}%
     \registerFDFobject{#1}%
   \or
     \doPDFgetobjectreference{FDF}\currentFDFparent\PDFobjectreference
    %\global\objectreferencingtrue
     \doPDFannotationobject{FDF}{kids:#1}{#2}{#3}
       {/Subtype /Widget
        /Parent \PDFobjectreference
        /Ff \FDFflag\space
        /F \FDFplus\space
        /DA (\FDFattributes)
        \FDFlayer\space
        \FDFsurroundings\space
        /Q \FDFalign\space
        \FDFactions}%
   \or
     \doPDFgetobjectreference{FDF}\currentFDFparent\PDFobjectreference
    %\global\objectreferencingtrue
     \doPDFannotationobject{FDF}{kids:#1}{#2}{#3}
       {/Subtype /Widget
        /Parent \PDFobjectreference
        /F \FDFplus
        \FDFactions}%
   \fi
   \egroup}

\def\doFDFpresettextfield#1#2#3#4#5#6#7#8#9%
  {\doFDFpresetlinefield{#1}{#2}{#3}{#4}{#5}{#6}{MultiLine,#7}{#8}{#9}}

\def\doFDFpresetchoicefield#1#2#3#4#5#6#7#8%
  {\bgroup
   \setFDFlayer\@@DriverFieldLayer
   \setFDFswitches[#6]%
   \setFDFattributes[#5]%
   \setFDFvalues[#7][#4]%
   \setFDFactions[#8]%
   \ifcase\currentFDFmode
     \doPDFannotationobject{FDF}{#1}{#2}{#3}
       {/Subtype /Widget
        /T (#1) /FT /Ch
        /DV (#4) /V (#4)
        /Ff \FDFflag\space
        /F \FDFplus\space
        /DA (\FDFattributes)
        \FDFlayer\space
        \FDFsurroundings\space
        /Opt [\FDFvalues]
        \FDFactions}%
     \registerFDFobject{#1}%
   \or
     \setFDFkids[kids:][\currentFDFkids]%
     \doPDFdictionaryobject{FDF}{#1}
       {/T (#1) /FT /Ch
        \FDFkids\space
        /DV (#4) /V (#4)
        /Ff \FDFflag\space
        /F \FDFplus\space
        /DA (\FDFattributes)
        \FDFlayer\space
        \FDFsurroundings\space
        /Opt [\FDFvalues]
        \FDFactions}%
     \registerFDFobject{#1}%
   \or
     \doPDFgetobjectreference{FDF}\currentFDFparent\PDFobjectreference
    %\global\objectreferencingtrue
     \doPDFannotationobject{FDF}{kids:#1}{#2}{#3}
       {/Subtype /Widget
        /Parent \PDFobjectreference
        /Ff \FDFflag\space
        /F \FDFplus\space
        /DA (\FDFattributes)
        \FDFlayer\space
        \FDFsurroundings\space
        \FDFactions}%
   \or
     \doPDFgetobjectreference{FDF}\currentFDFparent\PDFobjectreference
    %\global\objectreferencingtrue
     \doPDFannotationobject{FDF}{kids:#1}{#2}{#3}
       {/Subtype /Widget
        /Parent \PDFobjectreference
        /F \FDFplus
        \FDFactions}%
   \fi
   \egroup}

\def\doFDFpresetpopupfield#1#2#3#4#5#6#7#8%
  {\doFDFpresetchoicefield{#1}{#2}{#3}{#4}{#5}{PopUp,#6}{#7}{#8}}

\def\doFDFpresetcombofield#1#2#3#4#5#6#7#8%
  {\doFDFpresetchoicefield{#1}{#2}{#3}{#4}{#5}{PopUp,Edit,#6}{#7}{#8}}

\newif\ifFDFvalues

\def\doFDFpresetpushcheckfield#1#2#3#4#5#6#7#8% in acro<5 (\FDFdefault)
  {\bgroup                                    % in acro>5 /\FDFdefault
   \setFDFlayer\@@DriverFieldLayer
   \ifcase#8\relax\FDFvaluesfalse\else\FDFvaluestrue\fi
   \setFDFswitches[#5]%
   \setFDFactions[#7]%
   \doifelse{#4}{1}
     {\def\FDFdefault{On}}
     {\def\FDFdefault{Off}}%
   \ifcase\currentFDFmode
     \doFDFappearance{On}{#6}{#8}%
     \doPDFannotationobject{FDF}{#1}{#2}{#3}
       {/Subtype /Widget /T (#1) /FT /Btn
        \ifFDFvalues
          /DV /\FDFdefault\space
          /V  /\FDFdefault\space
          /AS /\FDFdefault\space
        \fi
        \FDFlayer
        /Ff \FDFflag\space
        /F \FDFplus\space
        \FDFlayer\space
        \FDFappearance\space
% /IF << /SW /N >> % strange, only works for stupid buttons
        \FDFactions}%
     \registerFDFobject{#1}%
   \or % no appearance and layer ?
     \setFDFkids[kids:][\currentFDFkids]%
     \doPDFdictionaryobject{FDF}{#1}
       {/T (#1) /FT /Btn
        \FDFkids\space
        \ifFDFvalues
          /DV /\FDFdefault\space
          /V  /\FDFdefault\space
          /AS /\FDFdefault\space
        \fi
        /Ff \FDFflag\space
        /F \FDFplus\space
        \FDFactions}%
     \registerFDFobject{#1}%
   \or
     \doFDFappearance{On}{#6}{#8}%
     \doPDFgetobjectreference{FDF}\currentFDFparent\PDFobjectreference
    %\global\objectreferencingtrue
     \doPDFannotationobject{FDF}{kids:#1}{#2}{#3}
       {/Subtype /Widget
        /Parent \PDFobjectreference\space
        \ifFDFvalues
          /DV /\FDFdefault\space
          /V  /\FDFdefault\space
          /AS /\FDFdefault\space
        \fi
        /Ff \FDFflag\space
        /F \FDFplus\space
        \FDFlayer\space
        \FDFappearance\space
        \FDFactions}%
   \or
     \doFDFappearance{On}{#6}{#8}%
     \doPDFgetobjectreference{FDF}\currentFDFparent\PDFobjectreference
    %\global\objectreferencingtrue
     \doPDFannotationobject{FDF}{kids:#1}{#2}{#3}
       {/Subtype /Widget
        /Parent \PDFobjectreference\space
        /F \FDFplus\space
        \ifFDFvalues
          /DV /\FDFdefault\space
          /V  /\FDFdefault\space
          /AS /\FDFdefault\space
        \fi
        \FDFlayer\space
        \FDFappearance
        \FDFactions}%
   \fi
   \egroup}

\def\doFDFpresetpushfield#1#2#3#4#5#6#7%
  {\doFDFpresetpushcheckfield{#1}{#2}{#3}{#4}{PushButton,#5}{#6}{#7}{0}}

\def\doFDFpresetcheckfield#1#2#3#4#5#6#7%
  {\doFDFpresetpushcheckfield{#1}{#2}{#3}{#4}{#5}{#6}{#7}{1}}

\def\doFDFpresetradiofield#1#2#3#4#5#6#7#8%
  {\bgroup
   \setFDFlayer\@@DriverFieldLayer
   \FDFvaluestrue
   \setFDFswitches[#5]%
   \setFDFactions[#8]%
   \doifelsenothing{#4}
     {\def\FDFdefault{Off}}
     {\def\FDFdefault{#4}}%
   \@EA\aftersplitstring\FDFdefault\at=>\to\FDFdefaultvalue
   \ifx\FDFdefaultvalue\empty\else\let\FDFdefault\FDFdefaultvalue\fi
   \ifcase\currentFDFmode
     \doFDFappearance{#1}{#7}{1}%
     \doPDFgetobjectreference{FDF}{#6}\PDFobjectreference
     \doPDFannotationobject{FDF}{#1}{#2}{#3}
       {/Subtype /Widget
        /Parent \PDFobjectreference\space
        /F \FDFplus\space
        /AS /\FDFdefault\space
        \FDFlayer\space
        \FDFappearance\space
        \FDFactions}%
     \registerFDFobject{#1}%
   \or
     \setFDFkids[kids:][\currentFDFkids]%
     \doPDFgetobjectreference{FDF}{#6}\PDFobjectreference
     \doPDFdictionaryobject{FDF}{#1}
       {/Parent \PDFobjectreference\space
        \FDFkids\space
        /F \FDFplus\space
        \FDFactions}%
     \registerFDFobject{#1}%
   \or
     %\doFDFappearance{#1}{#7}{1}%
     \doFDFappearance{\currentFDFparent}{#7}{1}%
     \doPDFgetobjectreference{FDF}\currentFDFparent\PDFobjectreference
    %\global\objectreferencingtrue % nb
     \doPDFannotationobject{FDF}{kids:#1}{#2}{#3}
       {/Subtype /Widget
        /Parent \PDFobjectreference\space
        /AS /\FDFdefault\space
        /F \FDFplus\space
        \FDFlayer\space
        \FDFappearance\space
        \FDFactions}%
   \or
     %\doFDFappearance{#1}{#7}{1}%
     \doFDFappearance{\currentFDFparent}{#7}{1}%
     \doPDFgetobjectreference{FDF}\currentFDFparent\PDFobjectreference
    %\global\objectreferencingtrue
     \doPDFannotationobject{FDF}{kids:#1}{#2}{#3}
       {/Subtype /Widget
        /Parent \PDFobjectreference\space
        /AS /\FDFdefault\space
        /F \FDFplus\space
        \FDFlayer\space
        \FDFappearance\space
        \FDFactions}%
   \fi
   \egroup}

%D \macros
%D   {setFDFstrings}
%D
%D This one creates a string array.

%\def\setFDFstrings[#1]%
%  {\let\FDFstrings\empty
%   \def\docommand##1{\edef\FDFstrings{\FDFstrings(##1)}}%
%   \processcommacommand[#1]\docommand}

% Beware, RadiosInUnison is really needed in the pre 1.5/6 time this
% was the default but out of a sudden it's no longer the case.  Also
% the NoToggleToOff interferes with kids of kids and both it will
% break older documents, i.e. so much for pdf as standard. With
% features like widgets we can probably best wait till adobe tools
% themselves support it because that's probably the moment that
% functionality gets frozen/becomes definitive. Actually, acrobat
% flattens the kids tree, so that's yet another situation. The
% interesting thing is that it worked ok in acrobat 2/3 but got bugged
% in later versions. [The rationale is in html compatibility, which
% seems to be more important than compatibility of documents, which in
% turn renders acrobat useless for forms.] Anyway, synchronization is
% broken or not depending on the combination pdfversion/acrobatversion.

\def\doFDFpresetradiorecord#1#2#3#4#5%
  {\bgroup
   % < pdf 1.5 (1.5 was broken)
   % \setFDFswitches[Radio,NoToggleToOff,RadiosInUnison,#3]%
   % > pdf 1.5
   \setFDFswitches[Radio,RadiosInUnison,#3]%
   % older, else fatal error
   % \setFDFkids[#4][]%
   % newer
   \setFDFvalues[#4][#2]% inits kidlist
   \expanded{\setFDFkids[][\FDFkidlist]}%
   %
   \setFDFactions[#5]%
   \doPDFdictionaryobject{FDF}{#1}
     {%/Subtype /Widget
      /FT /Btn /T (#1) /Rect [0 0 0 0]
      % used to be this
      % /V (#2)
      % then this
      % /DV (#2)
      % since this bomded in 5
      % /V (#2)
      % and now finally this works
      /H /N
      % /opt is buggy in 5.05, only works once, sigh
      %\ifx\FDFfirstvalues\FDFsecondvalues
        /V /#2
      %\else
      %  /V /\FDFdefaultindex\space
      %  /Opt [\FDFsecondvalues]
      %\fi
      /Ff \FDFflag\space
      /F \FDFplus\space
      \FDFkids\space
      \FDFactions}%
   \egroup}

%D At the cost of some more references, we can save bytes,
%D by sharing appearance dictionaries. This code needs more
%D documentation. Surprise:

\def\dodoFDFappearance#1#2%
  {\ifx#2\empty\else
     \dogetcommacommandelement1\from#2\to\commalistelement
     \ifx\commalistelement\empty\else
       \doPDFgetobjectreference{SYM}\commalistelement\PDFobjectreference
       \edef\N{\ifFDFvalues\N /#1 \fi\PDFobjectreference\space}%
     \fi
     \dogetcommacommandelement2\from#2\to\commalistelement
     \ifx\commalistelement\empty\else
       \doPDFgetobjectreference{SYM}\commalistelement\PDFobjectreference
       \edef\R{\ifFDFvalues\R /#1 \fi\PDFobjectreference\space}%
     \fi
     \dogetcommacommandelement3\from#2\to\commalistelement
     \ifx\commalistelement\empty\else
       \doPDFgetobjectreference{SYM}\commalistelement\PDFobjectreference
       \edef\D{\ifFDFvalues\D /#1 \fi\PDFobjectreference\space}%
       \def\FDFappearance{/H /P }%
     \fi
   \fi}

\def\redoFDFappearance#1%
  {\ifx#1\empty\else
     \dogetcommacommandelement3\from#1\to\commalistelement
     \ifx\commalistelement\empty\else
       \def\FDFappearance{/H /P }%
     \fi
   \fi}

\def\doFDFappearance#1#2#3%
  {\ifcase#3\relax % push only field
     \edef\yes{#2}%
     \let\no\empty
   \else % on / off field
     \dogetcommacommandelement1\from#2,\to\yes
     \dogetcommacommandelement2\from#2,\to\no
   \fi
   \def\FDFappearance{/H /N}%
   \doifobjectfoundelse{FDF}{ap:#1:\yes:\no}
     {\redoFDFappearance\yes
      \redoFDFappearance\no}
     {\presetobject{FDF}{ap:#1:\yes:\no}% funny hack
      \let\N\empty\let\R\empty\let\D\empty
      \dodoFDFappearance{#1}\yes
      \dodoFDFappearance{Off}\no
      \doPDFdictionaryobject{FDF}{ap:#1:\yes:\no}
        {\ifx\N\empty\else/N \ifFDFvalues<<\N>>\else\N\fi\fi
         \ifx\R\empty\else/R \ifFDFvalues<<\R>>\else\R\fi\fi
         \ifx\D\empty\else/D \ifFDFvalues<<\D>>\else\D\fi\fi}}%
   \doPDFgetobjectreference{FDF}{ap:#1:\yes:\no}\PDFobjectreference
   \edef\FDFappearance{\FDFappearance /AP \PDFobjectreference}}

%\def\doFDFdefault#1#2%
%  {\dogetcommacommandelement1\from#1,\to\commalistelement
%   \dogetcommacommandelement1\from\commalistelement\to\commalistelement
%   \doifelse{\commalistelement}{#2} % kan ook met \ifx
%     {\def\FDFdefault{On}}{\def\FDFdefault{Off}}}

\def\doFDFdefault#1#2%
  {\doifelse{#2}{1}{\def\FDFdefault{On}}{\def\FDFdefault{Off}}}

%D Layer support:

\def\setFDFlayer#1% todo : \ifx\PDFobjectreference\noPDFobjectreference ipv found
  {\letempty\FDFlayer
   \doifsomething{#1}%
     {\checkproperty[#1]% == \dodocheckproperty\@@DriverFieldLayer
      \doifobjectreferencefoundelse{PDLN}{#1}
        {\doPDFgetobjectreference{PDLN}{#1}\!!stringa % we need to avoid a clash with other macros
         \edef\FDFlayer{/OC \!!stringa}}%
        \donothing}}

%D The three appearances {\em normal}, \type{roll over} and
%D \type{push down} are passed as comma separated triplets,
%D that is, the second argument can look like:
%D
%D \starttyping
%D {yes,ok,fine},{no,rubish,awful}
%D \stoptyping

%D \macros
%D   {doFDFdefinefieldset,doFDFgetfieldset,doFDFiffieldset}
%D
%D Field sets, the ones we use in submitting and resetting
%D fields, are implemented using the next low level specials:
%D
%D \starttyping
%D \doFDFdefinefieldset{TAG}{name,name,...}
%D \doFDFgetfieldset{TAG}
%D \doFDFiffieldset{TAG}{sequence}
%D \stoptyping

\def\doFDFdefinefieldset#1#2% tag commalist
  {\let\FDFfieldset\empty
   \def\docommand##1%
     {\doPDFgetobjectreference{FDF}{##1}\PDFobjectreference
      \edef\FDFfieldset{\FDFfieldset\PDFobjectreference\space}}%
   \processcommacommand[#2]\docommand % nb: command
   \setevalue{FDF:set:#1}{\FDFfieldset}}

\def\doFDFgetfieldset#1%
  {\getvalue{FDF:set:#1}}

\def\doFDFiffieldset#1#2%
  {\ifundefined{FDF:set:#1}\else#2\fi}

%D In the goto specials we took care of secondary references.
%D Here we define the macros used.

\def\doPDFresetgotowhereever
  {\global\let\secondaryPDFreferences\empty}

\doPDFresetgotowhereever % just to be sure

% we can (in etex) share more by testing on this

\def\savesecondaryPDFreference#1%
 %{\setxvalue{PDF-SR:\the\nofsecondaryreferences}{#1}}
  {\@EA\xdef\csname PDF-SR:\the\nofsecondaryreferences\endcsname{#1}}

\def\savesecondaryPDFreference % #1 == \action
  {\global\@EA\let\csname PDF-SR:\the\nofsecondaryreferences\endcsname}

%\def\getsecondaryPDFreferences%
%  {\ifcase\nofsecondaryreferences\else
%    %\doifdefined{PDF-SR:\the\nofsecondaryreferences}
%     \xdef\secondaryPDFreferences%
%       {/Next << \csname PDF-SR:\the\nofsecondaryreferences\endcsname\space
%                 \secondaryPDFreferences >>}%
%     \global\advance\nofsecondaryreferences \minusone
%     \expandafter\getsecondaryPDFreferences
%   \fi}

% test should happen in core-ref

\beginETEX \ifcsname

\def\getsecondaryPDFreferences
  {\ifcase\nofsecondaryreferences\else
     \ifcsname PDF-SR:\the\nofsecondaryreferences\endcsname
       \xdef\secondaryPDFreferences
         {/Next << \csname PDF-SR:\the\nofsecondaryreferences\endcsname\space
                   \secondaryPDFreferences >>}%
     \fi
     \global\advance\nofsecondaryreferences \minusone
     \expandafter\getsecondaryPDFreferences
   \fi}

\endETEX

\beginTEX

\def\getsecondaryPDFreferences
  {\ifcase\nofsecondaryreferences\else
     \@EA\ifx\csname PDF-SR:\the\nofsecondaryreferences\endcsname\relax\else
       \xdef\secondaryPDFreferences
         {/Next << \csname PDF-SR:\the\nofsecondaryreferences\endcsname\space
                   \secondaryPDFreferences >>}%
     \fi
     \global\advance\nofsecondaryreferences \minusone
     \expandafter\getsecondaryPDFreferences
   \fi}

\endTEX

%D \macros
%D   {loadFDFfields, showFDFfields,
%D    getFDFfield, setFDFfield}
%D
%D Once filled in, we can export or submit the field in the
%D \FDF\ file format. Such a file can be loaded by
%D
%D \starttyping
%D \loadFDFfields{fiel-ini}
%D \stoptyping
%D
%D or inspected by
%D
%D \starttyping
%D \showFDFfields{fiel-ini}
%D \stoptyping
%D
%D After both commands, one can use
%D
%D \starttyping
%D \getFDFfield{name}
%D \setFDFfield{name}{value}
%D \stoptyping
%D
%D to inspect and overrule the data.
%D
%D By default \CONTEXT\ calls the perl script \type{fdf2tex}.
%D This script reads the \type{fdf} file and produces a file
%D named \type{filename.fdt}. If one disables the call to this
%D script, by saying:
%D
%D \starttyping
%D \runFDFconverterfalse
%D \stoptyping
%D
%D or when \CONTEXT\ cannot find the \type{fdt} file, it tries
%D to interpret the \type{fdf} file directly. Both mechanisms
%D are rather crude.

\newif\ifrunFDFconverter \runFDFconvertertrue

%D The \PERL\ script produces a file formatted as:
%D
%D \starttyping
%D \beginFDFobject
%D \beginFDFdata
%D \beginFDFfields
%D \FDFfield[name=,value=]
%D \endFDFfields
%D \endFDFdata
%D \endFDFobject
%D \stoptyping
%D
%D One reason for using key value pairs is that we cannot be
%D sure or the order in which the name and value are given
%D (actually the reverse).

\def\PERLloadFDFfields#1% will become obsolete soon
  {\bgroup
   \global\let\allFDFfields\empty
   \ifrunFDFconverter
     \executesystemcommand{fdf2tex #1}%
     \let\beginFDFobject\relax \let\endFDFobject\relax
     \let\beginFDFdata  \relax \let\endFDFdata  \relax
     \let\beginFDFfields\relax \let\endFDFfields\relax
     \def\FDFfield[##1]%
       {\getparameters[FDF][##1]%
        \doglobal\addtocommalist\FDFname\allFDFfields
        \global\setFDFfield{\FDFname}{\FDFvalue}}%
     \ReadFile{#1.fdt}%
   \fi
   \egroup}

%D The next macro does the same job, but now in the \TEX\ way
%D of doing things. Easy eh? Will become obsolete!

\bgroup

\catcode`\/=\@@other
\global\let\normalslash=/
\catcode`\/=\@@escape

\gdef\TEXloadFDFfields#1% will become obsolete due to XFDF
  {\bgroup
   \setbox0=\hbox
     {\global\let\allFDFfields\empty
      \scratchcounter=0
      \escapechar=-1
      \catcode`\/=\@@escape
      \catcode`\(=\@@begingroup
      \catcode`\)=\@@endgroup
      \catcode`\%=\@@letter
      \let/A =\relax \let/AS =\relax \let/Kids=\relax \let/Fields=\relax
      \let/F =\relax \let/ID =\relax \let/SetF=\relax \let/setFf =\relax
      \let/Ff=\relax \let/Opt=\relax \let/ClrF=\relax \let/ClrFf =\relax
      \let/AP=\relax \let/FDF=\relax \let/Root=\relax
      \def/T##1{\check\Title{##1}}
      \def/V{\bgroup\catcode`\/=\@@other\futurelet\next/doV}
      \def/doV{\ifx\next\normalslash\@EA/doVb\else\@EA/doVa\fi}
      \def/doVa##1{\egroup\check\Value{##1}}
      \def/doVb##1##2 {\egroup\check\Value{##2}} % watch the space
      \def\check##1##2%
        {\def##1{##2}
         \advance\scratchcounter by 1
         \ifodd\scratchcounter \else
           \convertcommand\Title\to\asciia
           \global\setFDFfield{\asciia}{\Value}
           \doglobal\addtocommalist\Title\allFDFfields
         \fi}
      \ReadFile{#1.fdf}}%
   \egroup}

\egroup

%D Whatever mechanism is used, the next macros can be used to
%D fetch the values.

\def\getFDFfield  #1{\getvalue {FDFfield::#1}}
\def\setFDFfield#1#2{\setevalue{FDFfield::#1}{#2}}

%D Of course the fields are only present when the file is
%D loaded.

\def\loadFDFfields#1%
  {\PERLloadFDFfields{#1}%
   \ifx\allFDFfields\empty
     \TEXloadFDFfields{#1}%
   \fi}

\def\showFDFfields#1%
  {\bgroup
   \loadFDFfields{#1}
   \def\docommand##1{\par##1 = \getFDFfield{##1}\par}%
   \processcommacommand[\allFDFfields]\docommand
   \egroup}

%D \macros
%D   {sanitizePDFencoding,sanitizePDFdocencoding}
%D
%D We already dealt with the encoding vector. Conversion from
%D \TEX\ \ASCII\ encoding to the other one, is accomplished by
%D the next few macros. Wach out: we don't group here.

%D This will be reimplemented using the mapping mechanism.

% \def\enablePDFdocencoding
%   {\reducetocoding[pdfdoc]\simplifycommands}

\def\enablePDFdocencoding
  {\enablecoding[pdfdoc]%
   \enablelanguagespecifics[\currentlanguage]% redundant ?
   \simplifycommands}

\long\def\sanitizePDFdocencoding#1\to#2%
  {\enablePDFcrlf
   \enablePDFdocencoding
  %\honorunexpanded % otherwise problems with "e etc in de
   \edef#2{#1}}

\bgroup
\catcode`\^^M=\@@active
\gdef\enablePDFcrlf%
  {\def\\{\string\r}%
   \def\par{\\\\}%
   \def\endgraf{\\\\}%
   \catcode`\^^M=\@@active%
   \let^^M=\\}
\egroup

\let\sanitizePDFencoding\sanitizePDFdocencoding

%D A bit out of place, but useful:

\newevery\everysetfield\relax

\appendtoksonce
  \enablePDFdocencoding
  \enablePDFcrlf
\to \everysetfield

%D \macros
%D   {doPDFinsertcomment}
%D
%D An example its use is the next special, one that deals with
%D text annotations.

% starting point (keep this)
%
% \long\def\doPDFinsertcomment#1#2#3#4#5#6#7#8%
%   {\bgroup % title width height color open symbol collect data
%    \doifelsenothing{#1}
%      {\let\PDFidentifier\empty}
%      {\def\PDFidentifier{/T (#1)}}%
%    \doifelsenothing{#4}
%      {\let\PDFattributes\empty}
%      {\def\PDFattributes{/C \FDFcolor{#4}}}%
%    \doifundefinedelse{PDFsymbol#6}
%      {\let\PDFsymbol\empty}
%      {\def\PDFsymbol{/Name \getvalue{PDFsymbol#6} }}%
%    \sanitizePDFencoding#8\to\PDFdata
%    \setbox\scratchbox\vbox to #3
%      {\vfill
%       \doPDFannotation{#2}{#3}
%         {/Subtype /Text
%          \ifcase#5 \else/Open true\fi
%          /Contents \ifPDFunicode <\PDFdata> \else(\PDFdata) \fi
%          \PDFsymbol
%          \PDFidentifier
%          \PDFattributes}}%
%    \wd\scratchbox\zeropoint
%    \ht\scratchbox\zeropoint
%    \dp\scratchbox\zeropoint
%    \box\scratchbox
%    \egroup}

\newcounter\nofFDFcomments

\newif\ifPDFpopupcomments \PDFpopupcommentstrue

\def\doPDFflushcomments
  {\box\PDFsymbolbox}

\long\def\doPDFinsertcomment#1#2#3#4#5#6#7#8% % \@@DriverCommentLayer set otherwise
  {\bgroup % title width height color open symbol collect data
   \presetPDFsymbolappearance{#4}{#6}{#2}{#3}\!!zeropoint% sets width/height
%    \doifelsenothing{#1}
%      {\let\PDFidentifier\empty}
%      {\def\PDFidentifier{/T (#1)}}%
   \doifelsenothing{#1}
     {\let\PDFidentifier\empty}
     {\sanitizePDFencoding#1\to\PDFcommenttitle
      \def\PDFidentifier{/T \ifPDFunicode <\PDFcommenttitle>\else (\PDFcommenttitle)\fi}}%
   \sanitizePDFencoding#8\to\PDFdata
   \setFDFlayer\@@DriverCommentLayer
   \startPDFsymbolappearance
     \ifPDFpopupcomments
       \doglobal\increment\nofFDFcomments
       \doifobjectreferencefoundelse{FDF}{c:\nofFDFcomments}
         {\doPDFgetobjectreference{FDF}{c:\nofFDFcomments}\PDFobjectreference
          \donetrue}
         \donefalse
       \ifdone
         \setbox\scratchbox\hbox
           {\doPDFannotationobject{FDF}{c::\nofFDFcomments}{#2}{#3}% text window, size does not work
              {/Subtype /Popup
               /Parent \PDFobjectreference}}%
         \ifcase#7\relax
           \vbox to \height{\forgetall\vskip#3\box\scratchbox\vss}%
         \else % incredible trial and error hack
           % it's quite a mess, the annot width cannot be set, well, it can
           % but the appearance and text sizes get mixed up
%            \setbox\scratchbox\vbox to \height{\forgetall\vskip#3\box\scratchbox\vss}%
%            \global\setbox\PDFsymbolbox\vbox
%              {\hsize#2%
%               \forgetall
%               \vsmash{\box\PDFsymbolbox}
%               \box\scratchbox}%
           % this may change when acrobat gets less bugged
           \setbox\scratchbox\vbox to #3{\forgetall\vss\box\scratchbox}%
           \wd\scratchbox#2%
           \global\setbox\PDFsymbolbox\vbox
             {\startoverlay{\box\PDFsymbolbox}{\box\scratchbox}\stopoverlay}%
         \fi
       \fi
       % generic
       \doifobjectreferencefoundelse{FDF}{c::\nofFDFcomments}
         {\doPDFgetobjectreference{FDF}{c::\nofFDFcomments}\PDFobjectreference
          \donetrue}
         \donefalse
       \doPDFannotationobject{FDF}{c:\nofFDFcomments}{\width}{\height}
         {/Subtype /Text
          \ifcase#5 \else/Open true\fi
          % pdftex (efficient)
          % \ifdone /Popup \PDFobjref\pdflastannot\fi
          % generic (less efficient)
          \ifdone /Popup \PDFobjectreference\fi
          /Contents \ifPDFunicode <\PDFdata> \else(\PDFdata) \fi
          \PDFidentifier
          \FDFlayer
          \PDFsymbol
          \PDFattributes}%
     \else
       \doPDFannotation{#2}{#3}
         {/Subtype /Text
          \ifcase#5 \else/Open true\fi
          /Contents \ifPDFunicode <\PDFdata> \else(\PDFdata) \fi
          \FDFlayer
          \PDFsymbol
          \PDFidentifier
          \PDFattributes}%
     \fi
   \stopPDFsymbolappearance
   \egroup}

% symbols with a reasonable default of 18/24 pt

\newbox\PDFsymbolbox

\def\PDFsymbolNew       {/Insert}
\def\PDFsymbolBalloon   {/Comment}
\def\PDFsymbolAddition  {/NewParagraph}
\def\PDFsymbolHelp      {/Help}
\def\PDFsymbolParagraph {/Paragraph}
\def\PDFsymbolKey       {/Key }

\def\PDFsymbolGraph     {/Graph}
\def\PDFsymbolPaperclip {/Paperclip}
\def\PDFsymbolAttachment{/Attachment}
\def\PDFsymbolTag       {/Tag}

\def\startPDFsymbolappearance
  {\setbox\scratchbox\vbox to \totalheight \bgroup \vfill}

\def\stopPDFsymbolappearance
  {\egroup
   \setbox\scratchbox\hbox{\lower\depth\box\scratchbox}%
   \wd\scratchbox\width
   \ht\scratchbox\height
   \dp\scratchbox\depth
   \box\scratchbox}

\def\presetPDFsymbolappearance#1#2#3#4#5% symbol color width height depth
  {\doifelsenothing{#1}
     {\let\PDFattributes\empty}
     {\def\PDFattributes{/C \FDFcolor{#1}}}%
   \scratchdimen#3\edef\width {\the\scratchdimen}%
   \scratchdimen#4\edef\height{\the\scratchdimen}%
   \scratchdimen#5\edef\depth {\the\scratchdimen}%
   \advance\scratchdimen\height\edef\totalheight{\the\scratchdimen}%
   \doifelsenothing{#2}
     {\let\PDFsymbol\empty}
     {\ifundefined{PDFsymbol#2}%
        \getfromcommacommand[#2][1]\let\PDFsymbolnormalsymbol\commalistelement
        \getfromcommacommand[#2][2]\let\PDFsymboldownsymbol  \commalistelement
        \doifsymboldefinedelse\PDFsymbolnormalsymbol
          {\doifsymboldefinedelse\PDFsymboldownsymbol
             {\dopresetPDFsymbolappearance
                \PDFsymbolnormalsymbol\PDFsymboldownsymbol}
             {\dopresetPDFsymbolappearance
                \PDFsymbolnormalsymbol\PDFsymbolnormalsymbol}}
          {\doifsymboldefinedelse\PDFsymboldownsymbol
             {\dopresetPDFsymbolappearance
                \PDFsymboldownsymbol\PDFsymboldownsymbol}
             {\let\PDFsymbol\empty}}%
      \else
        \def\PDFsymbol{/Name \getvalue{PDFsymbol#2} }%
      \fi}}

\def\dopresetPDFsymbolappearance#1#2%
  {\dopresetfieldsymbol{#1}%
   \dopresetfieldsymbol{#2}%
   \setbox\scratchbox\hbox{\symbol[#1]}%
   \edef\width {\the\wd\scratchbox}%
   \edef\height{\the\ht\scratchbox}%
   \edef\depth {\the\dp\scratchbox}%
   \scratchdimen\height \advance\scratchdimen\depth
   \edef\totalheight{\the\scratchdimen}%
   \doPDFgetobjectreference{SYM}{#1}\FDFsymbolNappearance
   \doPDFgetobjectreference{SYM}{#2}\FDFsymbolDappearance
   \edef\PDFsymbol
     {/AP <</N \FDFsymbolNappearance /D \FDFsymbolDappearance>>}}

%D Hooked into \CONTEXT, this special supports
%D
%D \starttyping
%D \startcomment
%D   hello beautiful\\world
%D \stopcomment
%D
%D \startcomment[hello]
%D   de \'e\'erste keer
%D   the f\'irst time
%D \stopcommen
%D
%D \startcommentaar[hallo][color=green,width=4cm,height=3cm]
%D   first
%D
%D   second
%D \stopcommentaar
%D \stoptyping
%D
%D So, special characters, forced linebreaks using \type{\\}
%D and \type{\par} are handled in the appropriate way.

%D \macros
%D  {doPDFovalbox}
%D
%D For drawing ovals we use quite raw \PDF\ code. The next
%D implementation does not differ that much from the one
%D implemented in the \POSTSCRIPT\ driver.

\def\doPDFovalcalc#1#2#3%
  {\dimen2=#1%
   \advance\dimen2 #2\relax
   \PointsToBigPoints{\dimen2}#3}

\def\doPDFovalbox#1#2#3#4#5#6#7#8% todo: \scratchdimen/\scatchbox
  {\bgroup
   \dimen0=#4\divide\dimen0 \plustwo
   \doPDFovalcalc{0pt}{+\dimen0}\xmin
   \doPDFovalcalc{#1}{-\dimen0}\xmax
   \doPDFovalcalc{#2}{-\dimen0}\ymax
   \doPDFovalcalc{-#3}{+\dimen0}\ymin
   \advance\dimen0 by #5%
   \doPDFovalcalc{0pt}{+\dimen0}\xxmin
   \doPDFovalcalc{#1}{-\dimen0}\xxmax
   \doPDFovalcalc{#2}{-\dimen0}\yymax
   \doPDFovalcalc{-#3}{+\dimen0}\yymin
   \doPDFovalcalc{#4}{\zeropoint}\stroke
   \doPDFovalcalc{#5}{\zeropoint}\radius
   \edef\dostroke{#6}%
   \edef\dofill{#7}%
   \edef\mode{\number#8 \space}%
   % no \ifcase, else \relax in pdfcode
   \setbox\scratchbox\hbox
     {\ifnum\dostroke\dofill>\zerocount
        \ifPDFstrokecolor\else\ifnum\dostroke=\plusone
          \writestatus\m!colors{pdf stroke color will fail}\wait
        \fi\fi
        \PDFcode
          {q
           \stroke\space               w
           \ifcase\mode
             \xxmin\space \ymin \space m
             \xxmax\space \ymin \space l
             \xmax \space \ymin \space \xmax \space \yymin\space y
             \xmax \space \yymax\space l
             \xmax \space \ymax \space \xxmax\space \ymax \space y
             \xxmin\space \ymax \space l
             \xmin \space \ymax \space \xmin \space \yymax\space y
             \xmin \space \yymin\space l
             \xmin \space \ymin \space \xxmin\space \ymin \space y
             h
           \or % 1
             \xxmin\space \ymin \space m
             \xxmax\space \ymin \space l
             \xmax \space \ymin \space \xmax \space \yymin\space y
             \xmax \space \ymax \space l
             \xmin \space \ymax \space l
             \xmin \space \yymin\space l
             \xmin \space \ymin \space \xxmin\space \ymin \space y
             h
           \or % 2
             \xxmin\space \ymin \space m
             \xmax \space \ymin \space l
             \xmax \space \ymax \space l
             \xxmin\space \ymax \space l
             \xmin \space \ymax \space \xmin \space \yymax\space y
             \xmin \space \yymin\space l
             \xmin \space \ymin \space \xxmin\space \ymin \space y
             h
           \or % 3
             \xmin \space \ymin \space m
             \xmax \space \ymin \space l
             \xmax \space \yymax\space l
             \xmax \space \ymax \space \xxmax\space \ymax \space y
             \xxmin\space \ymax \space l
             \xmin \space \ymax \space \xmin \space \yymax\space y
             \xmin \space \ymin \space l
             h
           \or % 4
             \xmin \space \ymin \space m
             \xxmax\space \ymin \space l
             \xmax \space \ymin \space \xmax \space \yymin\space y
             \xmax \space \yymax\space l
             \xmax \space \ymax \space \xxmax\space \ymax \space y
             \xmin \space \ymax \space l
             \xmin \space \ymin\space l
             h
           \or % 5
             \xmin \space \ymin \space m
             \xmax \space \ymin \space l
             \xmax \space \yymax\space l
             \xmax \space \ymax \space \xxmax\space \ymax \space y
             \xmin \space \ymax \space l
             \xmin \space \ymin \space l
             h
           \or % 6
             \xmin \space \ymin \space m
             \xxmax\space \ymin \space l
             \xmax \space \ymin \space \xmax \space \yymin\space y
             \xmax \space \ymax \space l
             \xmin \space \ymax \space l
             \xmin \space \ymin \space l
             h
           \or
             \xxmin\space \ymin \space m
             \xmax \space \ymin \space l
             \xmax \space \ymax \space l
             \xmin \space \ymax \space l
             \xmin \space \yymin\space l
             \xmin \space \ymin \space \xxmin\space \ymin \space y
             h
           \or
             \xmin \space \ymin \space m
             \xmax \space \ymin \space l
             \xmax \space \ymax \space l
             \xxmin\space \ymax \space l
             \xmin \space \ymax \space \xmin \space \yymax\space y
             \xmin \space \ymin \space l
             h
           \or % 9 top open
             \xmin \space \ymax \space m
             \xmin \space \yymin\space l
             \xmin \space \ymin \space \xxmin\space \ymin \space y
             \xxmax\space \ymin \space l
             \xmax \space \ymin \space \xmax \space \yymin\space y
             \xmax \space \ymax \space l
           \or % 10 right open
             \xmax \space \ymax \space m
             \xxmin\space \ymax \space l
             \xmin \space \ymax \space \xmin \space \yymax\space y
             \xmin \space \yymin\space l
             \xmin \space \ymin \space \xxmin\space \ymin \space y
             \xmax\space  \ymin \space l
           \or % 11 bottom open
             \xmax \space \ymin \space m
             \xmax \space \yymax\space l
             \xmax \space \ymax \space \xxmax \space \ymax\space y
             \xxmin\space \ymax \space l
             \xmin \space \ymax \space \xmin \space \yymax\space y
             \xmin \space \ymin \space l
           \or % 12 left open
             \xmin \space \ymax \space m
             \xxmax\space \ymax \space l
             \xmax \space \ymax \space \xmax \space \yymax\space y
             \xmax \space \yymin\space l
             \xmax \space \ymin \space \xxmax\space \ymin \space y
             \xmin \space \ymin \space l
           \or % 13
             \xmin \space \ymax \space m
             \xxmax\space \ymax \space l
             \xmax \space \ymax \space \xmax \space \yymax\space y
             \xmax\space  \ymin \space l
           \or % 14
             \xmax \space \ymax \space m
             \xmax \space \yymin\space l
             \xmax \space \ymin \space \xxmax\space \ymin \space y
             \xmin \space \ymin \space l
           \or % 15
             \xmax \space \ymin \space m
             \xxmin\space \ymin \space l
             \xmin \space \ymin \space \xmin \space \yymin\space y
             \xmin \space \ymax \space l
           \or % 16
             \xmin \space \ymin \space m
             \xmin \space \yymax\space l
             \xmin \space \ymax \space \xxmin\space \ymax \space y
             \xmax \space \ymax \space l
           \or % 17
             \xxmax\space \ymax \space m
             \xmax \space \ymax \space \xmax \space \yymax\space y
           \or % 18
             \xmax \space \yymin\space m
             \xmax \space \ymin \space \xxmax\space \ymin \space y
           \or % 19
             \xxmin\space \ymin \space m
             \xmin \space \ymin \space \xmin \space \yymin\space y
           \or % 20
             \xmin \space \yymax\space m
             \xmin \space \ymax \space \xxmin\space \ymax \space y
           \or % 21
             \xxmax\space \ymax \space m
             \xmax \space \ymax \space \xmax \space \yymax\space y
             \xmin \space \yymax\space m
             \xmin \space \ymax \space \xxmin\space \ymax \space y
           \or % 22
             \xxmax\space \ymax \space m
             \xmax \space \ymax \space \xmax \space \yymax\space y
             \xmax \space \yymin\space m
             \xmax \space \ymin \space \xxmax\space \ymin \space y
           \or % 23
             \xmax \space \yymin\space m
             \xmax \space \ymin \space \xxmax\space \ymin \space y
             \xxmin\space \ymin \space m
             \xmin \space \ymin \space \xmin \space \yymin\space y
           \or % 24
             \xxmin\space \ymin \space m
             \xmin \space \ymin \space \xmin \space \yymin\space y
             \xmin \space \yymax\space m
             \xmin \space \ymax \space \xxmin\space \ymax \space y
           \or % 25
             \xxmax\space \ymax \space m
             \xmax \space \ymax \space \xmax \space \yymax\space y
             \xmax \space \yymin\space m
             \xmax \space \ymin \space \xxmax\space \ymin \space y
             \xxmin\space \ymin \space m
             \xmin \space \ymin \space \xmin \space \yymin\space y
             \xmin \space \yymax\space m
             \xmin \space \ymax \space \xxmin\space \ymax \space y
           \or % 26
             \xmax \space \yymin\space m
             \xmax \space \ymin \space \xxmax\space \ymin \space y
             \xmin \space \yymax\space m
             \xmin \space \ymax \space \xxmin\space \ymax \space y
           \or % 27
             \xxmax\space \ymax \space m
             \xmax \space \ymax \space \xmax \space \yymax\space y
             \xxmin\space \ymin \space m
             \xmin \space \ymin \space \xmin \space \yymin\space y
           \or % 28
           \fi
           \ifnum\mode>8
                                       S
           \else
             \ifnum\dostroke=\plusone  S \fi
             \ifnum\dofill  =\plusone  f \fi
           \fi
           Q}%
      \fi}%
   \wd\scratchbox#1\ht\scratchbox#2\dp\scratchbox#3\box\scratchbox
   \egroup}

%D \macros
%D   {doPDFstartgraymode,doPDFstopgraymode,
%D    doPDFstartrgbcolormode,doPDFstartcmykcolormode,doPDFstartgraycolormode,
%D    doPDFstopcolormode}
%D
%D In \PDF\ there are two color states, one for strokes and one
%D for fills. This means that we have to set the color in a
%D rather redundant looking way. Unfortunately this makes the
%D \PDF\ file much larger than needed. We can save few bytes
%D by not setting the stroke color. Due to zip compression we
%D only save a few percent.

% \newif\ifPDFstrokecolor \PDFstrokecolortrue

\def\doPDFstartgraymode#1%
%   {\PDFdirectcode{#1 g\ifPDFstrokecolor\space#1 G\fi}}
  {\PDFcode{#1 g\ifPDFstrokecolor\space#1 G\fi}}

\def\doPDFstopgraymode
%   {\PDFdirectcode{0 g\ifPDFstrokecolor\space0 G\fi}}
  {\PDFcode{0 g\ifPDFstrokecolor\space0 G\fi}}

\def\doPDFstartrgbcolormode#1#2#3%
%   {\PDFdirectcode{#1 #2 #3 rg\ifPDFstrokecolor\space#1 #2 #3 RG\fi}}
  {\PDFcode{#1 #2 #3 rg\ifPDFstrokecolor\space#1 #2 #3 RG\fi}}

\def\doPDFstartcmykcolormode#1#2#3#4%
%   {\PDFdirectcode{#1 #2 #3 #4 k\ifPDFstrokecolor\space#1 #2 #3 #4 K\fi}}
  {\PDFcode{#1 #2 #3 #4 k\ifPDFstrokecolor\space#1 #2 #3 #4 K\fi}}

\def\doPDFstartgraycolormode#1%
%   {\PDFdirectcode{#1 g\ifPDFstrokecolor\space#1 G\fi}}
  {\PDFcode{#1 g\ifPDFstrokecolor\space#1 G\fi}}

\def\doPDFstartspotcolormode#1#2% redefining spotcolors is not possible anyway
  {\ifundefined{pdf:scs:#2}%
     \bgroup
     \getcommacommandsize[#2]%
     \ifcase\commalistsize\or
       \setxvalue{pdf:scs:#2}{#2 SCN #2 scn}% \setxvalue{pdf:scs:#2}{#2 SC #2 sc}%
     \else
       \let\PDFspotcolorspecs\empty
       \def\dospotcolorcommand##1{\edef\PDFspotcolorspecs{\PDFspotcolorspecs##1\space}}%
       \processcommacommand[#2]\dospotcolorcommand
       \setxvalue{pdf:scs:#2}{\PDFspotcolorspecs SCN \PDFspotcolorspecs scn}%
     \fi
     \egroup
   \fi
%    \PDFdirectcode{/#1 cs /#1 CS \PDFgetspotcolorspec{#2}}}
   \PDFcode{/#1 cs /#1 CS \PDFgetspotcolorspec{#2}}}

\def\PDFgetspotcolorspec#1%
  {\executeifdefined{pdf:scs:#1}\empty} % better no default than one with too less args

\def\doPDFstartnonecolormode
%   {\PDFdirectcode{/None CS 1 SC /None cs 1 sc}}
  {\PDFcode{/None CS 1 SC /None cs 1 sc}}

\def\doPDFstopcolormode
%   {\PDFdirectcode{0 g\ifPDFstrokecolor\space0 G\fi}}
  {\PDFcode{0 g\ifPDFstrokecolor\space0 G\fi}}

%D We need to register the spot colors and their fallbacks.

% we cannot use /DeviceN since GS <=7.21 breaks on it
% and Jaws does not handle it at all {[/DeviceN [/All|/None]
% /Device#2 \PDFobjref\pdflastobj]} so we use separation
% colors that work and print ok

\def\doPDFregistersomespotcolor#1#2#3#4% implemented in the driver
  {\writestatus\m!systems{missing spot color definition}\wait}

\def\doPDFregisternonecolor % internal command
  {\doPDFregistergrayspotcolor{None}{1}%
   \globallet\doPDFregisternonecolor\relax}

\def\doPDFregisterspotcolorname#1#2% implemented in the driver
  {}

\definespecial\doregisterspotcolorname{\doPDFregisterspotcolorname}

\def\dodoPDFregisterrgbspotcolor#1#2#3#4#5#6#7% name noffractions names p's r g b
  {\doPDFregistersomespotcolor{#1}{#2}{#3}{#4}{RGB}{0.0 1.0 0.0 1.0 0.0 1.0}%
     {\ifcase#2\or dup #5 mul exch dup #6 mul exch #7 mul\else#5 #6 #7\fi}}

\def\dodoPDFregistercmykspotcolor#1#2#3#4#5#6#7#8% name noffractions names p's c m y k
  {\doPDFregistersomespotcolor{#1}{#2}{#3}{#4}{CMYK}{0.0 1.0 0.0 1.0 0.0 1.0 0.0 1.0}%
     {\ifcase#2\or dup #5 mul exch dup #6 mul exch dup #7 mul exch #8 mul\else #5 #6 #7 #8\fi}}

\def\dodoPDFregistergrayspotcolor#1#2#3#4#5% name noffractions names p's s
  {\doPDFregistersomespotcolor{#1}{#2}{#3}{#4}{Gray}{0.0 1.0}%
     {\ifcase#2\or #5 mul\else #5\fi}}

% \let\doPDFregisterrgbspotcolor \dodoPDFregisterrgbspotcolor
% \let\doPDFregistercmykspotcolor\dodoPDFregistercmykspotcolor
% \let\doPDFregistergrayspotcolor\dodoPDFregistergrayspotcolor

\def\doPDFregisterrgbspotcolor#1#2#3#4#5#6#7% name noffractions names p's r g b
  {\ifRGBsupported
     \dodoPDFregisterrgbspotcolor{#1}{#2}{#3}{#4}{#5}{#6}{#7}%
   \else
     \edef\@@cl@@r{#5}\edef\@@cl@@g{#6}\edef\@@cl@@b{#7}%
     \ifCMYKsupported
       \convertRGBtoCMYK\@@cl@@r\@@cl@@g\@@cl@@b
       \dodoPDFregistercmykspotcolor{#1}{#2}{#3}{#4}\@@cl@@c\@@cl@@m\@@cl@@y\@@cl@@k
     \else
       \convertRGBtoGRAY\@@cl@@r\@@cl@@g\@@cl@@b
       \dodoPDFregistergrayspotcolor{#1}{#2}{#3}{#4}\@@cl@@s
     \fi
   \fi}

\def\doPDFregistercmykspotcolor#1#2#3#4#5#6#7#8% name noffractions names p's c m y k
  {\ifCMYKsupported
     \dodoPDFregistercmykspotcolor{#1}{#2}{#3}{#4}{#5}{#6}{#7}{#8}%
   \else
     \edef\@@cl@@c{#5}\edef\@@cl@@m{#6}\edef\@@cl@@y{#7}\edef\@@cl@@k{#8}%
     \ifRGBsupported
       \convertCMYKtoRGB\@@cl@@c\@@cl@@m\@@cl@@y\@@cl@@k
       \dodoPDFregisterrgbspotcolor{#1}{#2}{#3}{#4}\@@cl@@r\@@cl@@g\@@cl@@b
     \else
       \convertCMYKtoGRAY\@@cl@@c\@@cl@@m\@@cl@@y\@@cl@@k
       \dodoPDFregistergrayspotcolor{#1}{#2}{#3}{#4}\@@cl@@s
     \fi
   \fi}

\def\doPDFregistergrayspotcolor{\dodoPDFregistergrayspotcolor}

%D New and very experimental.

\def\doPDFregistercmykindexcolor#1#2#3#4#5#6#7#8% name noffractions names p's c m y k
  {\doPDFregistersomeindexcolor{#1}{#2}{#3}{#4}{CMYK}{0.0 1.0 0.0 1.0 0.0 1.0 0.0 1.0}%
     {dup #5 mul exch dup #6 mul exch dup #7 mul exch #8 mul}}

\def\doPDFregisterrgbindexcolor#1#2#3#4#5#6#7% name noffractions names p's r g b
  {\doPDFregistersomeindexcolor{#1}{#2}{#3}{#4}{RGB}{0.0 1.0 0.0 1.0 0.0 1.0}%
     {dup #5 mul exch dup #6 mul exch #7 mul}}

\def\doPDFregistergrayindexcolor#1#2#3#4#5% name noffractions names p's s
  {\doPDFregistersomeindexcolor{#1}{#2}{#3}{#4}{Gray}{0.0 1.0}%
     {pop}}

\let\checkpredefinedcolor\predefineindexcolor % we need an index in order to negate bitmaps

\def\doPDFregisterfigurecolor#1% always an index color
  {\dogetobjectreference
     {PDFIX}
     {\internalspotcolorname{#1}}
     \PDFimagecolorreference}

%D \macros
%D   {doPDFstartrotation,doPDFstoprotation}
%D
%D Rotating some text can be accomplished by setting the first
%D four elements of the transform matrix. We only support some
%D fixed angles. The \type{q}'s take care of grouping.

% The original:
%
% \def\doPDFstartrotation#1%
%   {\PDFcode{q}%
%    \processaction
%      [#1]
%      [ 90=>\PDFcode{ 0  1 -1  0 0 0 cm},
%       180=>\PDFcode{-1  0  0 -1 0 0 cm},
%       270=>\PDFcode{ 0 -1  1  0 0 0 cm},
%       360=>\PDFcode{ 1  0  0  1 0 0 cm}]}
%
% We cannot directly pass an angle, but have to calculate
% factors (rx and ry). As in the \METAPOST\ to \PDF\
% converter module we need to compensate the deformation
% by setting (sx and sy).
%
% Optimized but bigger:
%
% \def\doPDFstartrotation#1%
%   {\PDFcode{q}%
%    \processaction
%      [#1]
%      [  0=>\PDFcode{ 1  0  0  1 0 0 cm},
%        90=>\PDFcode{ 0  1 -1  0 0 0 cm},
%       180=>\PDFcode{-1  0  0 -1 0 0 cm},
%       270=>\PDFcode{ 0 -1  1  0 0 0 cm},
%       360=>\PDFcode{ 1  0  0  1 0 0 cm},
%        #1=>%\calculatecos{#1}% already calculated in core macro
%            %\calculatesin{#1}% already calculated in core macro
%            \edef\cos{\calculatedcos{#1}}%
%            \edef\sin{\calculatedsin{#1}}%
%            \PDFcode{\cos        \space %  cos
%                     \sin        \space %  sin
%                     \negated\sin\space % -sin
%                     \cos        \space %  cos
%                     0 0 cm}]}
%
% Since the sine and cosine values are preset and rounded we
% can use the next alternative without running into inaccuracies.

\def\doPDFstartrotation#1% grouped
  {\calculatecos{#1}% already calculated in core macro
   \calculatesin{#1}% already calculated in core macro
   \edef\cos{\calculatedcos{#1}}%
   \edef\sin{\calculatedsin{#1}}%
   \PDFcode{q \cos\space\sin\space\negated\sin\space\cos\space0 0 cm}}

\def\doPDFstoprotation
  {\PDFcode{Q}}

%D \macros
%D   {doPDFstartscaling,doPDFstopscaling}
%D
%D Scaling is rather straightforward:

\def\@@PDFzeroscale{.0001}

\def\doPDFstartscaling#1#2% the test is needed because acrobat is bugged!
  {\PDFcode{q \ifdim#1\points=\zeropoint\@@PDFzeroscale\else#1\fi\space 0 0
              \ifdim#2\points=\zeropoint\@@PDFzeroscale\else#2\fi\space 0 0 cm}}

% \def\doPDFstartscaling#1#2% the test is needed because acrobat is bugged!
%   {\PDFcode{q\ifdim#1\points=\zeropoint\else\ifdim#2\points=\zeropoint\else
%             \space#1\space 0 0 #2\space 0 0 cm\fi\fi}}

\def\doPDFstopscaling
  {\PDFcode{Q}}

%D \macros
%D   {doPDFstartmirroring,doPDFstopmirroring}
%D
%D Mirroring is implemented in a similar way:

\def\doPDFstartmirroring
  {\PDFcode{-1 0 0 1 0 0 cm}}

\def\doPDFstopmirroring
  {\PDFcode{-1 0 0 1 0 0 cm}}

%D \macros
%D   {doPDFstartnegative,doPDFstopnegative}
%D
%D When producing output for an image setter, sometimes negative
%D output is needed.

\def\doPDFstartnegative
  {\ifx\initializePDFnegative\undefined\else
     \initializePDFnegative
%      \PDFdirectcode{/GSnegative gs}%
     \PDFcode{/GSnegative gs}%
   \fi}

\def\doPDFstopnegative
  {\ifx\initializePDFnegative\undefined\else
     \initializePDFnegative
%      \PDFdirectcode{/GSpositive gs}%
     \PDFcode{/GSpositive gs}%
   \fi}

%D \macros
%D   {doPDFstartoverprint,doPDFstopoverprint}
%D
%D Some printers like overprint more than knockout.

\def\doPDFstartoverprint
  {\ifx\initializePDFoverprint\undefined\else
     \initializePDFoverprint
%      \PDFdirectcode{/GSoverprint gs}%
     \PDFcode{/GSoverprint gs}%
   \fi}

\def\doPDFstopoverprint
  {\ifx\initializePDFoverprint\undefined\else
     \initializePDFoverprint
%      \PDFdirectcode{/GSknockout gs}%
     \PDFcode{/GSknockout gs}% wrong
   \fi}

%D Transparency support:

\newif\ifPDFtransparencysupported

\def\PDFtransparancydictionary#1#2#3% type fraction extras
  {<</Type /ExtGState
     /ca #2 /CA #2
     /BM /\ifcase#1 Normal\or Normal\or Multiply\or Screen\or
          Overlay\or SoftLight\or HardLight\or ColorDodge\or
          ColorBurn\or Darken\or Lighten\or Difference\or
          Exclusion\else Compatible\fi
     #3>>}

\def\dodoPDFstarttransparency#1#2%
  {\presetPDFtransparency{#1}{#2}%
   \PDFcode{\PDFtransparencyidentifier\space gs }}

\def\dodoPDFstoptransparency
  {\PDFcode{/Tr0 gs }}

\def\doPDFstarttransparency
  {\ifPDFtransparencysupported
     \global\let\doPDFstarttransparency\dodoPDFstarttransparency
     \global\let\doPDFstoptransparency \dodoPDFstoptransparency
     \initializetransparency
     \expandafter\doPDFstarttransparency
   \else
     \expandafter\gobbletwoarguments
   \fi}

% \let\doPDFstoptransparency\relax
%
% This is tricky: because a text stream is handled before
% the page body is built, we can run into stops that will
% match an outer start; however, the stop is needed in case
% of a text color: [text color text] [other color text] on a
% first page combined with color splitting will go wrong if
% we stick to the relaxing method.

\def\doPDFstoptransparency
  {\ifPDFtransparencysupported
     \initializetransparency
     \dodoPDFstoptransparency
   \fi}

%D These use:

\let\PDFtransparencyresetreference \empty
\let\PDFtransparencyresetidentifier\empty

\let\PDFtransparencyreference \empty
\let\PDFtransparencyidentifier\empty

\let\presetPDFtransparency \gobbletwoarguments
\let\initializetransparency\relax

%D New trickery:

\definespecial\dostartgraphicgroup{\PDFcode{q}}
\definespecial\dostopgraphicgroup {\PDFcode{Q}}

%D Even newer trickery:

\definespecial\dostartviewerlayer {\doPDFstartlayer}
\definespecial\dostopviewerlayer  {\doPDFstoplayer}
\definespecial\dodefineviewerlayer{\doPDFdefinelayer}

\let\PDFtextlayers\empty
\let\PDFpagelayers\empty
\let\PDFhidelayers\empty
\let\PDFvidelayers\empty

% \def\doPDFstartlayer#1{\PDFdirectcode{/OC /#1 BDC}}
% \def\doPDFstoplayer   {\PDFdirectcode        {EMC}}

\def\doPDFstartlayer#1{\PDFcode{/OC /#1 BDC}}
\def\doPDFstoplayer   {\PDFcode        {EMC}}

% resource -> prop -> mc's -> OCG|OCMD (nested)

% ocg:

% /Intent/Design

% ocmd

% /P /AllOn

% kan zelf ocmd bevatten

\def\doPDFdefinelayer#1#2#3#4#5% tag title visible type printable
%  {\driverreferenced \doPDFdictionaryobject{PDLN}{#1}
  {\doPDFdictionaryobject{PDLN}{#1}
     {/Type /OCG
      \ifcase#4 \or
        /Intent /Design % disable layer hiding by user
      \fi
      \ifnum#5=\zerocount
        /Usage << /Print << /PrintState /OFF >> >> % printable or not
      \fi
      /Name (#2)}%
   \doPDFgetobjectreference{PDLN}{#1}\PDFobjectreference
   \xdef\PDFtextlayers{\PDFtextlayers\space\PDFobjectreference}%
   \doifelse{#3}\v!start
     {\xdef\PDFvidelayers{\PDFvidelayers\space\PDFobjectreference}}%
     {\xdef\PDFhidelayers{\PDFhidelayers\space\PDFobjectreference}}%
  %\driverreferenced \doPDFdictionaryobject{PDLD}{#1}
   \doPDFdictionaryobject{PDLD}{#1}
     {/Type /OCMD
      /OCGs [\PDFobjectreference]}%
   \doPDFgetobjectreference{PDLD}{#1}\PDFobjectreference
   \xdef\PDFpagelayers{\PDFpagelayers\space /#1 \PDFobjectreference}}

\def\flushPDFtextlayers
  {\ifx\PDFtextlayers\empty \else
     \driverreferenced \doPDFarrayobject{PDF}{textlayers}{\PDFtextlayers}%
     \doPDFgetobjectreference{PDF}{textlayers}\!!stringa
     \ifx\PDFvidelayers\empty
       \def\!!stringb{[null]}%
     \else
       \driverreferenced \doPDFarrayobject{PDF}{videlayers}{\PDFvidelayers}%
       \doPDFgetobjectreference{PDF}{videlayers}\!!stringb
     \fi
     \ifx\PDFhidelayers\empty
       \def\!!stringc{[null]}%
     \else
       \driverreferenced \doPDFarrayobject{PDF}{hidelayers}{\PDFhidelayers}%
       \doPDFgetobjectreference{PDF}{hidelayers}\!!stringc
     \fi
     \doPDFaddtocatalog
       {/OCProperties
        << % display in menu
           /D  << /Order \!!stringa
                  /ON    \!!stringb
                  /OFF   \!!stringc >>
           % used properties
           /OCGs \!!stringa >>}%
     \globallet\flushPDFtextlayers\relax
   \fi}

\def\flushPDFpagelayers
  {\ifx\PDFpagelayers\empty \else
     \doPDFpageresource{/Properties <<\PDFpagelayers>>}%
   \fi}

\prependtoksonce \flushPDFpagelayers \to \everyshipout
\prependtoksonce \flushPDFtextlayers \to \everylastshipout

\def\PDFlayeractionlist{null}

\def\PDFexecutehidelayer   {/SetOCGState /State [/OFF    \PDFlayeractionlist]}
\def\PDFexecutevidelayer   {/SetOCGState /State [/ON     \PDFlayeractionlist]}
\def\PDFexecutetogglelayer {/SetOCGState /State [/Toggle \PDFlayeractionlist]}

\def\domakeviewerlayerlist#1%
  {\bgroup
   \globallet\PDFlayeractionlist\empty
   \def\docommand##1%
     {\doPDFgetobjectreference{PDLN}{##1}\PDFobjectreference
      \xdef\PDFlayeractionlist{\PDFlayeractionlist\space\PDFobjectreference}}%
   \processcommalist[#1]\docommand
   \egroup}

%D Something rather pdf dependent:

% #1 => 1=fill 2=stroke 3=strokedfill 4=invisible
% #2 => linewidth
% #3 => spacing (beware, one needs to set the hsize as well)

\def\doPDFstartfonteffect#1#2#3%
  {\ifdim#2>\zeropoint
     \PointsToBigPoints{#2}\ascii
%      \PDFdirectcode{\ascii\space w}%
     \PDFcode{\ascii\space w}%
   \fi
   \ifdim#3\points=\onepoint\else
     \scratchdimen#3\points
%      \PDFdirectcode{\withoutpt{\the\scratchdimen}\space Tc}%
     \PDFcode{\withoutpt{\the\scratchdimen}\space Tc}%
   \fi
%    \PDFdirectcode{\purenumber#1 Tr}}
   \PDFcode{\purenumber#1 Tr}}

\def\doPDFstopfonteffect
%   {\PDFdirectcode{1 w 0 Tc 0 Tr}}
  {\PDFcode{1 w 0 Tc 0 Tr}}

%D Handy for the \METAPOST\ to \PDF\ converter:

\newevery \everyPDFximage \relax
\newevery \everyPDFxform  \relax

\appendtoksonce
    \collectPDFresources
    \global\let\currentPDFresources\collectedPDFresources
\to \everyPDFxform

\let\collectedPDFresources\empty

\def\collectPDFresources
  {\doifobjectreferencefoundelse{FDF}{docushades}
     {\doPDFgetobjectreference{FDF}{docushades}\PDFobjectreference
      \xdef\collectedPDFresources{\collectedPDFresources/Shading    \PDFobjectreference}}\donothing
   \doifobjectreferencefoundelse{FDF}{docuextgstates}
     {\doPDFgetobjectreference{FDF}{docuextgstates}\PDFobjectreference
      \xdef\collectedPDFresources{\collectedPDFresources/ExtGState  \PDFobjectreference}}\donothing
   \doifobjectreferencefoundelse{FDF}{colorspaces}
     {\doPDFgetobjectreference{FDF}{colorspaces}\PDFobjectreference
      \xdef\collectedPDFresources{\collectedPDFresources/ColorSpace \PDFobjectreference}}\donothing
   \global\let\collectPDFresources\relax}

%D And that was all.

\stopspecials

\ifx\fullytransparentcolor\undefined \else

  \def\fullytransparentcolor
    {\doPDFregisternonecolor
     \doPDFstartnonecolormode}

  \let\doPDFstarttransparency\gobbletwoarguments
  \let\doPDFstoptransparency\relax

\fi

\loadmarkfile{spec-fdf}

\protect \endinput
